\section{Mise à jour cohérente}

\begin{frame}{}
    \setbeamercolor{block body}{bg=blue, fg=white}
    \begin{block}{}
        \Huge
        \centering
        \vspace{1em}
        Qualité et cohérence des données
        \vspace{1em}
    \end{block}
\end{frame}

\begin{frame}{Cohérence à base de règles}
    \begin{multline*}
        \mathcal{D} = \{
        PrescExam(Mia, \textit{x-ray}), ExamResult(Mia, \textit{x-ray}, \textit{join inflammation})
        \only<5>{,\\{\color{blue}SOSY(Mia, \textit{join inflammation})}}
        \}
    \end{multline*}

    \vfill
    \only<2>{Une base de données est cohérente si elle respecte un ensemble de règles prédéfini : \glspl{tgd}}

    \only<3->{
        \begin{itemize}
            \item[$c_1$] $PrescExam(X, Y) \to ExamResult(X, Y, Z)$
            \item<only@4,5>[$c_2$] $ExamResult(X, Y, Z) \to SOSY(X, Z)$
        \end{itemize}
        \vfill\centering
        $\mathcal{D}$ est
        \only<3,5>{\color{DarkGreen}Valide}%
        \only<4>{\color{red}Invalide}%
    }
\end{frame}

\begin{frame}{Politique de mises à jour}
    
    \begin{block}{}
        \begin{itemize}
            \item Comment maintenir la cohérence lors d'une mise à jour ?
                    \begin{itemize}
                        \item<only@2-> Refuser les mises à jour qui ne satisfont pas les contraintes
                        \item<only@3-> \emph{Mise à jour prioritaire} et s'assurer que les données respectent les contraintes spécifiées \cite{chabinConsistentUpdatingDatabases2020} 
                    \end{itemize}
            \item<only@5-> Comment gérer les variables existentielles ? \only<6->{\color{blue}valeurs nulles liées}
            \item<only@7-> Notre proposition est :
                \begin{itemize}
                    \item Incrémentale
                    \item Connectée à un \gls{sgbd} en utilisant des requêtes
                \end{itemize}
        \end{itemize}
    \end{block}

    \only<4->{\begin{multline*}
        \mathcal{D} = \{
        PrescExam(Mia, \textit{x-ray})
        \only<6->{,\\ExamResult(Mia, \textit{x-ray}, {\color{blue} N_1}), SOSY(Mia, {\color{blue} N_1})}
        \}\\
    \end{multline*}

    \vspace{-1em}

    \begin{itemize}
        \item[$c_1$] $PrescExam(X, Y) \to ExamResult(X, Y, Z)$
        \item[$c_2$] $ExamResult(X, Y, Z) \to SOSY(X, Z)$
    \end{itemize}}
\end{frame}

%\section{Mise à jour cohérente}
\begin{frame}{Chase : insertion}
    \begin{itemize}
        \item[$c_1$] $PrescExam(X, Y) \to ExamResult(X, Y, Z)$
        \item[$c_2$] $ExamResult(X, Y, Z) \to SOSY(X, Z)$
    \end{itemize}

    \begin{block}{}
        \begin{itemize}
            \item L'insertion de $PrescExam(Diego, RMI)$ déclenche la génération de nouveaux atomes\\
            \item<only@2-> Comment empêcher une génération infinie d'atomes ? \only<3->{Profondeur maximale}
        \end{itemize}

        \only<4->{\[
            \only<4,5>{Q_{chase}^{[c]}(X) \leftarrow body(c), \lnot head(c)}
            \only<6>{Q_{chase}^{[c_1]}(X) \leftarrow PrescExam(Diego, RMI), \lnot ExamResult(Diego, RMI, Z)}
            \only<7>{Q_{chase}^{[c_2]}(X) \leftarrow ExamResult(Diego, RMI, N_1), \lnot SOSY(Diego, N_1)}
        \]}
    \end{block}

    \only<5->{\begin{multline*}
    \mathcal{D} = \{
        PrescExam(Diego, RMI)
        \uncover<6->{, \color<only@6>{blue}ExamResult(Diego, RMI, N_1)}
        \uncover<7->{,\\\color<only@7>{blue}SOSY(Diego, N_1)}
    \}
    \end{multline*}}
\end{frame}

\begin{frame}{Chase : suppression}
    \begin{itemize}
        \item[$c_1$] $PrescExam(X, Y) \to ExamResult(X, Y, Z)$
        \item[$c_2$] $ExamResult(X, Y, Z) \to SOSY(X, Z)$
    \end{itemize}

    \begin{multline*}
        \mathcal{D} = \{
            PrescExam(Diego, RMI), \only<-3>{ExamResult(Diego, RMI, tumor)}\only<4->{\cancel{ExamResult(Diego, RMI, tumor)}}
            ,\\\only<1,3>{SOSY(Diego, tumor)}\only<2,4->{\cancel{SOSY(Diego, tumor)}}
            \only<5->{, ExamResult(Diego, RMI, N_1)}\only<6->{, SOSY(Diego, N_1)}
        \}
        \end{multline*}
\end{frame}

\begin{frame}{Simplification}
    \begin{block}{}
        La redondance signifie qu'il existe des informations équivalentes ou plus spécifiques.
        La simplification consiste à éliminer les redondances.

        \[
            \mathcal{D} = \{P(a, b), P(b, c), P(c, d)\}
        \]
    \end{block}
    \vfill
    \only<1-3>{\begin{center}
        \only<1>{$P(a, N_1)$ est redondant}
        \only<2>{$P(a, N_1), P(N_1, d)$ n'est pas redondant}
        \only<3>{$P(a, N_1), P(N_1, N_2), P(N_2, d)$ est redondant}
    \end{center}}
    \only<4->{
        Retrouver les partitions simplifiables
        \begin{itemize}
            \item<5-> Retrouver les valeurs nulles impactées par la mise à jour (NullBucket)
            \item<6-> Identifier les atomes liés par des valeurs nulles (LinkedNulls)
            \item<7-> Interroger la base en remplaçant les valeurs nulles par des variables\\
                \[P(a, N_1), P(N_1, b) \Rightarrow Q \gets P(a, x), P(x, b)\]
        \end{itemize}
    }
    \only<5>{\vfill\begin{columns}[t]
        \begin{column}{.25\textwidth}
            \centering
            Étant donné\\$A(a, b)$
        \end{column}
        \hfill
        \begin{column}{.25\textwidth}
            \centering
            $B(a, b)$\\ n'est pas compatible
        \end{column}
        \hfill
        \begin{column}{.25\textwidth}
            \centering
            $A(c, N_1)$\\ n'est pas compatible
        \end{column}
        \hfill
        \begin{column}{.25\textwidth}
            \centering
            $A(a, N_2)$\\ est compatible
        \end{column}
    \end{columns}}
    % \only<3->{\begin{align*}
    %     \only<3->{P(N_1, N_2)            & ~\to~ P(a, b)                      \\}
    %     \only<4->{P(N_1, N_2)            & ~\xcancel{\to}~ P(a, N_2)          \\}
    %     \only<5->{P(N_1, N_2), P(N_2, c) & ~\xcancel{\to}~ P(a, b), P(N_2, c) \\}
    %     \only<6->{P(N_1, N_2), P(N_2, c) & ~\to~ P(a, b), P(b, c)}
    % \end{align*}}
\end{frame}

\begin{frame}{Évaluation : Temps d'exécution par \acrshort{sgbd}}
    \centering
    \begin{tikzpicture}
        \tikzstyle{every node}=[font=\small]
        \begin{axis}[xmax=1000, bar width=.6em, xbar, every axis plot/.append style={fill}, axis x line=none, axis line style={draw=none}, tick style={draw=none}, symbolic y coords = {MySQL}, enlarge y limits = 0.5, restrict x to domain=0:*, nodes near coords, ytick=data, xlabel={Temps}, x unit=\si{\ms}, legend columns=-1, legend style={draw=none, at={(0.5,0)}, anchor=north}, width=.9\linewidth, height=.4\textheight]
            \addplot+[area legend] table [y=db, x=chase, col sep=comma] {time_mysql.csv};
            \addlegendentry{Chase};

            \addplot+[area legend] table [y=db, x=nullBucket, col sep=comma] {time_mysql.csv};
            \addlegendentry{NullBucket};

            \addplot+[area legend] table [y=db, x=partitions, col sep=comma, col sep=comma] {time_mysql.csv};
            \addlegendentry{LinkedNulls};

            \addplot+[area legend] table [y=db, x=simplifications, col sep=comma] {time_mysql.csv};
            \addlegendentry{Simplifications};
        \end{axis}
    \end{tikzpicture}

    \vfill
    \[
        \only<2->{A(N_1, \dots, N_2)}
        \only<3->{, B(N_2, \dots, N_3, \dots)}
        \only<5->{, B(\dots, N_4, \dots, N_3)}
        \only<5->{, C(N_3, \dots, N_5)}
    \]
    \begin{columns}[t]
        \begin{column}{.3\textwidth}
            \begin{tabular}{ccc}
                A \\
                \hline
                \dots & \dots & \dots \\
                $N_1$\tikzmark{n11} & \dots & $N_2$\tikzmark{n21} \\
                \dots & \dots & \dots
            \end{tabular}
        \end{column}
        \begin{column}{.3\textwidth}
            \begin{tabular}{cccc}
                B \\
                \hline
                \dots & \dots & \dots & \dots \\
                $N_2$\tikzmark{n22} & \dots & $N_3$\tikzmark{n31} & \dots \\
                \dots & \dots & \dots & \dots \\
                \dots & $N_4$\tikzmark{n41} & \dots & $N_3$\tikzmark{n32}
            \end{tabular}
            \vfill
        \end{column}
        \begin{column}{.3\textwidth}
            \begin{tabular}{ccc}
                C \\
                \hline
                \dots & \dots & \dots \\
                $N_3$\tikzmark{n33} & \dots & $N_5$\tikzmark{n51} \\
                \dots & \dots & \dots
            \end{tabular}
        \end{column}
    \end{columns}
    \begin{tikzpicture}[overlay, remember picture, shorten >=1.2em, shorten <=.5em, -latex, thick, red, bend left]
        \path<2-> (pic cs:n11) edge[bend left=20] (pic cs:n21);
        \path<3-> (pic cs:n21) edge (pic cs:n22);
        \path<4-> (pic cs:n22) edge (pic cs:n31);
        \path<5->[shorten >=.5em] (pic cs:n31) edge (pic cs:n32);
        \path<5-> (pic cs:n31) edge (pic cs:n33);
        \path<6->[shorten >=.5em] (pic cs:n32) edge (pic cs:n41);
        \path<6-> (pic cs:n33) edge (pic cs:n51);
    \end{tikzpicture}
\end{frame}

\section{Modélisation graphe}

\begin{frame}{Les graphes comme solution ?}
    \centering
    \begin{adjustbox}{width=.8\textwidth}
        \begin{tikzpicture}[shorten >=3pt,->,node distance=16em,on grid,every text node part/.style={align=center}]
            \node[draw] (a) {\textbf{:Atom}\\\textit{symbol}: $P$\\\textit{terms}: $\{t_1, \dots, t_n\}$};        % Atom
            \node[draw,circle,left = of a] (nc) {\textbf{:Element}\\\textbf{:Constant}\\\textit{value}: $t_i$};   % Element/Constant
            \node[draw,circle,right = of a] (nn) {\textbf{:Element}\\\textbf{:Null}\\\textit{value}: $t_j$};      % Element/Null
            \path
            (a) edge [] node[below, near start] {$0..*$} node[below, near end] {$1..*$} node[above] {\textbf{:P} \{\textit{rank}: $i$\}} (nc)
            (a) edge [] node[below, near start] {$0..*$} node[below, near end] {$1..*$} node[above] {\textbf{:P} \{\textit{rank}: $j$\}} (nn)
            ;
        \end{tikzpicture}
    \end{adjustbox}
\end{frame}

\begin{frame}{Exemple}
    \centering
    \begin{multline*}
        \mathcal{D} = \{ Pat(Mia), PrescExam(Mia, N_1), SOSY(Mia, pain),\\
        Diag(Mia, N_3, N_2), ResultExam(Mia, \textit{x-ray}, N_3) \}
    \end{multline*}
    \vfill
    \begin{adjustbox}{width=.8\linewidth}
        \begin{tikzpicture}[shorten >=3pt,->,node distance=4cm,on grid,every text node part/.style={align=center}]
            \node[draw,circle] (Mia) {$n_1$\\Mia};
            \node[draw,left = of Mia] (pat) {$n_{11}$\\Pat};
            \node[draw,circle,above = 2cm of pat] (n1) {$n_2$\\$N_1$};
            \node[draw,circle,below = 2cm of pat] (temp) {$n_3$\\x-ray};
            \node[draw,right = of n1] (presc) {$n_4$\\PrescExam};
            \node[draw,right = of presc] (sosy) {$n_5$\\SOSY};
            \node[draw,right = of temp] (res) {$n_6$\\ResultExam};
            \node[draw,circle,right = of res] (n4) {$n_7$\\$N_3$};
            \node[draw,circle,right = of sosy] (fever) {$n_8$\\pain};
            \node[draw,right = of Mia] (diag) {$n_9$\\Diag};
            \node[draw,circle,right = of diag] (n2) {$n_{10}$\\$N_2$};
            \path
            (pat) edge [] node[above] {$r_1$} node[below] {rank: 0} (Mia)
            (presc) edge [] node[right] {$r_2$} node[left] {rank: 0} (Mia)
            edge [] node[above] {$r_3$} node[below] {rank: 1} (n1)
            (sosy) edge [] node[left] {$r_4$} node[right] {rank: 0} (Mia)
            edge [] node[above] {$r_5$} node[below] {rank: 1} (fever)
            (diag) edge [] node[above] {$r_6$} node[below] {rank: 0} (Mia)
            edge [] node[right] {$r_7$} node[left] {rank: 1} (n4)
            edge [] node[above] {$r_8$} node[below] {rank: 2} (n2)
            (res) edge [] node[right] {$r_9$} node[left] {rank: 0} (Mia)
            edge [] node[above] {$r_{10}$} node[below] {rank: 1} (temp)
            edge [] node[above] {$r_{11}$} node[below] {rank: 2} (n4)
            ;
        \end{tikzpicture}
    \end{adjustbox}
\end{frame}

\begin{frame}{LinkedNulls}
    \begin{multline*}
        \mathcal{D} = \{ Pat(Mia), PrescExam(Mia, N_1), SOSY(Mia, pain),\\
        {\only<2->{\color{red}}Diag(Mia, N_3, N_2)},
        {\only<3>{\color{red}}ResultExam(Mia, \textit{x-ray}, N_3)}
        \}
    \end{multline*}
    \vfill
    \centering
    \begin{adjustbox}{width=.8\linewidth}
        \begin{tikzpicture}[shorten >=3pt,->,node distance=4cm,on grid,every text node part/.style={align=center}]
            \node[draw,circle] (Mia) {$n_1$\\Mia};
            \node[draw,left = of Mia] (pat) {$n_{11}$\\Pat};
            \node[draw,circle,above = 2cm of pat] (n1) {$n_2$\\$N_1$};
            \node[draw,circle,below = 2cm of pat] (temp) {$n_3$\\x-ray};
            \node[draw,right = of n1] (presc) {$n_4$\\PrescExam};
            \node[draw,right = of presc] (sosy) {$n_5$\\SOSY};
            \node[draw,circle,right = of sosy] (fever) {$n_8$\\pain};

            \only<1-2>{\node[draw,right = of temp] (res) {$n_6$\\ResultExam};}
            \only<3>{\node[draw,right = of temp,red] (res) {$n_6$\\ResultExam};}

            \only<1>{\node[draw,circle,right = of res] (n4) {$n_7$\\$N_3$};}
            \only<2-3>{\node[draw,circle,right = of res,red] (n4) {$n_7$\\$N_3$};}

            \only<1>{\node[draw,right = of Mia] (diag) {$n_9$\\Diag};}
            \only<2-3>{\node[draw,right = of Mia,red] (diag) {$n_9$\\Diag};}

            \node[draw,circle,right = of diag,red] (n2) {$n_{10}$\\$N_2$};

            \path (pat) edge [] node[above] {$r_1$} node[below] {rank: 0} (Mia);
            \path (presc) edge [] node[right] {$r_2$} node[left] {rank: 0} (Mia);
            \path (presc) edge [] node[above] {$r_3$} node[below] {rank: 1} (n1);
            \path (sosy) edge [] node[left] {$r_4$} node[right] {rank: 0} (Mia);
            \path (sosy) edge [] node[above] {$r_5$} node[below] {rank: 1} (fever);
            \path (diag) edge [] node[above] {$r_6$} node[below] {rank: 0} (Mia);
            \path (res) edge [] node[right] {$r_9$} node[left] {rank: 0} (Mia);
            \path (res) edge [] node[above] {$r_{10}$} node[below] {rank: 1} (temp);

            \only<1>{\path (diag) edge [] node[right] {$r_7$} node[left] {rank: 1} (n4);}
            \only<2-3>{\path (diag) edge [red] node[right] {$r_7$} node[left] {rank: 1} (n4);}

            \only<1>{\path (diag) edge [] node[above] {$r_8$} node[below] {rank: 2} (n2);}
            \only<2-3>{\path (diag) edge [red] node[above] {$r_8$} node[below] {rank: 2} (n2);}

            \only<1-2>{\path (res) edge [] node[above] {$r_{11}$} node[below] {rank: 2} (n4);}
            \only<3>{\path (res) edge [red] node[above] {$r_{11}$} node[below] {rank: 2} (n4);}
            ;
        \end{tikzpicture}
    \end{adjustbox}
\end{frame}

\begin{frame}{Évaluation : Temps d'exécution par \acrshort{sgbd}}
    \centering
    \begin{tikzpicture}
        \tikzstyle{every node}=[font=\small]
        \begin{axis}[xmax=1000, bar width=.6em, xbar, every axis plot/.append style={fill}, axis x line=none, axis line style={draw=none}, tick style={draw=none}, symbolic y coords = {MySQL,Neo4J}, enlarge y limits = 0.5, restrict x to domain=0:*, nodes near coords, ytick=data, xlabel={Temps}, x unit=\si{\ms}, legend columns=-1, legend style={draw=none, at={(0.5,0)}, anchor=north}, width=.9\linewidth, height=.6\textheight]
            \addplot+[area legend] table [y=db, x=chase, col sep=comma] {time_per_db.csv};
            \addlegendentry{Chase};

            \addplot+[area legend] table [y=db, x=nullBucket, col sep=comma] {time_per_db.csv};
            \addlegendentry{NullBucket};

            \addplot+[area legend] table [y=db, x=partitions, col sep=comma, col sep=comma] {time_per_db.csv};
            \addlegendentry{LinkedNulls};

            \addplot+[area legend] table [y=db, x=simplifications, col sep=comma] {time_per_db.csv};
            \addlegendentry{Simplifications};
        \end{axis}
    \end{tikzpicture}

    \begin{block}{Travaux futures}
        \begin{itemize}
            \item Les bases de données graphes n'offrent pas de performances intéressantes (chase)
            \item Modélisations mixtes relationnelles et basées sur les graphes \cite{hassanGRFusionGraphsFirstClass2018}
        \end{itemize}
    \end{block}
\end{frame}
