\documentclass[english,french,table,aspectratio=43]{beamer}

\pdfinclusioncopyfonts=1
\pdfgentounicode=1
\pdfminorversion=5 
\pdfcompresslevel=9
\pdfobjcompresslevel=2

%%%%%%%%%%%%%%%%%%%%%%%%%%%%%%%%%%%%%%%%%%%%%%%%%%%%%%%%%%%%%%%%%%%%%
% Packages :
\usepackage{etex}
\usepackage[utf8]{inputenc}          % Attention encodage => UTF 8
\usepackage[T1]{fontenc}             % Codage de caractères T1
\usepackage{lmodern}
\usepackage{cmap}

\rmfamily

% Declare that Latin Modern Roman (lmr) should take
% its bold (b) and bold extended (bx) weight, and small capital (sc) shape, 
% from the corresponding Computer Modern Roman (cmr) font, for the T1 font encoding.
\DeclareFontShape{T1}{lmr}{b}{sc}{<->ssub*cmr/bx/sc}{}
\DeclareFontShape{T1}{lmr}{bx}{sc}{<->ssub*cmr/bx/sc}{}

\usepackage{xspace}
\usepackage{babel}
\usepackage[fixlanguage]{babelbib}
\usepackage[autolanguage]{numprint}
\usepackage[sectionbib]{bibunits}
\usepackage{hyphenat}
\usepackage{csquotes}
\usepackage{subfiles}
\usepackage{amsmath}
\usepackage{amssymb}
\usepackage{amsthm}
\usepackage{mathtools}
\usepackage{latexsym}

\usepackage{circledsteps}
\usepackage{orcidlink}
\usepackage{adjustbox}
%\usepackage{ragged2e}
\usepackage{graphicx}
\usepackage{hyperref}
\usepackage{multirow}
\usepackage{booktabs}
\usepackage{bookmark}
\usepackage{listings}
\usepackage{siunitx}
\usepackage{framed}
\usepackage{cancel}
\usepackage{color}
\usepackage{url}
\usepackage{doi}

\usepackage{tikz}
\usepackage{tikz-dependency}
%\usepackage{tikz-qtree}
%\usepackage{tikz-qtree-compat}
\usepackage{forest}

\usepackage{pgfplots}
\usepackage{pgfplotstable}

\usepackage[svgnames]{xcolor}
%\usepackage[inline]{enumitem}
\usepackage[acronym,shortcuts,translate=babel]{glossaries}
\usepackage[linesnumbered,ruled,vlined,boxed,frenchkw,onelanguage]{algorithm2e}
\usepackage[type={CC},modifier={by-nc-nd},lang=french,version={4.0}]{doclicense}

%\usepackage[a-3u,ua,x-4]{pdfx}

% Bib
\selectbiblanguage{french}

% Algorithm2e
\renewcommand{\CommentSty}[1]{\footnotesize\ttfamily\textcolor{Green}{#1}}
\renewcommand{\KwSty}[1]{\textnormal{\textcolor{blue}{\bfseries #1}}\unskip}
\renewcommand{\FuncSty}[1]{\textsf{#1}}
\renewcommand{\ProcFnt}[1]{\textsf{#1}}

\NewCommandCopy{\legacyunderscore}{\_}
\renewcommand{\_}{\ifincsname_\else\legacyunderscore\fi}

\makeatletter
\def\@algocf@pre@ruled{\begin{shaded}}%
\def\@algocf@post@ruled{\end{shaded}\relax}%
\makeatother

\colorlet{shadecolor}{WhiteSmoke}

% Tikz
\usetikzlibrary{
  automata, graphs, positioning, fit, tikzmark,
  shapes, shapes.multipart, arrows.meta, quotes
}
\usetikzmarklibrary{listings}

% \drawBrace[xshift]{beginningNode}{endingNode}{comment}
% This command draws a brace between two tikzmarks, to their right, no matter which
% one is the rightmost, and includes a node midway the brace, to write the comment.
\newcommand*{\drawBrace}[4][.5em]{%
    \node[draw=none, fit={(#2) (#3)}] (rectg) {};
    \draw [decoration={brace,amplitude=0.5em},decorate,thick] ([xshift=#1]rectg.north east) -- ([xshift=#1]rectg.south east) node[midway,right,xshift=1em]{#4};
}%

\newcommand*{\drawCircle}[1]{%
    \raisebox{-.5ex}{\tikz{\draw[fill=#1,line width=.5pt] circle[radius=1ex];}}%
}%

\tikzset{
  labeled node/.style={
    rectangle split, rectangle split parts=2,
    rectangle split draw splits=false,
    rounded corners,
    very thick,
    draw=blue!50,
    rectangle split part fill={blue!50, white},
    align=left,
    font=\sffamily\bfseries\boldmath\small
  },
  labeled edge/.style={
    rectangle split, rectangle split parts=1,
    rectangle split draw splits=false,
    rounded corners,
    very thick,
    draw=orange!60,
    rectangle split part fill={orange!60, white},
    align=left,
    font=\sffamily\bfseries\boldmath\small
  },
  labeled property edge/.style={
    rectangle split, rectangle split parts=2,
    rectangle split draw splits=false,
    rounded corners,
    very thick,
    draw=orange!60,
    rectangle split part fill={orange!60, white},
    align=left,
    font=\sffamily\bfseries\boldmath\small
  },
  every two node part/.style={font=\sffamily\small},
}

\tikzstyle{flow-node} = [text centered, minimum height=2em, minimum width=4em, text width=4em, draw]
\tikzstyle{flow-start} = [rectangle, rounded corners, flow-node]
\tikzstyle{flow-process} = [rectangle, text centered, minimum height=2em, text width=7em, draw]
\tikzstyle{flow-data} = [trapezium, trapezium left angle=70, trapezium right angle=110, flow-node]
\tikzstyle{flow-decision} = [diamond, aspect=2.5, flow-node]
\tikzstyle{flow-arrow} = [thick, ->, >=latex, rounded corners]

% Pgf
\usepgfplotslibrary{units}
\usepgfplotslibrary{colorbrewer}
\usepgfplotslibrary{groupplots}

\pgfplotsset{
    compat=newest,
    cycle list/Paired,
    scaled x ticks=false,
    scaled y ticks=false
}

% Listing
\renewcommand{\lstlistingname}{Extrait de code}
\renewcommand{\lstlistlistingname}{Liste des extraits de code}

\lstset{
  commentstyle=\color{Green},
  keywordstyle=\color{blue}\bfseries,
  stringstyle=\color{teal},
  backgroundcolor=\color{WhiteSmoke},
  numberstyle=\footnotesize\color{gray},
  basicstyle=\ttfamily\small,
  breakatwhitespace=true,
  breaklines=true,
  captionpos=b,
  keepspaces=true,
	stepnumber=1,
  numbers=left,
  numbersep=5pt,
  showspaces=false,
  showstringspaces=false,
  showtabs=false,
  tabsize=2,
  frame=lines
}

\lstdefinelanguage{cypher}{
	morekeywords={
		MATCH, OPTIONAL, WHERE, NOT, AND, OR, XOR, RETURN, DISTINCT, ORDER, BY,
    ASC, ASCENDING, DESC, DESCENDING, UNWIND, AS, UNION, WITH, ALL, CREATE,
    DELETE, DETACH, REMOVE, SET, MERGE, SET, SKIP, LIMIT, IN, CALL, CASE, WHEN, EXISTS,
    INDEX, DROP, UNIQUE, CONSTRAINT, EXPLAIN, PROFILE, START, FOREACH, GROUP, HAVING},
	sensitive=true,
	morecomment=[l]{//},
	morecomment=[s]{/*}{*/},
	morestring=[b]{"},
	literate=*
	  {<<}{\color{instruction}\guillemotleft{}}{1}
	  {>>}{\textcolor{instruction}{\guillemotright{}}\color{black}}{1}
}

\lstdefinelanguage{sparql}{
	morekeywords={SELECT, DISTINCT, WHERE, OPTIONAL, FILTER, NOT, EXISTS, MINUS, sameTerm, bound},
}

\mode<presentation> {
    \usetheme{Madrid}
    \useoutertheme[subsection=false]{miniframes}
    \setbeamertemplate{frametitle}[default][center]
    \setbeamertemplate{navigation symbols}{}
}

\makeglossaries          
\loadglsentries{../these/glossaries}

\date{1er Juillet, 2024}
\title[Construction automatique de BD m\'edicale]{Construction automatique de bases de donnée pour le domaine m\'edical: Int\'egration de texte et maintien de la coh\'erence}
\author[Nicolas Hiot]{
    Nicolas Hiot\inst{1,2}
}
\institute[]{
    \inst{1}LIFO -- Universit\'e d'Orl\'eans, INSA CVL, France\qquad
    \inst{2}EnnovLabs -- Ennov, France\\
    \url{nicolas.hiot@etu.univ-orleans.fr}\qquad
    \url{nhiot@ennov.com}
}
\titlegraphic{
    \includegraphics[height=1cm]{../logos/logoOrleans2022.png}
    \hspace{.8cm}
    \includegraphics[height=1cm]{../logos/logoLIFO.png}
    \hspace{.8cm}
    \includegraphics[height=.9cm]{../logos/logoEnnov.png}
    \hspace{.8cm}
    \includegraphics[height=.8cm]{../logos/logoDOING.png}
}

% \AtBeginSection{
%   \begin{frame}{Sommaire}
%     \tableofcontents[currentsection]
%   \end{frame}
% }

\begin{document}

\frame{\titlepage}

\section{Introduction}

% \begin{frame}{Contexte}
%   Textes médicaux : cas cliniques, pharmacovigilance, cosméto-vigilance, etc.\pause
%   \begin{description}
%       \item[Objectif] Exploiter ces données en les rendant interrogeables\pause
%       \item[Solution] Construire automatiquement une base de données\pause
%           \begin{itemize}
%               \item Structurer automatiquement les information du texte
%               \item Garantir la qualité et la cohérence des données
%           \end{itemize}
%   \end{description}
% \end{frame}

\begin{frame}{Données textuelles médicales}
  \begin{itemize}
      \item Le domaine médical génère en quantité des données textuelles difficiles à exploiter efficacement.
      \begin{itemize}
          \item Cas cliniques
          \item Pharmacovigilance
          \item Cosmetovigilance
      \end{itemize}
      \vfill
      \item Décrivent l'historique des patients:
      \begin{itemize}
          \item Diagnostiques
          \item Traitements et réactions
          \item Antécédents médicaux
      \end{itemize}
      \vfill
      \item Importance pour la prise de décision clinique:
      \begin{itemize}
          \item Meilleure compréhension des cas complexes
          \item Amélioration de la qualité des soins
          \item Détection précoce des effets indésirables
      \end{itemize}
  \end{itemize}
\end{frame}

\begin{frame}{Axes de Recherches}
  \begin{itemize}
      \item Proposition : Construire automatiquement une base de données à partir d'un corpus textuel médical
      \item Objectifs :
      \begin{itemize}
          \item Faciliter l'accès à des informations précises et structurées
          \item Permettre l'interrogation et l'exploitation defficaces des données
      \end{itemize}
      \vfill\pause
      \item Axes de Recherches :
      \begin{itemize}
          \item Extraction et Structuration automatique de l'information
          \item Assurer de la qualité et de la cohérence des données:
      \end{itemize}
  \end{itemize}
\end{frame}

% \begin{frame}{Contexte}
%   \begin{block}{Genèse}
%     Textes médicaux : cas cliniques, pharmacovigilance, cosméto-vigilance, etc.\pause
%     \begin{description}
%         \item[Objectif] Exploiter ces données en les rendant interrogeables\pause
%         \item[Solution] Construire une base de données
%             \begin{itemize}
%                 \item Structurer les information pour les rendre exploitable
%                 \item Garantir la qualité et la cohérence des données
%             \end{itemize}
%     \end{description}
%     \end{block}

%   \pause

%   \begin{block}{Contraintes}
%     \begin{itemize}
%         \item Données spécifiques à chaque client
%         \item Absence de données annotées
%         \item Puissance de calcul et quantité de données limitées
%         \item Nécessité d'un système rapide et efficace
%         %\item Besoin d'explicabilité des résultats
%     \end{itemize}
%     \end{block}
% \end{frame}

\begin{frame}{Exemple}
    \centering

    \begin{displaycquote}[extrait de PubMed]{mandalDeepBlueDot2022}
        A female patient in the age group 55-60 years presented to us with blurring of vision in both eyes.
        On slit-lamp examination, numerous circular to oval fleck-like discrete blue opacities at the level of deep corneal stroma and Descemet's membrane was observed.
    \end{displaycquote}

    \pause

    \vfill

    \Huge$\Downarrow$\normalsize

    %\vfill

    \begin{figure}[htb]
        \small
        \begin{adjustbox}{max width=\linewidth}
            \begin{tikzpicture}[node distance=5em]
                \node[labeled node] (eyes) {Anatomy \nodepart{two} name: \emph{both eyes}};
                \node[labeled node, above=of eyes] (sosy) {Symptom \nodepart{two} name: \emph{blurring of vision}};
                \node[labeled node, right=8em of sosy] (Person) {Person \nodepart{two} name: \emph{unknown}\\age: \emph{55-60}};
                \node[labeled node, right=8em of Person] (exam) {Examination \nodepart{two} name: \emph{slit-lamp}};
                \node[labeled node, right=8em of exam] (result) {Symptom \nodepart{two} name: \emph{numerous circular}\\~~~~~~~~~~~\emph{to oval fleck-like}};
                \node[labeled node, below=of exam] (stroma) {Anatomy \nodepart{two} name: \emph{corneal stroma}};
                \node[labeled node, below=of result] (membrane) {Anatomy \nodepart{two} name: \emph{descemet membrane}};

                \path [very thick, -Latex]
                (Person) edge node[labeled edge, anchor=center] {HasSym} (sosy)
                (sosy) edge node[labeled edge, anchor=center] {ConcernsAnat} (eyes)
                (Person) edge node[labeled edge, anchor=center] {PassExam} (exam)
                (exam) edge node[labeled edge, anchor=center] {GivesRes} (result)
                (result) edge node[labeled edge, anchor=center] {ConcernsAnat} (stroma)
                (result) edge node[labeled edge, anchor=center] {ConcernsAnat} (membrane);
            \end{tikzpicture}
        \end{adjustbox}
    \end{figure}
\end{frame}

% \begin{frame}{Corpus médical}
%     \begin{block}{Corpus de cas cliniques en français (DEFT)} % \cite{grabar2020cas} 
%         \begin{itemize}
%             \item 100 fichiers dé-identifiés
%             \item Situations réelles et fictives
%             \item Imbrication d'entités
%             \item Annoté  en entités (8350 annotations), mais pas en relations
%         \end{itemize}
%     \end{block}

%     \begin{block}{Entités}
%         \begin{itemize}
%             \item Patient (anatomie)
%             \item Pratique clinique (examen, pathologie, signe ou symptôme)
%             \item Traitements (dose, durée, fréquence, mode, substance, traitement, valeur)
%             \item Temporalité (date, moment)
%         \end{itemize}
%     \end{block}
% \end{frame}

% \begin{frame}{Relations}
%     \begin{table}[htb]
%         \centering
%         \footnotesize
%         \resizebox{\textwidth}{!}{ 
%             \def\arraystretch{1.1}%
%             \begin{tabular}{p{2.2cm}p{3.3cm}p{8cm}}
%                 %\hline
%                 Relation                             & Paire d'entités                               & Exemples                                                                                                                                                                 \\ \hline\hline
%                 \multirow{2}{*}{\emph{means}}        & \multirow{2}{*}{\emph{mode}-\emph{substance}} & \multirow{2}{*}{\shortstack[l]{\emph{une} [\emph{crème}]{\textsuperscript{\emph{mode}}} \emph{de} [\emph{nistatin}]{\textsuperscript{\emph{sub}}} \emph{a été préscrite} \\
%                 'an [ointment]{\textsuperscript{\emph{mode}}} of [nistatin]{\textsuperscript{\emph{sub}}} was prescribed'}}                                                                                                                                                     \\
%                                                      &                                               &                                                                                                                                                                          \\
%                 \hline

%                 \multirow{2}{*}{\emph{measure\_of}}  & \emph{dose}-\emph{substance}                  & \multirow{2}{*}{\shortstack[l]{\emph{la} [\emph{doxorubicine}]{\textsuperscript{\emph{sub}}} \emph{à raison de} [\emph{37,5 mg/m2/dose}]{\textsuperscript{\emph{dose}}}  \\ '[\emph{37,5 mg/m2/dose}]{\textsuperscript{\emph{dose}}} of [doxorubicine]{\textsuperscript{\emph{sub}}}'}}  \\
%                                                      & \emph{value}-\emph{substance}                 &                                                                                                                                                                          \\
%                 \hline

%                 \multirow{4}{*}{\emph{accompagnies}} & \emph{pathology}-\emph{pathology}             & \multirow{4}{*}{\shortstack[l]{\emph{le patient présentait une} [\emph{fatigue importante}]{\textsuperscript{\emph{sosy}}} \emph{de même}                                \\\emph{qu'une} [\emph{hyperthermie}]{\textsuperscript{\emph{sosy}}} \\ 'the patient displayed [great tiredness]{\textsuperscript{\emph{sosy}}} as well \\ as [hyperthermia]{\textsuperscript{\emph{sosy}}}'}} \\
%                                                      & \emph{sosy}-\emph{sosy}                       &                                                                                                                                                                          \\
%                                                      & \emph{pathology}-\emph{sosy}                  &                                                                                                                                                                          \\
%                                                      & \emph{substance}-\emph{substance}             &                                                                                                                                                                          \\
%                 \hline

%                 \multirow{3}{*}{\shortstack[l]{\emph{reveals}/                                                                                                                                                                                                                  \\\emph{searches}/\\\emph{tests}}}  & \emph{exam}-\emph{value} & \multirow{3}{*}{\shortstack[l]{\emph{l'}[\emph{échographie abdominale et pelvienne}]{\textsuperscript{\emph{exam}}} \emph{révèle la} \\ \emph{présence d'une} [\emph{masse surrénalienne à droite}]{\textsuperscript{\emph{sosy}}} \\ 'the [abdominal and pelvic ultrasonography]{\textsuperscript{\emph{exam}}} reveals \\ a [growth in the right adrenal gland]{\textsuperscript{\emph{sosy}}}' }} \\
%                                                      & \emph{exam}-\emph{sosy}                       &                                                                                                                                                                          \\
%                                                      & \emph{exam}-\emph{pathology}                  &                                                                                                                                                                          \\
%                                                      &                                               &                                                                                                                                                                          \\
%                 \hline

%                 \multirow{4}{*}{\emph{location}}     & \emph{anatomy}-\emph{pathology}               & \multirow{4}{*}{\shortstack[l]{\emph{elle a subi une} [\emph{résection au niveau}                                                                                        \\ \emph{du} [\emph{lobe supérieur droit}]{\textsuperscript{\emph{anat}}}]{\textsuperscript{\emph{treat}}} \\ 'she underwent a [resection of the [upper right lobe]{\textsuperscript{\emph{anat}}}]{\textsuperscript{\emph{treat}}}'}} \\
%                                                      & \emph{anatomy}-\emph{sosy}                    &                                                                                                                                                                          \\
%                                                      & \emph{anatomy}-\emph{exam}                    &                                                                                                                                                                          \\
%                                                      & \emph{anatomy}-\emph{treatment}               &                                                                                                                                                                          \\
%                 %\hline
%             \end{tabular}
%         }
%     \end{table}
% \end{frame}

%\part{Intégration de données textuelles}

%\frame{\partpage}

\subfile{parts/struct}

%\part{Maintenance et cohérence d'une base de données incomplète}

%\frame{\partpage}

\subfile{parts/nulls}

% \part{Extraction d'informations}

% \frame{\partpage}

%\subfile{parts/tal}

\begin{frame}{Perspectives}
    \vfill
    \begin{itemize}
        \item Comment prendre en compte les contraintes dés la structuration automatique
        \vfill
        \item Construire une vue unifiée de multiples sources de données (IA, Journalisme des données, \dots)
    \end{itemize}
    \vfill
\end{frame}

\frame{\titlepage}

\begin{frame}[allowframebreaks]{References}
    \bibliographystyle{apalike}
    \bibliography{../these/biblio}
\end{frame}

\end{document}
