Ce chapitre est consacré à la description de notre méthode de mise à jour incrémentale sur des bases de données graphes.
Il débute par la liste des contributions (section~\ref{sec:update:contrib}) et continue sur la modélisation de la base (section~\ref{sec:update:db}) pour ensuite présenter nos solutions incrémentales pour réduire la redondance (section~\ref{sec:update:simplify}) et calculer les effets de bord (section~\ref{sec:update:chase}).
Après la présentation des procédures d'insertion et de suppression (section~\ref{sec:update:update}), nous présenterons des résultats expérimentaux (section~\ref{sec:update:evaluation}) en comparant une implémentation sur un \gls{sgbd} relationnel avant de conclure le chapitre (section~\ref{sec:update:conclusion}).

\section{Contributions}
\label{sec:update:contrib}
\subfile{chapter_2/contributions}

\section{Modélisation de la base de données sous Neo4J}
\label{sec:update:db}
\subfile{chapter_2/graph_design}

\section{Réduction incrémentale de la redondance}
\label{sec:update:simplify}
\subfile{chapter_2/core}

\section{Chase incrémental et effet de bords}
\label{sec:update:chase}
\subfile{chapter_2/chase}

\section{Mise à jour incrémentale}
\label{sec:update:update}
\subfile{chapter_2/update}

\section{Evaluation}
\label{sec:update:evaluation}
\subfile{chapter_2/evaluation}

\section{Conclusion}
\label{sec:update:conclusion}
\subfile{chapter_2/conclusion}
