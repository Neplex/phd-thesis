En travaillant avec des bases de données graphes, la proposition de mise à jour doit opérer en harmonie avec les \glspl{sgbd} tout en prenant en compte les contraintes propres à ces systèmes.
Bien que les travaux exposés dans cette thèse s'alignent sur la politique détaillée dans \cite{chabinConsistentUpdatingDatabases2020}, ils se distinguent par les éléments suivants :

\begin{description}[wide=0pt]
    \item[Conception de la base de données] ~
          \begin{enumerate}
              \item Les \glspl{sgbd} pour les \glspl{lpg} partent généralement de l'hypothèse que les graphes sont complets, ce qui ne correspond pas à la réalité des données souvent incomplètes.
                    L'approche adoptée par \gls{neo4j}, qui considère les valeurs nulles comme des valeurs inexistantes, s'avère insuffisante pour traiter les valeurs nulles liées.
                    Pour résoudre ce problème, une nouvelle conception de base de données est proposée, traitant les valeurs nulles comme une forme de constante.

              \item Notre modélisation vise à éliminer des problèmes d'efficacité constatés lors d'une implémentation sur un \gls{sgbd} relationnel.
                    Pour cela, la modélisation proposée permet une indexation des valeurs nulles, facilitant ainsi l'identification de tous les atomes directement ou indirectement liés à ces valeurs.
                    Cette approche simplifie grandement les opérations de manipulation des valeurs nulles impactées par une mise à jour pour le calcul du \gls{core}.

              \item Mise en place de requêtes capables d'identifier les parties du graphe de données concernées par la procédure de mise à jour cohérente.
                    Notamment pour le calcul du \gls{chase} et du \gls{core}.
                    Ces requêtes sont rendues efficaces par les choix de conceptions de la base de données.
          \end{enumerate}

    \item[Approche incrémentale] ~
          \begin{enumerate}
              \item Le calcul des mises à jour se limite aux contraintes et aux données spécifiquement concernées.
                    En contraste avec la méthodologie présentée dans \cite{chabinConsistentUpdatingDatabases2020}, qui exige la vérification de l'ensemble de la base de données pour chaque mise à jour.

              \item La simplification des valeurs est guidée par les valeurs nulles pouvant potentiellement être simplifiées en raison de la mise à jour.
                    Cette approche diffère de celle de \cite{chabinConsistentUpdatingDatabases2020}, où la simplification prend en compte l'ensemble des valeurs nulles indépendamment de la mise à jour.
          \end{enumerate}
\end{description}
