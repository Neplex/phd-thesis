Cette section constitue une vue d'ensemble des concepts et des notions fondamentales qui serviront de base aux chapitres suivants.
Dans un premier temps une définition de la \glsfirst{fol} ainsi que les concepts de contraintes et de règles dans ce contexte seront présentés.
Par la suite, nous aborderons les bases de données en adoptant le formalisme \gls{fol} et offrirons une brève introduction aux bases de données graphes.

\subsection{Logique du premier ordre}
\label{sec:update:pre:logic}

En \glsfirst{fol}, l'information se matérialise par un ensemble de faits.
Ces faits sont constitués de constantes représentant les individus, ainsi que de symboles de prédicats définissant les relations entre ces individus.
Par exemple, le fait $auteur(Alice)$ comprend le symbole de prédicat $auteur$ d'arité $1$ et la constante $Alice$.
Si l'on suppose que chaque symbole de prédicat détient une signification (ici, intuitivement claire), ce fait indique que $Alice$ est un auteur.
Au-delà des faits, des règles d'inférence peuvent être établies pour engendrer de nouveaux faits à partir de faits existants.
Cependant, notre attention se porte ici sur les contraintes.
Ces dernières consistent en des règles visant à spécifier les relations et la signification associée aux prédicats.
Par exemple, la contrainte exprimée par la règle $\forall x~auteur(x) \implies \exists y~ecrit(x,y)$ signifie que pour chaque individu $x$ considéré comme auteur ($auteur(x)$) dans l'ensemble de fait, celui-ci doit également contenir au moins une œuvre $y$ écrite par cet auteur ($ecrit(x,y)$).

Formellement, on utilise comme alphabet :
\begin{itemize}
    \item un ensemble de prédicats $P, Q, R, \dots$
    \item un ensemble de variables $x, y, z, \dots$
    \item un ensemble de constantes $a, b, c, \dots$
    \item un ensemble de valeurs nulles marquées $N_1, N_2, N_3, \dots$
    \item le quantificateur universel $\forall$ (quel que soit)
    \item le quantificateur existentiel $\exists$ (il existe)
\end{itemize}
Avec cet alphabet, on définit un atome $P(t_1, \dots, t_n)$ où $P$ est le symbole de prédicat d'arité $n$ et $t_1, \dots, t_n$ sont les termes.
Les termes peuvent être une variable, une constante ou une valeur nulle.
Un fait est un atome composé exclusivement de constantes, tandis qu'un atome instancié est un atome ne contenant aucune variable.
Étant donné un atome instancié $A$, $null(A)$ représente l'ensemble des valeurs nulles de $A$.

Un homomorphisme $h$ entre deux ensembles d'atomes $A$ et $B$ est une application surjective des termes de $A$ vers les termes de $B$, telle que :
\begin{enumerate*}[label=(\roman*)]
    \item si, $h(t) = t$, alors $t$ est une constante
    \item si, $P(t_1, \dots, t_n) \in A$, alors $P(h(t_1), \dots, h(t_n)) \in B$.
\end{enumerate*}
S'il existe un homomorphisme inverse $h^-$ tel que $h^-(h(t)) = t$ alors ont dit que $A$ est \emph{isomorphe} à $B$.

$\Phi$ est l'ensemble de toutes les formules $\phi$, existentiellement quantifiées, de la forme $(\exists X)(\varphi_1(X_1) \land \dots \land \varphi_n(X_n))$ où $X$ est l'ensemble des variables et $\varphi_i(X_i)$ est une formule atomique dont $X_i \subset X$ représente l'ensemble des variables libres.
$atoms(\phi)$ est l'ensemble des formules atomique de $phi$.
Un modèle $M$ d'une formule $\phi$ est l'ensemble des faits qui ont un homomorphisme de $atoms(\phi)$ vers $M$.
Étant donné deux formules $\phi_1$ et $\phi_2$ : $\phi_1 \implies \phi_2$ si tout modèle de $\phi_1$ est un modèle de $\phi_2$.
$\phi_1$ et $\phi_2$ sont \emph{équivalents}, noté $\phi_1 \iff \phi_2$, si et seulement si $\phi_1 \implies \phi_2$ et $\phi_2 \implies \phi_1$.
$\phi_1$ et \emph{plus simple que} $\phi_2$, noté $\phi_1 \sqsubseteq \phi_2$ si $\phi_1 \iff \phi_2$ et $atoms(\phi_1) \subseteq atoms(\phi_2)$.
Une simplification $\phi_1$ de $\phi_2$ est \emph{minimale} si $\phi_1 \sqsubseteq \phi_2$ et qu'il n'existe pas de $\phi_1'$ tel que $\phi_1' \sqsubset \phi_1$.
Par exemple, étant donné la formule $\phi = (\exists x, y)(P(a, x) \land P(a, y))$, les formules $(\exists x)(P(a, x))$ et $(\exists y)(P(a, y))$ sont équivalentes et sont toutes deux une simplification de $\phi$.
De plus, \cite{chabinConsistentUpdatingDatabases2020} démontre que si $\phi_1$ et $\phi_2$ sont deux simplifications minimales d'une même formule $\phi$ alors les ensembles $atoms(\phi_1)$ et $atoms(\phi_2)$ sont isomorphes.
La simplification minimale d'une formule $\phi$ est appelé le \gls{core} et s'écrit $core(\phi)$.

\subsection{Base de données}
\label{sec:update:pre:db}

L'\emph{instance} d'une base de données $\mathcal{D}$ peut être définie comme la Skolémisation d'une formule $\phi$, représentant un ensemble minimal d'atomes instanciés, pour lequel $core(\phi) = \phi$.
La Skolémisation d'une formule $\phi$, notée $Sk(\phi)$, revient à substituer l'ensemble des variables par des constantes ou des valeurs nulles, tout en éliminant chaque occurrence du quantificateur existentiel.

Une \emph{requête} effectuée sur une base de données est une formule logique de la forme $(\forall X)(Q(X) \gets \phi(X))$, où $X$ représente l'ensemble des variables libres de $\phi$.
La réponse à une requête effectuée sur une base de données $\mathcal{D}$ correspond à l'ensemble des homomorphismes $h$ tels que $h(\phi(X)) \subseteq \mathcal{D}$.

Une \emph{contrainte} (ou règle) sur une base de donnée est une dépendance génératrice de n-uplets, également appelée \glsfirst{tgd} en anglais, de la forme $(\forall X, Y)(body(X, Y) \to (\exists Z)head(X, Z))$.
Plus simplement écrit $body(X, Y) \to head(X, Z)$, où $X$, $Y$ et $Z$ représentent des ensembles de variables, $body(X, Y)$ est un ensemble d'atomes et $head(X, Z)$ est un atome.
Une contrainte $c$ est dite satisfaite par un ensemble $I$ d'atomes instanciés, noté $I \vDash c$ si pour tout homomorphisme $h$ tel que $h(body(c)) \subseteq I$ il existe un homomorphisme $h'$ tel que $h'(body(c)) = h(body(c))$ et $h'(head(c)) \in I$.
Intuitivement, cela signifie que pour qu'une contrainte soit satisfaite, pour toute instanciation du corps présente dans la base de données, la tête correspondante doit aussi être présente dans la base.
Une instance $I$ d'une base de données est dites \emph{cohérente} vis-à-vis d'un ensemble de contraintes $\mathbb{C}$, noté $I \vDash \mathbb{C}$ si pour toute contrainte $c$ dans $\mathbb{C}$, $I \vDash c$.

\subsection{Base de données graphes}
\label{sec:update:pre:graph}

Dans cette section, nous visons à fournir une introduction concise aux bases de données de graphes.
\cite{pokornyGraphDatabasesTheir2015} offre une vue d'ensemble des avantages et des inconvénients de ce type de bases de données.
Lorsque nous évoquons les bases de données de graphes, nous pouvons identifier deux grandes familles :
\begin{description}
    \item[Les bases de triplets] Notamment représentées par le modèle \glsfirst{rdf} développé par le \gls{w3c}.
          Ces bases enregistrent des triplets sous la forme $\langle x, P, y \rangle$, où chaque triplet représente un atome $P(x, y)$ et est constitué de trois éléments : $x$, le sujet représenté par l'\gls{uri} d'une ressource (un nœud), $P$ est un symbole de prédicat, et $y$, l'objet, peut être soit un \gls{uri}, soit un littéral (une valeur de propriétés).
          \gls{rdf} est interrogé en utilisant le langage \gls{sparql}.

    \item[Les Graphes de Propriétés] Aussi connus sous le nom de \glsfirst{lpg} en anglais, sont des bases de données composées de nœuds reliés par des arcs. Les deux sont labellisés et peuvent contenir des propriétés.
          Parmi les \gls{sgbd} de ce type, \gls{neo4j} est l'un des plus populaires.
          Notamment, \gls{neo4j} est à l'origine du langage de requête \gls{cypher}, qui sert de fondation au nouveau langage du standard \gls{iso} : \gls{gql}, actuellement en cours de développement.
\end{description}

Les bases de données graphes sont spécialement conçues pour simplifier l'interrogation de la structure des données.
Le modèle de la base de données (ou schéma, pour faire une analogie avec le modèle relationnel, bien qu'aucune règle structurelle ne soit imposée) est étroitement lié aux types de requêtes envisagées.
Dans le cadre de \gls{neo4j}, par exemple, les nœuds peuvent être étiquetés avec une ou plusieurs étiquettes, appelées \emph{labels}, tandis que les relations ne peuvent en porter qu'une, dénommée \emph{type} (si plusieurs étiquettes sont nécessaires pour une relation, cette dernière sera dupliquée).

% \paragraph{\acs{rdfs}}
\subsubsection{RDFS}
\gls{rdfs} fournit des relations pour la définition d'ontologies (ou vocabulaires) destinées à structurer des ressources \gls{rdf}.
Il permet de définir le schéma de la base : les classes (correspondant aux types de nœuds) et les propriétés (correspondant aux arêtes).
De plus, il offre la possibilité de spécifier des relations de subordination et de subsomption en établissant une hiérarchie entre les éléments au moyen des propriétés \textbf{subClass} et \textbf{subProp}.
\gls{rdfs} permet également la séparation de l'instance et du schéma.

\subsubsection{Interrogation de graphe avec Cypher}
\gls{cypher} est un langage déclaratif de requête conçu pour les graphes de propriétés.
Ce langage offre la facilité d'exprimer des requêtes en utilisant le mot-clé \verb|MATCH| pour la recherche de motifs, exprimés sous forme d'art ASCII.
Les requêtes sont formulées comme une séquence d'opérations, où le résultat d'une opération devient le paramètre de la suivante.
Même si les requêtes portent sur des graphes, les résultats des opérations sont toujours une table d'affectation des variables (avec autant de colonnes que de variables et autant de lignes que de réponses).

\begin{figure}
    \begin{subfigure}{\linewidth}
        \begin{adjustbox}{width=\linewidth}
            \begin{tikzpicture}[shorten >=2pt,thick,-Latex,node distance=3cm and 6cm,on grid]
                \node[labeled node] (eyes) {Anatomy \nodepart{two} name: \emph{both eyes}};
                \node[labeled node, above=of eyes] (sosy) {Symptom \nodepart{two} name: \emph{blurring of vision}};
                \node[labeled node, right=of sosy] (person) {Person \nodepart{two} name: \emph{unknown}\\age: \emph{55-60}};
                \node[labeled node, right=of person] (exam) {Examination \nodepart{two} name: \emph{slit-lamp}};
                \node[labeled node, right=of exam] (result) {Symptom \nodepart{two} name: \emph{numerous circular}\\~~~~~~~~~~~\emph{to oval fleck-like}};
                \node[labeled node, below=of exam] (stroma) {Anatomy \nodepart{two} name: \emph{corneal stroma}};
                \node[labeled node, below=of result] (membrane) {Anatomy \nodepart{two} name: \emph{descemet membrane}};

                \path
                (person) edge node[labeled edge, anchor=center] {HasSym} (sosy)
                (sosy) edge node[labeled edge, anchor=center] {ConcernsAnat} (eyes)
                (person) edge node[labeled edge, anchor=center] {PassExam} (exam)
                (exam) edge node[labeled edge, anchor=center] {GivesRes} (result)
                (result) edge node[labeled edge, anchor=center] {ConcernsAnat} (stroma)
                (result) edge node[labeled edge, anchor=center] {ConcernsAnat} (membrane);
            \end{tikzpicture}
        \end{adjustbox}
        \caption{Instance sous forme de graphe}
    \end{subfigure}
    \begin{subfigure}{\linewidth}
        \centering
        \begin{adjustbox}{varwidth=\linewidth}
            \centering
            \begin{tabular}{c|l}
                \multicolumn{2}{l}{\textbf{Anatomy}} \\
                \hline
                \hline
                id & name                            \\
                \hline
                0  & \emph{both eyes}                \\
                1  & \emph{corneal stroma}           \\
                2  & \emph{descemet membrane}
            \end{tabular}
            \quad
            \begin{tabular}{c|l}
                \multicolumn{2}{l}{\textbf{Symptom}}             \\
                \hline
                \hline
                id & name                                        \\
                \hline
                0  & \emph{blurring of vision}                   \\
                1  & \emph{numerous circular to oval fleck-like}
            \end{tabular}
            \quad
            \begin{tabular}{c|l}
                \multicolumn{2}{l}{\textbf{Examination}} \\
                \hline
                \hline
                id & name                                \\
                \hline
                0  & \emph{slit-lamp}
            \end{tabular}
            \quad
            \begin{tabular}{c|l|l}
                \multicolumn{2}{l}{\textbf{Person}} \\
                \hline
                \hline
                id & name           & age           \\
                \hline
                0  & \emph{unknown} & \emph{55-60}  \\
            \end{tabular}
            \quad
            \begin{tabular}{c|c}
                \multicolumn{2}{l}{\textbf{ConcernsAnat}} \\
                \hline
                \hline
                symptom & anatomy                         \\
                \hline
                0       & 0                               \\
                1       & 1                               \\
                1       & 2
            \end{tabular}
            \quad
            \begin{tabular}{c|c}
                \multicolumn{2}{l}{\textbf{GivesRes}} \\
                \hline
                \hline
                exam & symptom                        \\
                \hline
                0    & 1                              \\
            \end{tabular}
            \quad
            \begin{tabular}{c|c}
                \multicolumn{2}{l}{\textbf{PassExam}} \\
                \hline
                \hline
                person & exam                         \\
                \hline
                0      & 0                            \\
            \end{tabular}
            \quad
            \begin{tabular}{c|c}
                \multicolumn{2}{l}{\textbf{HasSym}} \\
                \hline
                \hline
                person & symptom                    \\
                \hline
                0      & 0                          \\
            \end{tabular}
        \end{adjustbox}
        \caption{Instance au format relationnel}
    \end{subfigure}
    \caption{Exemple d'instance d'une base de donnée graphe et son équivalent en relationnel}
    \label{fig:update:pre:graph}
\end{figure}

\begin{example}
    En considérant une instance de base de donnée graphe représente figure~\ref{fig:update:pre:graph}, la récupération de l'ensemble de l'ensemble des noms de partie du corps (\emph{Anatomy}) présents dans l'instance s'écrit, en \gls{cypher} : \verb|MATCH (x:Anatomy) RETURN x.name| ;
    Le \verb|MATCH| recherche le motif \verb|(x:Anatomy)|.
    Les parenthèses représentent un nœud, \verb|x| est le nom de la variable pour ce nœud et \verb|:Anatomy| indique que l'on cherche uniquement les nœuds qui ont l'étiquette \emph{Anatomy}.
    L'opération \verb|RETURN| termine la requête et renvoie la table d'instanciation suivante :

    \begin{center}
        \begin{tabular}{c|l}
              & x.name                   \\
            \hline
            0 & \emph{both eyes}         \\
            1 & \emph{corneal stroma}    \\
            2 & \emph{descemet membrane}
        \end{tabular}
    \end{center}

    La requête \verb|MATCH (x:Examination)-[y:GivesRes]->() RETURN x, y| retourne l'ensemble des nœuds \emph{Examination} qui sont dans une relation y ayant le type \emph{GivesRes} representé par \verb|-[y:GivesRes]->|.
    Ici, la relation est dirigée indiquant que l'on cherche uniquement les nœuds \emph{Examination} qui sont source de la relation.
    Le nœud de destination de la relation n'a aucune contrainte et est donc représenté uniquement par \verb|()|.
    Le langage \gls{cypher} permet également de construire des requêtes récursives.
    Par exemple, étant donné une relation \verb|knows| entre des personnes indiquant quelles se connaissent, pour obtenir l'ensemble des groupes de personnes où chaque membre connaît au moins une autre personne du groupe, on peut écrire en \gls{cypher} : \verb|MATCH (x:Person)-[:knows*]-(y:Person) RETURN COLLECT(DISTINCT y) AS persons|.
    Dans l'expression \verb|-[:knows*]-|, le symbole \verb|*| indique que la relation peut être traversée plusieurs fois de manière récursive.
    La requête est toujours assurée de se terminer, car \gls{neo4j} ne permet pas de traverser la même arête à plusieurs reprises pour un même motif.
    L'opération \verb|COLLECT(DISTINCT y)| permet de regrouper les instanciations de la variable \verb|y| en fonction des autres variables (équivalent à l'opération \verb|group by| en \gls{sql}), ici \verb|x|, en évitant les doublons (\verb|DISTINCT|).
    L'ensemble est retourné sous forme d'une liste.
    \gls{neo4j} autorise les types complexes dans les requêtes, mais aussi comme valeur de propriété.
\end{example}
