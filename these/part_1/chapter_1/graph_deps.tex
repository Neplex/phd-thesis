\cite{fanKeysGraphs2015,fanFunctionalDependenciesGraphs2016, fanDependenciesGraphs2019} sont une suite de travaux qui cherchent à définir des \glspl{gfd} comme les dépendances fonctionnelles.
Dans le modèle relationnel, les dépendances fonctionnelles sont définies sur une relation pour un ensemble d'attributs.
Pour les graphes, étant généralement dépourvus de schéma, il est nécessaire de définir le périmètre des dépendances fonctionnelles.
Les \glspl{gfd} doivent alors non seulement préciser la cohérence entre les valeurs des attributs, mais également détailler les structures topologiques qui les caractérisent et sont alors la combinaison :
\begin{itemize}
    \item d'une contrainte topologique, ou motif, sur le graphe qui définie la portée de la dépendance fonctionnelle et
    \item d'une dépendance fonctionnelle sur les attributs.
\end{itemize}

Comme pour les langages d'interrogation de graphes (cf. \gls{cypher}), un motif est un graphe contenant un ensemble fini de variables pour les nœuds, les arêtes et les propriétés.
Dans \cite{fanDependenciesGraphs2019}, un motif est défini comme un graphe dirigé $Q[\overline{x}] = (V_Q, E_Q, \lambda, \mu)$ où $V_Q$ est un ensemble fini de nœuds, $E_Q$ un ensemble fini d'arêtes, $\lambda : V_Q \cup E_Q \to L$ est une fonction partielle de labellisation, $\overline{x} \subset \textsc{var}$ est un ensemble de variables tel que $\lvert \overline{x} \rvert = \lvert V_Q \rvert$ ou $\lvert \overline{x} \rvert = \lvert E_Q \rvert$ et $\mu : \overline{x} \to V_Q \cup E_Q$ est une fonction bijective d'affectation de variables.
La recherche du motif $Q$ dans un graphe attribué $G = (V, E, L, F_A)$ ($F_A(v)$ est l'ensemble des propriétés $A_i = a_i$ où $v.A_i$ est une propriété de $v$ et $a_i$ est une constante) consiste a retrouvé l'ensemble des homomorphismes $h$ de ces variables tel que le graphe $h(Q)$ sois un sous graphe de $G$.
C'est-à-dire que pour chaque nœud $u \in V_Q$, $L_Q(u) = L(h(u))$ et pour chaque arête $e = (u, p, u')$ dans $Q$ il existe une arête $e' = (h(u), p, h(u'))$ dans $G$.

Une \glspl{gfd} est alors définie comme un tuple $(Q[\overline{x}], X \to Y)$ où $Q[\overline{x}]$ est le motif qui défini la contrainte topologique (le sous graphe sur lequel la contrainte s'applique) et $X$ et $Y$ sont deux ensembles de contraintes sur les variables dans $\overline{x}$.
Une contrainte est soit :
\begin{itemize}
    \item une contrainte sur une valeur de propriété : $x.A = c$ ;
    \item une contrainte d'équivalence entre deux propriétés : $x.A = y.B$ ;
    \item une contrainte d'identité qui signifie que deux nœuds ou arêtes sont les mêmes : $id(x) = id(y)$.
\end{itemize}
Où $x, y \in \overline{x}$ sont des variables, $A$ et $B$ sont des propriétés de nœuds ou d'arêtes et $c$ est une constante.
Les contraintes ont été étendues par la suite pour inclure des contraintes négatives avec un ensemble prédéfini de prédicats : $\neq, <, \le, >, \ge$.

\begin{figure}[ht]
    \centering
    \begin{subfigure}[b]{.53\textwidth}
        \centering
        \begin{tikzpicture}[->, -latex, auto, on grid, node distance=2cm, thick, main node/.style={draw}]
            \node[main node] (x) {Person: $x$};
            \node[main node] (y) [below of=1] {Document: $y$};
            \path[every node/.style={font=\sffamily\small}]
            (x) edge node {AuthorOf} (y);
        \end{tikzpicture}
        \caption*{$C_1 = (Q_1, y.type = ``thesis" \to x.status = ``PhD")$}
        \label{fig:gfd:q1}
    \end{subfigure}
    %\hfill
    \begin{subfigure}[b]{.37\textwidth}
        \centering
        \begin{tikzpicture}[->, -latex, auto, on grid, node distance=2.5cm, thick, main node/.style={draw}]
            \node[main node] (x) {Country: $x$};
            \node[main node] (y) [below left of=1] {City: $y$};
            \node[main node] (z) [below right of=1] {City: $z$};
            \path[every node/.style={font=\sffamily\small}]
            (y) edge node [left] {CapitalOf} (x)
            (z) edge node [right] {CapitalOf} (x);
        \end{tikzpicture}
        \caption*{$C_2 = (Q_2, \emptyset \to id(y) = id(z))$}
        \label{fig:gfd:q2}
    \end{subfigure}
    \caption{Exemple de dépendances fonctionnelles pour les graphes}
    \label{fig:gfd}
\end{figure}

\begin{example}
    La figure~\ref{fig:gfd} montre deux exemples de dépendance fonctionnelle.
    $C_1$ définie que tout auteur d'un document de type thèse est un doctorant où $Q_1$ est la contrainte topologique et $type$ est une propriété des nœuds Document.
    $C_2$ exprime le fait que si deux villes sont la capitale d'un même pays, alors elles sont équivalentes et doivent être représentées pat le même nœud dans le graphe.
\end{example}

\FloatBarrier
