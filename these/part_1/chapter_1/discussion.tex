Concernant \gls{setup} (\cite{chabinUsingGraphGrammar2019}), une étude menée dans \cite{chabinGraphRewritingSystem2020,chabinGraphRewritingRules2021} montre des temps d'exécution élevés pour l'application des règles sur de petites instances avec une première implémentation en mémoire primaire.
Ces performances s'expliquent par le calcul des effets de bords et la grande complexité de la réécriture de graphe.
Cela rend difficile l'utilisation de cette implémentation de \gls{setup} dans un scénario "en temps réel" ou des mises à jour arrivent régulièrement ou avec des grands graphes.
De plus, cette implémentation n'a pas de fonctionnalité de type \emph{fallback} permettant d'annuler les changements effectués sur la base en cas d'échec de la mise à jour, si le niveau de l'utilisateur n'est pas suffisant, ce qui implique de travailler sur une copie de la base.
Cependant, il reste efficace pour des mises à jour ponctuelles, des corrections de bases en mode déconnecté ou pour des problématiques liées à la sécurité dans le cadre du projet \gls{sendup} \cite{boiretPrivacyOperatorsSemantic2022}.
Bien que \gls{setup} permette la gestion de différents schémas définis sous \gls{rdfs}, il ne permet pas d'ajouter des contraintes métiers.
Définir de nouvelles contraintes impliquerait un travail conséquent et manuel pour leur traduction en règle de réécriture et la construction des effets de bords.

\cite{fanDependenciesGraphs2019} constitue une avancée majeure dans la théorie des dépendances fonctionnelles pour les graphes, explorant en détail leur complexité.
Contrairement à l'approche du \gls{setup}, cette méthode est adaptable, les dépendances fonctionnelles n'étant pas figées.
Comme pour les approches basées sur la réécriture de graphe, l'approche reste couteuse, mais des algorithmes parallèles sont proposés pour le raisonnement sur les \gls{gfd}.
Cependant, l'accent est mis sur la vérification des incohérences, sans fournir de solutions pour corriger les instances.
La question de l'identité des nœuds dans le contexte de la correction soulève des défis, notamment si deux nœuds sont déclarés équivalent : garder un seul des nœuds ou les fusionner en gérant les propriétés incohérentes n'est pas trivial et rejoins les problématiques rencontrées avec \gls{setup}.
Il peut être nécessaire de construire l'identité à partir des \gls{gfd} et des propriétés, de façon similaire aux clés primaires dans le modèle relationnel.
Dans les graphes, la notion d'équivalence régulière \cite{everettRegularEquivalenceGeneral1994} montre bien la complexité de cette tâche.
Les \gls{gfd} proposent de bien séparer la topologie de la contrainte qui s'applique alors uniquement sur les propriétés des nœuds ou relations identifiés par la contrainte topologique.
De ce fait, une politique de maintien de la cohérence, consisterai à changer les valeurs des attributs, insérer ou supprimer des attributs et, au maximum, à fusionner des nœuds ou des arêtes.
La création de nœuds ou d'arêtes reste impossible contrairement à ce qui est proposé avec \gls{setup}.

\begin{example}
      En prenant comme exemple le graphe figure~\ref{fig:update:discussion:g1} et la contrainte $C: Document(x) \to AuthorOf(y, x)$.
      Les \gls{gfd} ne permettrais pas de corriger l'instance contrairement à une approche comme \gls{setup} qui autorise de rajouter le nœud $Person$ et l'arête $AuthorOf$ manquante pour obtenir l'instance figure~\ref{fig:update:discussion:g2} qui est cohérente par rapport à la contrainte $C$.

      \begin{figure}[htb]
            \centering
            \begin{subfigure}[c]{.35\textwidth}
                  \centering
                  \begin{tikzpicture}[->, -latex, auto, on grid, node distance=4cm, thick, main node/.style={draw}]
                        \node[main node] (x) {Document: A};
                  \end{tikzpicture}
                  \caption{Instance d'origine non cohérente}
                  \label{fig:update:discussion:g1}
            \end{subfigure}
            \hfill
            \begin{subfigure}[c]{.1\textwidth}
                \centering
                \huge{$\Longrightarrow$}
            \end{subfigure}
            \hfill
            \begin{subfigure}[c]{.45\textwidth}
                  \centering
                  \begin{tikzpicture}[->, -latex, auto, on grid, node distance=4cm, thick, main node/.style={draw}]
                        \node[main node] (y) {Person: B};
                        \node[main node] (x) [right of=1] {Document: A};
                        \path[every node/.style={font=\sffamily\small}]
                        (y) edge node {AuthorOf} (x);
                  \end{tikzpicture}
                  \caption{Instance modifiée cohérente}
                  \label{fig:update:discussion:g2}
            \end{subfigure}
            \caption[Exemple de modification d'un graphe]{Exemple de modification d'un graphe qui implique la création de nouveaux éléments}
      \end{figure}
\end{example}

\pagebreak
Dans cette thèse, nous travaillons sur les principes établis dans \cite{chabinConsistentUpdatingDatabases2020} où les contraintes sont plus générales et non dirigées vers un modèle spécifique (comme les graphes d'attributs ou les graphes \gls{rdf}).
Dans ces travaux, visualiser la correction d'une base par l'introduction de nouveaux nœuds ou arêtes reste envisageable.
\cite{chabinConsistentUpdatingDatabases2020} introduit une sémantique formelle basée sur \gls{fol} pour mettre à jour des bases de données incomplètes contenant des valeurs nulles liées, en suivant les principes énoncés dans la sémantique de Reiter~\cite{reiterSoundSometimesComplete1986} (section \ref{reiterSemantic}).
Cette approche de mise à jour est déterministe et exige le respect d'un ensemble de \glspl{tgd}.
Les algorithmes présentés couvrent les opérations d'insertion et de suppression.
Pour valider cette proposition, une implémentation en mémoire primaire est proposée.
La mise à jour est structurée autour de deux actions fondamentales : \emph{chasing} et \emph{simplification}.
La procédure du \gls{chase} est utilisée pour garantir la cohérence avec un ensemble de contraintes d'intégrité exprimées sous forme de règles, pouvant engendrer de nouvelles valeurs nulles.
Le processus de simplification vise à réduire les redondances en éliminant les valeurs nulles qui peuvent être instanciées sans rompre leurs liens.
Il s'agit du calcul du \gls{core}~\cite{faginDataExchangeGetting2005}.
Cette approche présente une flexibilité accrue par rapport à celle présentée dans \cite{chabinUsingGraphGrammar2019}, grâce à l'utilisation de \glspl{tgd}.
Il convient de noter que l'utilisation de contraintes sous forme de \glspl{tgd} augmente l'expressivité, mais cela s'accompagne de difficultés nécessitant une procédure de poursuite pour calculer la sémantique et générer des effets secondaires lors des mises à jour, entraînant des insertions ou des suppressions supplémentaires pour maintenir la cohérence.
Une étude sur la procédure du \gls{chase} a été publiée dans \cite{onetChaseProcedureIts2013} et une évaluation dans \cite{benediktBenchmarkingChase2017}.
% Basée sur la sémantique énoncée par Reiter, cette approche présente une flexibilité accrue par rapport à celle présentée dans \cite{chabinUsingGraphGrammar2019}, grâce à l'utilisation de \glspl{tgd}.
% Cette approche a été choisie pour guider les travaux présentés dans cette partie.
