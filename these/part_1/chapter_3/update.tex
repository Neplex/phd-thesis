L'objectif de cette section est de d'écrire l'ensemble des opérations incrémentales nécessaire pour la mise à jour cohérente d'une base de donnée.

\subsection{Insertion}
L'algorithme~\ref{algo:update:insert} décrit la procédure pour l'insertion d'un ensemble d'atomes instanciés $I$ dans la base $\mathcal{D}$ selon l'ensemble de contraintes $\mathbb{C}$.


\begin{algorithm}[ht]
    \caption{$\textsc{Insert}(\mathcal{D}, \mathbb{C}, \delta_{max}, I)$}
    \label{algo:update:insert}
	$ToIns := \textsc{Chase4Insert}(\mathcal{D}, \mathbb{C}, \delta_{max}, I)$ \;
    $NullBucket := \{N_j \mid N_j \text{ est une valeur nulle obtenue par } Q_{bucket}(\mathcal{D} \cup ToIns)\}$ \;
    $\mathcal{D}' := \textsc{Simplify}(\mathcal{D} \cup ToIns, NullBucket)$ \;
    \uIf{$Q_{degree}(\mathcal{D}')_{[NullBucket,\delta_{max}]}$}{
        $Q_{\delta}(\mathcal{D}')_{[NullBucket,0]}$ \;
        \Return $\mathcal{D}'$ \;
    }
    \Else{
        \Return $\mathcal{D}$ \;
    }
\end{algorithm}
