Comme expliqué dans la section~\ref{sec:update:pre:db}, une base de données $\mathcal{D}$ doit être minimale, c.-à-d. égale à son core afin d'éviter la redondance des données.
La minimalité doit être maintenue à chaque mise à jour et de façon incrémentale.
Pour ce faire, des algorithmes incrémentaux sont proposé avec pour objectif de recalculer le core de la base uniquement en fonction des valeurs nulles impacté par la mise à jour.
Formellement, étant donné un ensemble d'atomes $I$ et un ensemble de valeurs nulles $\nu$ existant dans $I$, on recherche un homomorphisme $h$ tel que : pour toutes valeurs nulles $N \notin \nu$, $h(N) = N$, $h(I)$ est minimal et $h(I) \subseteq I$.
L'exemple \ref{ex:update:nulls-simp} montre cependant que le choix de $\nu$ n'est pas arbitraire.

\begin{example}
    \label{ex:update:nulls-simp}
\end{example}

Étant donné un ensemble d'atomes $I$ et un ensemble de valeurs nulles liées $\nu_0$ dans $I$, l'ensemble des valeurs nulles liées $\nu$ doit être calculé.
Pour cela, on calcule l'ensemble d'atomes $\textsf{LinkedNull}_{I,N}$ for chaque valeur nulle $N$ dans $\nu_0$.
Cet ensemble est la limite de la séquence $(\textsf{LinkedNull}^k_{I,N})_{k \ge 0}$ définie par :
\begin{align}
    \textsf{LinkedNull}^0_{I,N} &= \{atom \in I \mid N \in nulls(atom)\} \\
    \textsf{LinkedNull}^k_{I,N} &= \{atom \in I \mid (\exists atom' \in \textsf{LinkedNull}^{k-1}_{I,N})(nulls(atom) \cap nulls(atom') \neq \emptyset)\}
\end{align} 

Il est facile de voir que pour tout $k \geq 0$, $\textsf{LinkedNull}^k_{I,N} \subseteq \textsf{LinkedNull}^{k+1}_{I,N}$ et que $\textsf{LinkedNull}^k_{I,N} \subseteq I$.
La séquence $(\textsf{LinkedNull}^k_{I,N})_{k \ge 0}$ est donc monotone et bornée par $I$.
Comme $I$ est fini, la séquence à une unique limite qui est un sous-ensemble de $i$ noté $\textsf{LinkedNull}_{I,N}$.
La redondance peut donc être uniquement vérifiée sur l'ensemble d'atomes défini par $\bigcup_{N \in \nu_0}\textsf{LinkedNull}_{I,N}$ et l'ensemble $\nu$ de valeurs nulles dans cet ensemble.

\paragraph{\gls{cypher}}
L'ensemble $\textsf{LinkedNull}_{\mathcal{D}}$ (où $\mathcal{D}$ est la base de données) calculé à l'aide de la requête \gls{cypher} figure~\ref{fig:update:core:query}.
Il s'agit d'une requête paramétrée, reconnaissable par la présence d'une variable préfixé du symbole \verb|$|.
L'opérateur \verb|UNWIND| (ligne~\ref{fig:update:core:query:unwind}) permet de convertir une variable de type liste en une table de résultat.
Le \verb|MATCH| (ligne~\ref{fig:update:core:query:match}) recherche, dans le graphe, les homomorphismes du motif décrit.
Ici, on recherche un neoud, que l'on stockera dans la variable \verb|nullValueNode|, ayant les labels \textbf{:Element} et \textbf{:Nulls} et ayant comme valeur de la propriété \textit{value} la valeur stocké dans la variable \verb|nullPredName|.
La suite du motif (ligne~\ref{fig:update:core:query:range}) recherche un chemin \verb|pathP| de taille $n$, où $1 \leq n \leq \verb|maxPathLength|$, partant du noeud précédent.
\verb|maxPathLength| est défini comme le double du nombre de valeurs nulles dans la base et permet d'empêcher le parcours de chemins que l'on sait d'avance inutile.
La condition exprimée par l'opérateur \verb|WHERE|, vérifie que le neoud final du chemin soit différent du noeud de départ et qu'aucun noeud du chemin \verb|nodes(pathP)| n'ai le label \textbf{:Constant}.

L'opérateur \verb|WITH| (ligne~\ref{fig:update:core:query:collect}) permet de faire une projection d'une table intermédiaire.
Sa combinaison avec l'opérateur \verb|COLLECT| permet de regrouper les variables \verb|endNode| dans une variable \verb|linkedNodes| de type liste en fonction de la variable \verb|nullValueNode|.
Cette combinaison est similaire à l'opérateur \verb|GROUP BY| en \gls{sql}.
Intuitivement, on recherche pour chaque nulle l'ensemble des noeuds connectés uniquement par des neouds \textit{Atomes} ou des neouds \textit{Nulles}.
Les ensembles de neouds sont séparés (ligne~\ref{fig:update:core:query:with}) en deux listes \verb|linkedAtoms| et \verb|linkedNulls| contenant respectivement l'ensemble des noeuds \textit{Atomes} liés par des valeurs nulles et l'ensemble des noeuds \textit{Nulles} liant cet ensemble d'atomes.
L'opérateur \verb|UNWIND| final permet de retourner comme résultat de la requête chaque atome associé à l'ensemble des valeurs nulles auquel ils sont attachés.

\begin{figure}[htb]
	\begin{lstlisting}[escapechar=|]
UNWIND $nulls AS nullPredName |\label{fig:update:core:query:unwind}|
MATCH (nullValueNode:Element:Null {value: nullPredName}), |\label{fig:update:core:query:match}|
	pathP = (nullValueNode)-[*1..maxPathLength]-(endNode) |\label{fig:update:core:query:range}|
WHERE endNode <> nullValueNode AND
	ALL(n IN nodes(pathP) WHERE NOT (n:Constant))
WITH COLLECT(DISTINCT endNode) AS linkedNodes, nullValueNode |\label{fig:update:core:query:collect}|
WITH [n IN linkedNodes WHERE (n:Atom)] AS linkedAtoms, |\label{fig:update:core:query:with}|
	[nullValueNode] + [n IN linkedNodes WHERE (n:Null)] AS linkedNulls
UNWIND linkedAtoms AS a RETURN a.symbol as a, a.terms as e, linkedNulls
	\end{lstlisting}
	\caption{Requête Cypher pour calculer l'ensemble $\textsf{LinkedNull}_{\mathcal{D}}$}
	\label{fig:update:core:query}
\end{figure}

\begin{example}
	
\end{example}

\paragraph{Simplification}
La procédure pour éliminer la redondance sur un ensemble d'atomes $I$ en fonction d'un ensemble de valeurs nulles $\nu_0$ est donné par l'algorithme~\ref{algo:update:core}.
Ligne~\ref{algo:update:core:pset}, les ensembles d'atomes, liée par des valeurs nulles et qui contiennent au moins une valeur nulle présente dans $\nu_0$, sont calculés.
L'ensemble des valeurs nulles de $PSet$ est donc l'ensemble $\nu$.
À la ligne~\ref{algo:update:core:qcore}, pour chaque ensemble d'atomes $P$, on construit la requête $Q_{core}(X) \gets \phi(X)$ où $\phi$ est l'ensemble des atomes de $P$ dans lequel chaque valeur nulle $N_i$ est remplacé par la variable correspondante $x_i$.
En d'autres termes, la requête $Q_{core}$ retourne l'ensemble des homomorphismes sur les valeurs nulles de $P$.
Si plusieurs homomorphismes sont retournés par la requête, cela signifie qu'il existe plusieurs instanciations possibles de $P$ et qu'il faut le simplifier, car $P$ est redondant.
Dans ce cas, il faut choisir l'instanciation (ou homomorphisme) la \textit{plus spécifique} (ligne~\ref{algo:update:core:mostspecific}) et supprimer les atomes redondants (ligne~\ref{algo:update:core:select}).
Si l'on ne prend pas l'instanciation la \textit{plus spécifique} il faudra la simplifier à nouveau.

\begin{algorithm}[ht]
	\caption{$\textsc{Simplify}(I, \nu_0)$}
    \label{algo:update:core}
	$PSet := \{\textsf{LinkedNull}_{I,N} \mid N \in \nu_0\}$\; \label{algo:update:core:pset}
	\ForAll{$P \in PSet$}{
		Construire et exécuter la requête $Q_{core}(I)$ pour récupérer l'ensemble des homomorphismes $H = \{h_1, \dots, h_n\}$\; \label{algo:update:core:qcore}
		\If{$\mid (Q_{core}(I)) \mid > 1$}{
			$h_m :=$ $\text{PlusSpecifique}(H)$\; \label{algo:update:core:mostspecific}
			$I := (I \setminus P) \cup h_m(P)$\; \label{algo:update:core:select}
		}
	}
	\Return $I$\;
\end{algorithm}

Étant donné deux homomorphismes $h_1$ et $h_2$ sur le même alphabet $\Sigma$ : $h_1$ est dit \textit{moins spécifique} que $h_2$, noté $h_1 \preceq h_2$, s'il existe un homomorphisme $h$ sur $\Sigma$ tel que $h \circ h_1 = h_2$ (c.-à-d. qu'il existe une transformation de $h_1$ vers $h_2$ nommée $h$).
$p$ est le nombre de valeurs nulles $N_j$ distinctes dans $P = \textsf{LinkedNull}_{I,N}$ et $q$ est le nombre d'homomorphismes $h_i$ retournés par la requête $Q_core$ avec $h_1$ étant l'identité de $P$.
Pour trouver l'homomorphisme le plus spécifique sur $P$, on construit la table $H_P$ de $p$ colonnes et $q$ lignes où $H_P[i, j] = h_i(N_j)$.
Dans la table $H_P$, $h_i \preceq h_i'$ si et seulement si, pour tout $j$ :

\todo[inline]{Give insight abour HP calculus and cite the report for the proof and detailed explanation}
\begin{itemize}
	\item Si $H_P[i, j] \notin nulls(P)$ alors $H_P[i, j] = H_P[i', j]$
	\item Si $H_P[i, j] \in nulls(P)$ alors pour tout $j' \neq j$ tel que $H_P[i, j] = H_P[i, j']$ on a $H_P[i', j] = H_P[i', j']$
\end{itemize}

\begin{example}
	
\end{example}

Cette approche basée sur la table $H_P $ est comparable aux techniques d'optimisation de requêtes \cite{}.
Cependant, elle diffère de \cite{ahoEfficientOptimizationClass1979,chandraOptimalImplementationConjunctive1977} sur deux points :
\begin{enumerate*}[label=(\roman*)]
	\item notre tableau est basé sur les réponses aux requêtes plutôt que sur le corps de la requête ;
	\item nous calculons un homomorphisme le plus spécifique, alors que dans \cite{ahoEfficientOptimizationClass1979} ils rejettent les homomorphismes les moins spécifiques.
\end{enumerate*}
