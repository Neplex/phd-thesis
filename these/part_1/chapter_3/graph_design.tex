La proposition de mise à jour incrémentale sur des bases de données graphes est implémentée sur le gestionnaire de base de données graphes \gls{neo4j} en définissant un ensemble de requêtes \gls{cypher}.
La maintenance décrite dans \cite{chabinConsistentUpdatingDatabases2020} relève un cout important pour la récupération des atomes liés par des valeurs nulles.
L'utilisation de base de données graphe ajoute une possibilité d'optimisation de cette opération.
Le modèle de graphe doit mettre en avant ces relations et faciliter la recherche des homomorphismes.

Pour ce faire, on choisit le modèle Figure~\ref{} proche de la logique où $t_i$ représente les constantes et $t_j$ les valeurs nulles.
Les notations en dessous des arcs représente leur cardinalité : chaque \textit{Élément} est connecté à au moins un \textit{Atom} alors qu'un \textit{Atom} peut ne pas avoir de termes.
Dans ce modèle un atome $P(t_1, \dots, t_n)$ est représenté comme un arbre qui a pour racine un noeud \textit{Atome} lié à un ensemble de noeuds \textit{Élément} représentants les termes.
Les \textit{Atomes} ont le label \textbf{:Atom} et une unique propriété représentant le symbole de prédicat.
Les \textit{Éléments} ont le label \textbf{:Element} et une unique propriété représentant la valeur de l'élément.
On leur ajoute aussi un label \textbf{:Constant} ou \textbf{:Null} respectivement s'il s'agit d'une constante ou d'une valeur nulle.
Les \textit{Atomes} sont connectés aux \textit{Éléments} par une relation qui a pour unique propriété le rang auquel apparait l'élément dans l'atome.
