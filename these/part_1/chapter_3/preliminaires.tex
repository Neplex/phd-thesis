\subsection{Logique du premier ordre}

En assumant une certaine familiarité avec la logique du premier ordre, on utilise comme alphabet :
\begin{itemize}
    \item un ensemble de prédicats $P, Q, R, \dots$
    \item un ensemble de variables $x, y, z, \dots$
    \item un ensemble de constantes $a, b, c, \dots$
    \item un ensemble de valeurs nulles marquées $N_1, N_2, N_3, \dots$
    \item le quantificateur universel $\forall$ (quel que soit)
    \item le quantificateur existentiel $\exists$ (il existe)
\end{itemize}
Avec cet alphabet, on définit un atome $P(t_1, \dots, t_n)$ où $P$ est le symbole de prédicat d'arité $n$ et $t_1, \dots, t_n$ sont les termes.
Les termes peuvent être une variable, une constante ou une valeur nulles.
Un fait est un atome composé uniquement de constantes et un atome instancié est un atome qui ne contient aucunes variables.
Étant donné un atome instancié $A$, $null(A)$ représente l'ensemble des valeurs nulles de $A$.

Un homomorphisme $h$ entre deux ensembles d'atomes $A$ et $B$ est une application surjective sur les termes de $A$ vers les termes de $B$ telle que :
\begin{enumerate*}[label=(\roman*)]
    \item si, $h(t) = t$, alors $t$ est une constante et
    \item si, $P(t_1, \dots, t_n) \in A$, alors $P(h(t_1), \dots, h(t_n)) \in B$.
\end{enumerate*}
S'il existe un homomorphisme inverse $h^-$ tel que $h^-(h(t)) = t$ alors ont dit que $A$ est \textit{isomorphe} à $B$.

$\Phi$ est l'ensemble de toutes les formules $\phi$, existentiellement quantifiées, de la forme $(\exists X)(\varphi_1(X_1) \land \dots \land \varphi_n(X_n))$ où $X$ est l'ensemble des variables et $\varphi_i(X_i)$ est une formule atomique dont $X_i \subset X$ représente l'ensemble des variables libres.
$atoms(\phi)$ est l'ensemble des formules atomique de $phi$.
Un modèle $M$ d'une formule $\phi$ est l'ensemble des faits qui ont un homomorphisme de $atoms(\phi)$ vers $M$.
Étant donné deux formules $\phi_1$ et $\phi_2$ : $\phi_1 \implies \phi_2$ si tout modèle de $\phi_1$ est un modèle de $\phi_2$.
$\phi_1$ et $\phi_2$ sont \textit{équivalents}, noté $\phi_1 \iff \phi_2$, si et seulement si $\phi_1 \implies \phi_2$ et $\phi_2 \implies \phi_1$.
$\phi_1$ et \textit{plus simple que} $\phi_2$, noté $\phi_1 \sqsubseteq \phi_2$ si $\phi_1 \iff \phi_2$ et $atoms(\phi_1) \subseteq atoms(\phi_2)$.
Une simplification $\phi_1$ de $\phi_2$ est \textit{minimale} si $\phi_1 \sqsubseteq \phi_2$ et qu'il n'existe pas de $\phi_1'$ tel que $\phi_1' \sqsubset \phi_1$.
Par exemple, étant donné la formule $\phi = (\exists x, y)(P(a, x) \land P(a, y))$, les formules $(\exists x)(P(a, x))$ et $(\exists y)(P(a, y))$ sont équivalentes et sont toutes deux une simplification de $\phi$.
De plus, \cite{chabinConsistentUpdatingDatabases2020} démontre que si $\phi_1$ et $\phi_2$ sont deux simplifications minimales d'une même formule $\phi$ alors les ensembles $atoms(\phi_1)$ et $atoms(\phi_2)$ sont isomorphes.
La simplification minimale d'une formule $\phi$ est appelé le core et s'écrit $core(\phi)$.

\subsection{Base de données}

L'\textit{instance} d'une base de donnée peux être décrite comme la Skolémisation d'une formule $\phi$ correspondant a un ensemble minimal d'atomes instanciés tel que $core(\phi) = \phi$.
La Skolémisation d'une formule $\phi$ noté $Sk(\phi)$ consiste à instancier l'ensemble des variables par des constantes ou des valeurs nulles et à éliminer toutes les occurrences de quantificateur existentiel.

Une \textit{contrainte} (ou règle) sur une base de donnée est une dépendance génératrice de n-uplets (ou Tuple-generating dependency, TGD en anglais) de la forme $(\forall X, Y)(body(X, Y) \to (\exists Z)head(X, Z))$.
Plus simplement écrit $body(X, Y) \to head(X, Z)$, $X$, $Y$ et $Z$ sont des ensembles de variables, $bod(X, Y)$ est un ensemble d'atomes et $head(X, Z)$ est un atome.
Une contrainte $c$ est dite satisfaite par un ensemble $I$ d'atomes instanciés, noté $I \vDash c$ si pour tout homomorphisme $h$ tel que $h(body(c)) \subseteq I$ il existe un homomorphisme $h'$ tel que $h(body(c)) = h(body(c))$ et $h'(head(c)) \subseteq I$.
Intuitivement, cela signifie que pour qu'une contrainte soit satisfaite, s'il existe une instance du corps dans l'instance de la base de données, la tête correspondante doit aussi être présente.
Une instance $I$ d'une base de données est dites \textit{cohérente} par rapport à un ensemble de contraintes $\mathcal{C}$, noté $I \vDash \mathcal{C}$ si pour toute contrainte $c$ dans $\mathcal{C}$, $I \vDash c$.

\subsection{Base de données graphes}

Quand on parle de \index{base de données} graphe on peut distinguer deux grandes familles :
\begin{description}
    \item[Les bases de triplets] principalement le \gls{rdf} porté par le \gls{w3c}.
        Ce type de base de données enregistre des triplets de la forme $<x, P, y>$ qui sont des prédicats binaires où $P$ est un symbole de prédicat, $x$, le sujet, est l'URI d'une ressource (un noeud) et $y$, l'objet est un URI ou un littéral (une valeur de propriétés).
    \item[Les Graphes de Propriétés], en anglais \gls{lpg}
\end{description}
