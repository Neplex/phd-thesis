L'algorithme du \gls{chase} \cite{ahoTheoryJoinsRelational1979,maierTestingImplicationsData1979} est utilisé pour saturer une base en fonction des \glspl{tgd}.
Une contrainte est appliquée, c.-à-d. que l'on génère un nouvel atome instancié correspondant à la tête, quand le corps de la contrainte peux être instancié par des atomes dans la base.
L'algorithme du \gls{chase} se termine quand plus aucune règle ne peut être instanciée ou que plus aucun nouvel atome n'est généré.
Une version incrémentale de l'algorithme consiste à considérer uniquement les règles et leur instanciation sur l'ensemble d'atomes mis à jour.

\begin{example}[Application de contraintes]
    \label{ex:update:chase:forward}
    Si on considère l'ensemble de contraintes $\mathbb{C} = \{$\ref{ex:update:chase:forward:c1}, \ref{ex:update:chase:forward:c2}$\}$ et l'ensemble d'atomes $\mathcal{D} = \{A(a),$ $B(a, b),$ $B(c, b),$ $C(b)\}$.
    \begin{enumerate}[label=\textbf{$c_\arabic*$ :},ref=$c_\arabic*$]
        \item $A(x) \implies B(x, y)$ \label{ex:update:chase:forward:c1}
        \item $B(x, y) \implies C(y)$ \label{ex:update:chase:forward:c2}
    \end{enumerate}

    L'insertion de l'ensemble $I = \{A(b), A(d)\}$ dans $\mathcal{D}$ implique l'application de la contrainte \ref{ex:update:chase:forward:c1} uniquement avec les instanciation $h_1 = \{x \mapsto b\}$ et $h_2 = \{x \mapsto c\}$.
    L'instanciation $h = \{x \mapsto a\}$ n'a pas besoin d'etre calculé car $\mathcal{D}$ est déjà cohérente par rapport à $\mathbb{C}$.
    A ce stade, il n'est pas non plus necessaire de verifier si \ref{ex:update:chase:forward:c2} est applicable.
    L'application de $h_1($\ref{ex:update:chase:forward:c1}$)$ génére un nouvel atome $B(a, N_1)$.
    Comme il existe déjà une instanciation plus spécifique dans la base : $B(a, b)$, ce nouvel atome n'a pas besoin d'etre ajouté.
    L'application de $h_2($\ref{ex:update:chase:forward:c1}$)$ génére un nouvel atome $B(d, N_1)$.
    Comme il n'existe pas dans $\mathcal{D}$, on l'ajoute et on recommence l'algo.
    Avec l'insertion de $B(d, N_1)$, on declenche \ref{ex:update:chase:forward:c2} avec l'instanciation $h_3 = \{x \mapsto d, y \mapsto N_1\}$ qui genere a son tour l'atome $C(N_1)$.
    Plus aucun atome ne peux etre généré et on obtient donc l'ensemble $\mathcal{D}' = \{A(a),$ $B(a, b),$ $B(c, b),$ $C(b), B(d, N_1), C(N_1)\}$.
\end{example}

\paragraph{Cycles et génération infinie}
Dans ces travaux, les \glspl{tgd} peuvent contenir des variables existentielles dans la tête qui génèreront de nouvelles valeurs nulles lors de leur application.
Cela peux engendrer une génération infinie d'atomes si des cycles existent dans les contraintes.
Il est important de s'assurer que cette opération termine.
Pour faire face à cette problématique, on associe à chaque valeur nulle, un entier appelé le degré de $N$ et dénoté par $\delta(N^k) = k$.
À chaque mis à jour, le degré de toutes les valeurs nulles apparaissant dans $\mathcal{D}$ est fixé à $0$.
Pendant le \gls{chase}, à chaque itération, les nouvelle valeurs nulles générées se voient attribuer un degré égal à $\delta + 1$, où $\delta$ est le degré maximal des valeurs nulles presentes dans les atomes du corps de la contrainte, ou $0$ si aucune valeur nulle n'apparaît dans le corps de la contrainte.
On fixe donc un degré maximal $\delta_{max}$ qui permet l'interruption du \gls{chase} dès qu'une valeur nulle $N$ depasse ce seuil, tel que $\delta(N) > \delta_{max}$.

\begin{example}[Génération infinie d'atomes]
    \label{ex:update:chase:infini}
    Si on considère l'ensemble d'atomes $I = \{A(a, b)\}$ sur lequel on applique l'ensemble de contraintes $\mathbb{C} = \{$\ref{ex:update:chase:infini:c3}, \ref{ex:update:chase:infini:c4}$\}$ avec pour degrée maximal $\delta_{max} = 1$.
    \begin{enumerate}[label=\textbf{$c_\arabic*$ :},ref=$c_\arabic*$]
        \item $A(x, y) \implies B(y, z)$ \label{ex:update:chase:infini:c3}
        \item $B(x, y) \implies A(y, z)$ \label{ex:update:chase:infini:c4}
    \end{enumerate}

    L'application de la regle \ref{ex:update:chase:infini:c3} avec l'instanciation $h_1 = \{x \mapsto a, y \mapsto b\}$ génére l'atome $B(b, N_1^0)$.
    Ce dernier permet l'application de la contrainte \ref{ex:update:chase:infini:c3} avec l'instanciation $h_2 = \{x \mapsto b, y \mapsto N_1^0\}$ de qui génére l'atome $A(N_1^0, N_2^1)$, qui permet la génération de $B(N_2^1, N_3^2)$ via \ref{ex:update:chase:infini:c3}.
    l est facile de voir que la génération des atomes est infinie.
    Cependant, comme le degrée de $N_3^2$ depasse le degrée maximal $\delta_{max}$ on arrete cette branche de la génération.
    Si d'autres régles peuvent étres appliquée on continue de la génération.
    L'ensemble de resultat $I'$ est invalid car une valeur nulle dépasse le degrée maximal.
    On verifira dans la suite que cet ensemble n'est pas simplifiable (et donc valide si la simplification supprime cette valeur nulle) avant de rejeté la mise à jour si ce n'est pas le cas.
\end{example}

La procédure incrémentale du \gls{chase} est définie dans l'algorithme~\ref{algo:update:chase:insert}.
L'ensemble $ToIns$ represente les effet de bords de $I$ à ajouter dans $\mathcal{D}$ tel que si $\mathcal{D}' = \mathcal{D} \cup ToIns$ : $I \subseteq \mathcal{D}'$ et $\mathcal{D}' \vDash \mathbb{C}$.
La boucle (ligne~\ref{algo:update:chase:insert:while}) est l'application du \gls{chase} incrémentale qui respecte les conditions suivantes:
\begin{enumerate}[label=(\arabic*),ref=(\arabic*)]
    \item Une contrainte $c \in \mathbb{C}$ est appliquée \textit{uniquement} si au moins un atome de son corps fait parti des effet de bords $ToIns$  \label{algo:update:chase:insert:c1} ;
    \item Un atome est généré uniquement s'il ne depasse pas le degrée maximum \label{algo:update:chase:insert:c2} ;
    \item Un atome est généré uniquement s'il n'existe pas, dans $\mathcal{D}$, un atome plus specifique ou equivalent \label{algo:update:chase:insert:c3}
\end{enumerate}

\begin{algorithm}[ht]
    \caption{$\textsc{Chase4Insert}(\mathcal{D}, \mathbb{C}, \delta_{max}, I)$}
    \label{algo:update:chase:insert}
    $ToIns := I$ \;
    \While{\label{algo:update:chase:insert:while} il existe une contrainte $c \in \mathbb{C}$ et un $h$ tel que $h(body(c)) \in \mathcal{D}$ qui respecte :
    \begin{enumerate}
        \item[\ref{algo:update:chase:insert:c1}] $h(body(c)) \cap ToIns \neq \emptyset$
        \item[\ref{algo:update:chase:insert:c2}] $\delta(h'(head(c))) \leq \delta_{\max}$ où $h' \supseteq h$ instancie les variables existentielles de $head(c)$ par des null
        \item[\ref{algo:update:chase:insert:c3}] $\nexists h''$ tel que $h''(h'(head(c))) \in \mathcal{D} \cup ToIns$
    \end{enumerate}
    }{
        $ToIns := ToIns \cup \{h'(head(c))\}$ \;
    }
    \Return{$ToIns$} \;
\end{algorithm}

\paragraph{\gls{cypher}}
Etant donné une contrainte $c : L_1(\alpha_1), \dots, L_m(\alpha_m) \implies L_0(\alpha_0)$ où $\alpha$ est l'ensemble de taille $k_N$ des variables libres du corps de la contrainte.
$\alpha_i \subseteq \alpha$ est un sous-ensemble, $\alpha|_i \in \alpha$ denote le nom de la $i$-éme variable libre, $k_C$ est le nombre de constantes distincte dans $c$ et $K = k_N + k_C$ est le nombre de terms distinct de $c$.
La recherche de l'application de $c$ dans la base $\mathcal{D}$ se traduit par une requête $Q_{chase}^{[c]}(\mathcal{D})$ présentée figure~\ref{algo:update:chase:query} qui :
\begin{enumerate*}[label=(\roman*)]
    \item verifie l'existance d'une instance dans $\mathcal{D}$ et
    \item retourne une reponse non vide correspondant a l'instanciation de la tête uniquement si ell n'existe pas déja dans la base.
\end{enumerate*}
En \gls{fol}, cette requête s'ecrit $Q_{chase}^{[c]} \gets L_1(\alpha_1), \dots, L_m(\alpha_m), \lnot L_0(\alpha_0)$.
L'idée est que si il existe un homomorphisme $h_t$ tel que $h_t(body(c)) \subseteq \mathcal{D}$ alors $Q_{chase}^{[c]}$ retourne une reponse non vide uniquement si pour toute extension $h_t'$ de $h_t$ : $h_t'(L_0(\alpha_0)) \notin \mathcal{D}$.

La requête $Q_{chase}^{[c]}(\mathcal{D})$ commence par rechercher les instanciation de $sub(body(c))$ (ligne~\ref{query:update:chase:match}) où $sub$ est un homomorphisme tel que $sub(body(c)) \subseteq ToIns$.
Intutivement on recherche les extension $sub'$ de la substitution $sub$ dans $\mathcal{D}$.
Il s'agit d'une recherche d'homomorphisme comme decrite section~\ref{sec:update:db:homomorphisme}.
Le \verb|WHERE|, ligne~\ref{query:update:chase:where}, permet de verifier l'instanciation du corps de la contrainte $c$ par rapport à la substitution $sub$.
L'operateur \verb|NOT EXISTS|, ligne~\ref{query:update:chase:not-exist}, execute une sous-requête qui reutilise les variables introduite dans le premier \verb|MATCH|.
Cette sous requête recherche les instanciation de $sub'(head(c))$ dans $\mathcal{D}$.
Si la reponse de cette sous requête est vide, l'instanciation $sub'$ est retournée.
La requête retourne la substitution comme un mapping entre le nom des variables dans $c$ et leur instanciation.

\begin{figure}[ht]
	\begin{lstlisting}[mathescape]
UNWIND subs AS sub
MATCH $(\forall i, 1 \leq i \leq m)$ ($a_i$:Atom {symbol: `$L_i$`}), $\label{query:update:chase:match}$
      $(\forall i, 0 \leq i < k_N)$ ($a_i$)-[:`$L_i$` {rank: $j$}]->($e_i$:Element)
WHERE $(\forall i, k_N \leq i \leq k_C)$ ($a_i$)-[:`$L_i$` {rank: $j$}]->($e_i$ {value: sub.$i$}:Element) $\label{query:update:chase:where}$
AND NOT EXISTS { $\label{query:update:chase:not-exist}$
      MATCH ($a$:Atom {symbol: `$L_0$`}), $(\forall i, 0 \leq i \leq |\alpha_0|)$ ($a_i$)-[:`$L_i$` {rank: $j$}]->($e_i$)
} RETURN {$\alpha|_1$:$x_1$.value, $\alpha|_2$:$x_2$.value, $\dots$, $\alpha|_k$:$x_k$.value}
	\end{lstlisting}
	\caption{Format des requêtes $Q_{chase}^{[c]}$}
    \label{algo:update:chase:query}
\end{figure}

\begin{example}[Execution de la requête]
    \label{ex:update:chase:query}

\end{example}

\paragraph{\gls{chase} en arriére}
Il peut arriver que la suppression d'un atome instancié $a$ rende la base de données incohérente lorsqu'il est une conséquence d'une contrainte $c$.
C'est à dire que, pour toute contrainte $c$, s'il existe un homomorphisme $h$ tel que $h(head(c)) = a$ et une extension de $h$ noté $h'$ tel que $h'(body(c)) \subseteq \mathcal{D}$ alors la base ne respecte plus la contrainte $c$ et l'on doit regenerer la tête, que l'on veux supprimer, afin de maintenir la cohérence.
L'idée du chase arriére est de supprimer recursivement les atomes $a' \in h'(body(c))$ qui interviennent dans le corps.
Pour éviter le non-déterminisme, il est supposé que l'atome à supprimer a été marqué du symbol `$-$' lors de la conception de la règle.

Comme pour les insertions, l'algorithme \ref{algo:update:chase:delete} calcule, de manière incrémentale, les effets de bords.
$ToDel$ represente l'ensemble des atomes à supprimer et $ToIns$ l'ensemble des nouveaux atomes à insérer.
L'idée est de vérifier si $c$ génère un atome isomorphe à un atome en cours de suppression (ligne~\ref{algo:update:chase:delete:while}).
Pour ce faire, on utilise la requête $Q_{chase}^{[c]}$ où l'on instancie la tête de la régle au lieu du corps.
Si on génère un atome isomorphe à un atome dans $ToDel$, au moins un atome dans $h(body(c))$ doit être supprimé afin d'empêcher $c$ d'être déclenché.
Cet atome est ensuite ajouté à l'ensemble des atomes à supprimer $ToDel$ (ligne~\ref{algo:update:chase:delete:delete1}).
Lorque l'application de la régle donne un nouvel atome $a'$ non-isomorphe à $a$, il est inséré et peux a son tour declencher d'autre régles.
Les effet de bords sont calculé à l'aide d'un \gls{chase} en avant (ligne~\ref{algo:update:chase:delete:chaseforward}) avec l'algorithme~\ref{algo:update:chase:insert} et stockés dans l'ensemble $NewToIns$.
Ligne~\ref{algo:update:chase:delete:delta}, si les effets de bords contiennent un atome de $ToDel$ a supprimer et/ou que le degree maximal $\delta_{max}$ est depassé : l'application de la contrainte $c$ est invalide est l'atome à supprimer dans le coprs est ajouté à $ToDel$ (ligne~\ref{algo:update:chase:delete:delete2}).
Sinon, les nouveaux atomes sont ajoutés à $ToIns$ (ligne~\ref{algo:update:chase:delete:insert}).

\begin{algorithm}[ht]
	\caption{$\textsc{Chase4Delete}(\mathcal{D}, \mathbb{C}, \delta_{max}, del_{iso})$}
    \label{algo:update:chase:delete}
	$ToIns := \emptyset$
    $ToDel := del_{iso}$ \;
	\While{$\exists c \in \mathbb{C}$ et $h$ tel que $h(\text{head}(c)) \in ToDel$ et $h(\text{body}(c)) \subset (\mathcal{D} \setminus ToDel) \cup ToIns$ \label{algo:update:chase:delete:while}}{
		\uIf{il existe $h'$ tel que $h'(\text{body}(c)) = h(\text{body}(c))$ et $h'(\text{head}(c))$ est isomorphe à $h(\text{head}(c))$ \label{algo:update:chase:delete:isomorphe}}{
			$ToDel := ToDel \cup \{h'(\text{body}^-(c))\}$\;\label{algo:update:chase:delete:delete1}
		}
		\Else{
			$NewToIns := \textsc{Chase4Insert}(\mathcal{D}, \mathbb{C}, \delta_{max}, ToIns \cup \{h'(\text{head}(c))\})$\label{algo:update:chase:delete:chaseforward}\;
			\uIf{$NewToIns \cap ToDel = \emptyset$ et $\delta(N) < \delta_{max}$ pour tous les nulls $N$ dans $NewToIns$ \label{algo:update:chase:delete:delta}}{
				$ToIns := ToIns \cup NewToIns$\;\label{algo:update:chase:delete:insert}
			}
			\Else{
				$ToDel := ToDel \cup \{h'(\text{body}^-(c))\}$\;\label{algo:update:chase:delete:delete2}
			}
		}
	}
	\Return $ToDel, ToIns$
\end{algorithm}

\begin{example}[\gls{chase} en arriére]
    \label{ex:update:chase:backward}

\end{example}
