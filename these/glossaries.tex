% Glossaries
\newglossaryentry{hifun}{name=HiFun, description={Un langage de requête fonctionnel de haut niveau pour l'analyse des Big Data}}
\newglossaryentry{doing}{name=DOING, description={Réseaux discutant de problématique autours de l'extraction, le stockage et le traitement de données médicales. \url{https://www.univ-orleans.fr/lifo/evenements/doing/}}}
\newglossaryentry{wikipedia}{name=Wikip\'edia, description={Une encyclopédie universelle, multilangue, libre et collaborative. \url{https://wikipedia.org/}}}
\newglossaryentry{python}{name=Python, description={Un language de programmation interprété utilisant un typage dynamique. Il est principalement utilisé dans les sciences des données. \url{https://www.python.org/}}}
\newglossaryentry{cypher}{name=Cypher, description={Un langage de requetage conçu pas Neo4J pour les bases de données graphes basée sur de la reconnaissance de motifs decrit en art ASCII}}
\newglossaryentry{neo4j}{name=Neo4J, description={Un système de gestion de base de données utilisant des graphes de propriétés. \url{https://neo4j.com/}}}
\newglossaryentry{mysql}{name=MySQL, description={Un système de gestion de bases de données relationnelles. \url{https://www.mysql.com/}}}
\newglossaryentry{docker}{name=Docker, description={Docker est une plateforme permettant l'execution dans des environments reproductible et légers. \url{https://www.docker.com/}}}

\newglossaryentry{linux}{name=Linux, description={Linux ou GNU/Linux est une famille de systèmes d'exploitation open source de type Unix fondés sur le noyau Linux}}
\newglossaryentry{rocky}{name=Rocky Linux, description={Rocky Linux est une distribution \gls{linux} basée sur le code source du système d'exploitation Red Hat Enterprise Linux. \url{https://rockylinux.org/}}}
\newglossaryentry{git}{name=Git, description={Git est un système de gestion de versions décentralisé, libre et gratuit, créé par Linus Torvalds. Il est notament utilisé pour gérer du code source comme le noyau \gls{linux}. \url{https://git-scm.com/}}}
\newglossaryentry{rasa}{name=RASA, description={\url{https://rasa.com/}}}
\newglossaryentry{spacy}{name=SpaCy, description={SpaCy est une bibliothèque logicielle Python de traitement automatique des langues maintenue par ExplosionAI. \url{https://spacy.io/}}}
\newglossaryentry{nltk}{name=NLTK, description={Natural Language Toolkit est une bibliothèque logicielle en Python permettant le traitement automatique des langues créée par luniversité de Pennsylvanie. \url{https://www.nltk.org/}}}
\newglossaryentry{iso}{name=ISO, description={L'organisation internationale de normalisation}}

\newglossaryentry{ud}{name=Universal Dependencies, description={Universal Dependencies (UD) est un guide d'annotation pour la construction cohérente de grammaires textuelle (parties du discours, caractéristiques morphologiques et dépendances syntaxiques) dans différentes langues humaines. L'UD est un effort communautaire ouvert avec plus de \num{500} contributeurs produisant plus de \num{200} banques d'arbres dans plus de \num{100} langues. \url{https://universaldependencies.org/}}}

% Acronyms
\newacronym{lifo}{LIFO}{Laboratoire d'Informatique Fondamentale d'Orl\'eans}
\newacronym{insa}{INSA}{Institut National des Sciences Appliqu\'ees}
\newacronym{ged}{GED}{Gestion Electronique de Documents}
\newglossaryentry{ennov}{name=Ennov, description={Editeur de logiciels concevant une plateforme unifiée de gestion du contenu et de l'information pour soutenir le développement des produits des sciences de la vie. \url{https://ennov.com/}}}
\newglossaryentry{gartner}{name=Gartner, description={Une entreprise américaine de conseil et de recherche dans le domaine des techniques avancées}}
\newacronym{rim}{RIM/RIMS}{Regulatory Information Management System}
\newacronym{idmp}{IDMP}{Identification of Medicinal Products}
\newacronym{inserm}{INSERM}{Institut National de la Sant\'e Et de la Recherche M\'edicale}
\newacronym{ema}{EMA}{European Medicines Agency}
\newacronym{ifpma}{IFPMA}{International Federation of Pharmaceutical Manufacturers and Associations}
\newacronym{cih}{CIH}{Conseil International d'Harmonisation des exigences techniques pour l'enregistrement des m\'edicaments \`a usage humain}
\newacronym{rcp}{RCP}{R\'esum\'es des Caract\'eristiques du Produit}
\newacronym{pdr}{PDR}{Physician's Desk Reference}
\newacronym{ansm}{ANSM}{Agence Nationale de S\'ecurit\'e du M\'edicament et des produits de sant\'e}
\newacronym{fda}{FDA}{Food and Drug Administration}
\newacronym{cismef}{CISMeF}{Catalogue et Index des Sites M\'edicaux de langue Francaise}
\newacronym{hetop}{HeTOP}{Health Terminology/Ontology Portal}
\newacronym{xml}{XML}{Extensible Markup Language}
\newacronym{gfd}{GED}{Graph Entity Dependencies}
\newacronym{sendup}{SENDUP}{réSEaux Numériques de Données sémantiques : Utilité et vie Privée}

\newacronym{tal}{TAL}{Traitement Automatique des Langues}
\newacronym{nlp}{NLP}{Natural Language Processing}
\newacronym{nlu}{NLU}{Natural Language Understanding}
\newacronym{ner}{NER}{Named-Entity Recognition}
\newacronym{ie}{IE}{Information Extraction}
\newacronym{nli}{ILN}{Interface en Langage Naturel}
\newacronym{deft}{DEFT}{D\'Efi Fouille de Textes}
\newacronym{tfidf}{TF-IDF}{Term Frequency-Inverse Document Frequency}
\newacronym{svm}{SVM}{Support Vector Machine}
\newacronym{fst}{FST}{Finite-State Transducer}
\newacronym{crf}{CRF}{Conditional Random Fields}
\newacronym{cnn}{CNN}{Convolutional Neural Network}
\newacronym{bert}{BERT}{Bidirectional Encoder Representations from Transformers}
\newacronym{llm}{LLM}{Large Language Model}
\newacronym{gpt}{GPT}{Generative Pre-trained Transformer}
\newacronym{bio}{BIO}{Begin, Inside, Outside}
\newacronym{dsl}{DSL}{Domain Specific Language}

\newacronym{mea}{E/A}{mod\`ele Entit\'e-Association}
\newacronym{bnf}{BNF}{Forme de Backus-Naur}
\newacronym{cfg}{CFG}{Grammaire hors contexte}
\newacronym{pos}{PoS}{Part of Speech}
\newacronym{ldbcsnb}{LDBC SNB}{LDBC Social Network Benchmark}
\newacronym{qa}{QA}{Question Answering}

\newglossaryentry{corenlp}{name=CoreNLP, description={\url{https://stanfordnlp.github.io/CoreNLP/}}}
\newglossaryentry{duckling}{name=Duckling, description={Duckling est une librairie open source développée par \gls{meta} qui permet d'extraire des informations structurées telles que les dates, les heures et les montants d'argent à partir de texte brut grace à un ensemble de grammaire.\url{https://github.com/facebook/duckling}}}
\newglossaryentry{meta}{name=Meta, description={Meta Platforms ou Meta, anciennement connue sous le nom de Facebook, est une multinationale technologique américaine axée sur les plateformes sociales et la réalité virtuelle fondée en 2004 par Mark Zuckerberg.}}

\newglossaryentry{ray}{name=Ray, description={Ray est une librairie \gls{python} open-source conçue pour faciliter le développement et le déploiement d'applications parallèles et distribuées à grande échelle.}}
\newacronym{dag}{DAG}{Directed Acyclic Graph}

\newglossaryentry{solr}{name=SolR, description={Une plateforme logicielle de moteur de recherche s'appuyant sur la bibliothèque de recherche \gls{lucene}, créée par la \gls{apache}}}
\newglossaryentry{lucene}{name=Lucene, description={Une bibliothèque open source en Java et mise à disposition par la \gls{apache} pour l'indexation et la recherche de texte}}
\newglossaryentry{apache}{name=Fondation Apache, description={Une organisation à but non lucratif qui développe des logiciels open source}}

\newglossaryentry{wikidata}{name=WikiData, description={Une base de connaissances collaborative, ouverte à l'amélioration par la communauté, créée dans le but de rassembler et de centraliser les données qui sont utilisées par les divers projets au sein du mouvement Wikimédia. \url{https://www.wikidata.org/}}}
\newglossaryentry{geonames}{name=GeoNames, description={Une base de données géographiques ouverte et gratuite couvrant tous les pays et contenant plus de onze millions de noms de lieux. \url{https://www.geonames.org/}}}

\newacronym{setup}{SETUP}{Schema Evolution Through UPdates}
\newglossaryentry{hitup}{name=HitUp, description={}}
\newglossaryentry{pullup}{name=PullUp, description={}}

\newacronym{nlm}{NLM}{U.S. National Library of Medicine}
\newacronym[description={Un thésaurus de référence dans le domaine biomédical multilangue fournit par la \acs{nlm}. \url{https://mesh.inserm.fr/FrenchMesh/index.htm}}]{mesh}{MeSH}{Medical Subject Headings}
\newacronym[description={Un thésaurus intégrant une terminologie clé, des normes de classification et de codage, ainsi que des ressources associées afin de promouvoir la création de systèmes et de services d'information biomédicale plus efficaces et interopérables, y compris les dossiers médicaux électroniques. \url{https://www.nlm.nih.gov/research/umls/index.html}}]{umls}{UMLS}{Unified Medical Language System}
\newacronym[description={}]{meddra}{MedDRA}{Medical Dictionary for Regulatory Activities}
\newacronym[description={Une collection organisée de termes médicaux fournissant des codes, des termes, des synonymes et des définitions utilisés dans la documentation et les rapports cliniques. \url{https://www.snomed.org/}}]{snomed}{SNOMED}{SNOMED Clinical Terms}
\newacronym[description={}]{rxnorm}{RxNORM}{}

\newacronym{obie}{OBIE}{Ontology-Based Information Extraction}
\newacronym{owl}{OWL}{Web Ontology Language}
\newacronym{owa}{OWA}{Open World Assumption}
\newacronym{cwa}{CWA}{Closed World Assumption}

\newacronym{uri}{URI}{Uniform Resource Identifier}
\newacronym[shortplural=TGDs,longplural=Tuple-Generating Dependencies]{tgd}{TGD}{Tuple-Generating Dependency}
\newacronym{sql}{SQL}{Structured Query Language}
\newacronym[shortplural=SGBDs,longplural=Syst\'emes de G\'estion de Base de Donn\'ees]{sgbd}{SGBD}{Syst\'eme de G\'estion de Base de Donn\'ees}
\newacronym{sparql}{SPARQL}{Simple Protocol and RDF Query Language}
\newacronym{gql}{GQL}{Graph Query Language}
\newacronym{mql}{MQL}{Multidimensional Query Language}
\newacronym[shortplural=LPGs,longplural=Labeled Property Graphs]{lpg}{LPG}{Labeled Property Graph}
\newacronym{w3c}{W3C}{World Wide Web Consortium}
\newacronym{rdf}{RDF}{Resource Description Framework}
\newacronym{rdfs}{RDFS}{Resource Description Framework Schema}
\newglossaryentry{chase}{name=Chase, description={Algorithme de point fixe pour la saturation de contraintes dans une base de données}}
\newglossaryentry{core}{name=Core, description={}}

\newacronym{nac}{NAC}{Negative Application Conditions}
\newacronym{lhs}{LHS}{Left-Hand Side}
\newacronym{rhs}{RHS}{Right-Hand Side}

\newacronym{fol}{FOL}{logique du premier ordre}
\newacronym{ldbc}{LDBC}{Linked Data Benchmark Council}
