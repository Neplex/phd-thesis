% Glossaries
\newglossaryentry{hifun}{name=HiFun, description={Un langage de requête fonctionnel de haut niveau pour l'analyse des Big Data}}
\newglossaryentry{doing}{name=DOING, description={Réseaux discutant de problématique autours de l'extraction, le stockage et le traitement de données médicales. \url{https://www.univ-orleans.fr/lifo/evenements/doing/}}}
\newglossaryentry{wikipedia}{name=Wikipédia, description={Une encyclopédie universelle, multilangue, libre et collaborative. \url{https://wikipedia.org/}}}
\newglossaryentry{python}{name=Python, description={Un language de programmation interprété utilisant un typage dynamique. Il est principalement utilisé dans les sciences des données. \url{https://www.python.org/}}}
\newglossaryentry{cypher}{name=Cypher, description={Un langage de requetage conçu pas Neo4J pour les bases de données graphes basée sur de la reconnaissance de motifs decrit en art ASCII}}
\newglossaryentry{neo4j}{name=Neo4J, description={Un système de gestion de base de données utilisant des graphes de propriétés. \url{https://neo4j.com/}}}
\newglossaryentry{mysql}{name=MySQL, description={Un système de gestion de bases de données relationnelles. \url{https://www.mysql.com/}}}
\newglossaryentry{docker}{name=Docker, description={Docker est une plateforme permettant l'execution dans des environments reproductible et légers. \url{https://www.docker.com/}}}

\newglossaryentry{linux}{name=Linux, description={Linux ou GNU/Linux est une famille de systèmes d'exploitation open source de type Unix fondés sur le noyau Linux}}
\newglossaryentry{rocky}{name=Rocky Linux, description={Rocky Linux est une distribution \gls{linux} basée sur le code source du système d'exploitation Red Hat Enterprise Linux. \url{https://rockylinux.org/}}}
\newglossaryentry{git}{name=GIT, description={Git est un système de gestion de versions décentralisé, libre et gratuit, créé par Linus Torvalds. Il est notament utilisé pour gérer du code source comme le noyau \gls{linux}. \url{https://git-scm.com/}}}
\newglossaryentry{rasa}{name=RASA, description={\url{https://rasa.com/}}}
\newglossaryentry{spacy}{name=SpaCy, description={SpaCy est une bibliothèque logicielle Python de traitement automatique des langues maintenue par ExplosionAI. \url{https://spacy.io/}}}
\newglossaryentry{nltk}{name=NLTK, description={Natural Language Toolkit est une bibliothèque logicielle en Python permettant le traitement automatique des langues créée par luniversité de Pennsylvanie. \url{https://www.nltk.org/}}}
\newglossaryentry{iso}{name=ISO, description={L'organisation internationale de normalisation}}

% Acronyms
\newacronym{lifo}{LIFO}{Laboratoire d'Informatique Fondamentale d'Orl\'eans}
\newacronym{insa}{INSA}{Institut National des Sciences Appliquées}
\newacronym{ged}{GED}{Gestion Electronique de Documents}
\newglossaryentry{ennov}{name=Ennov, description={Editeur de logiciels concevant une plateforme unifiée de gestion du contenu et de l'information pour soutenir le développement des produits des sciences de la vie. \url{https://ennov.com/}}}
\newglossaryentry{gartner}{name=Gartner, description={Une entreprise américaine de conseil et de recherche dans le domaine des techniques avancées}}
\newacronym{rim}{RIM/RIMS}{Regulatory Information Management System}
\newacronym{idmp}{IDMP}{Identification of Medicinal Products}

\newacronym{tal}{TAL}{Traitement Automatique des Langues}
\newacronym{nlp}{NLP}{Natural Language Processing}
\newacronym{nlu}{NLU}{Natural Language Understanding}
\newacronym{ner}{NER}{Named-Entity Recognition}
\newacronym{ie}{IE}{Information Extraction}

\newacronym{setup}{SETUP}{Schema Evolution Through UPdates}
\newglossaryentry{hitup}{name=HitUp, description={}}
\newglossaryentry{pullup}{name=PullUp, description={}}

\newacronym{obie}{OBIE}{Ontology-Based Information Extraction}
\newacronym{owl}{OWL}{Web Ontology Language}
\newacronym{owa}{OWA}{Open World Assumption}
\newacronym{cwa}{CWA}{Closed World Assumption}
\newacronym{graphEntityDependencies}{GED}{Graph Entity Dependencies}

\newacronym{uri}{URI}{Uniform Resource Identifier}
\newacronym[shortplural=TGDs,longplural=Tuple-Generating Dependencies]{tgd}{TGD}{Tuple-Generating Dependency}
\newacronym{sql}{SQL}{Structured Query Language}
\newacronym{sgbd}{SGBD}{Syst\'eme de G\'estion de Base de Donn\'ees}
\newacronym{sparql}{SPARQL}{Simple Protocol and RDF Query Language}
\newacronym{gql}{GQL}{Graph Query Language}
\newacronym{mql}{MQL}{Multidimensional Query Language}
\newacronym{lpg}{LPG}{Labeled Property Graph}
\newacronym{w3c}{W3C}{World Wide Web Consortium}
\newacronym{rdf}{RDF}{Resource Description Framework}
\newacronym{rdfs}{RDFS}{Resource Description Framework Schema}
\newglossaryentry{chase}{name=Chase, description={Algorithme de point fixe pour la saturation de contraintes dans une base de données}}

\newacronym{nac}{NAC}{Negative Application Conditions}
\newacronym{lhs}{LHS}{Left-Hand Side}
\newacronym{rhs}{RHS}{Right-Hand Side}

\newacronym{fol}{FOL}{logique du premier ordre}
\newacronym{ldbc}{LDBC}{Linked Data Benchmark Council}
