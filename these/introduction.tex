\section{Un peu de contexte}

L'action \gls{doing} réunit des chercheurs issus des domaines des bases de données, de l'intelligence artificielle et du traitement du langage naturel.
Son objectif principal est de convertir les données en informations, puis en connaissances.
Plus précisément, elle se concentre sur le stockage des informations dans des bases de données graphes.
Les bases de données graphes sont privilégiées en raison de leur capacité à modéliser de manière efficace les relations complexes entre les données.
Cette modélisation permet une exploitation efficace de ces relations en vue de les transformer en connaissances exploitables.

Les axes de discussion de \gls{doing} se concentrent sur deux problématiques principales :
\begin{itemize}
    \item L'extraction d'informations à partir de données textuelles et leur représentation dans des bases de connaissances.
    \item Le développement de méthodes intelligentes pour le traitement et la maintenance de ces bases de données, incluant de nouvelles formes de requêtes pour l'analyse efficaces, flexibles et sécurisées adaptées à l'utilisateur, ainsi que des garanties de qualité et de protection de la vie privée.
\end{itemize}

Les travaux exposés dans cette thèse coïncidente avec l'action \gls{doing}, le projet de recherche ayant étant initié simultanément à la mise en place de cette action.
Les multiples collaborations entreprises dans le cadre de ces travaux sont principalement liées à cette action.

\paragraph{}
Par ailleurs, ce projet de recherche est mené en collaboration avec la société \gls{ennov}, un éditeur de logiciels spécialisé dans le secteur médical.
\gls{ennov} propose une suite logicielle basée sur une \gls{ged} conçue pour aider les entreprises pharmaceutiques et biomédicales à gérer leurs contenus, automatiser leurs processus métier et se conformer aux normes réglementaires internationales.
Cette suite logicielle permet de réduire les risques, faciliter la conformité et améliorer la prise de décision.
\gls{ennov} est un fournisseur majeur de solutions pour le \gls{rim} et l'\gls{idmp}.
Sa suite logicielle comprend des fonctionnalités de gestion de processus et de contenu pour la qualité, les affaires réglementaires, les études cliniques et la pharmacovigilance.
\gls{ennov} est intéressé à offrir une meilleure automatisation des tâches répétitives, une visibilité accrue dans toute l'organisation et une facilité de mise en œuvre.

\section{Axes de recherche}

Le projet de recherche se focalise sur les bases de données graphes et s'intéresse en particulier aux données issues du domaine médical.
Le projet s'articule autour de deux axes principaux.

\paragraph{}
Dans un premier temps, notre intérêt se focalise sur les bases de données contenant des informations incomplètes.
L'incomplétude des données peut provenir de diverses sources.
Dans le cadre de cette thèse, où l'exploitation des données textuelles est centrale, il est essentiel de prendre en compte cette incomplétude.
On s'intéresse à la maintenance d'une base de données cohérente par rapport à un ensemble de règles.
Dans cette optique, nous souhaitons suivre la politique de mise à jour proposée dans~\cite{chabinConsistentUpdatingDatabases2020}.
Notre objectif est d'étudier la mise en œuvre de cette politique de manière efficace sur un modèle de données de type graphe.
On se pose alors les questions suivantes : Comment imposer un ensemble de contraintes aux bases de données graphe ? Comment garantir la cohérence de ces bases tout au long de leur évolution ? Et comment implémenter cette politique en interaction avec des \gls{sgbd} ?

Bien qu'il existe des approches proposant des contraintes spécifiques pour les graphes, telles que celles présentées dans~\cite{fanDependenciesGraphs2019}, nous avons opté pour une modélisation du graphe de données qui correspond à la définition logique de nos règles.
Cela nous permet de mettre en œuvre la politique de~\cite{chabinConsistentUpdatingDatabases2020} en conservant le pouvoir d'expression des règles tout en améliorant les performances des opérations coûteuses, par rapport au modèle relationnel.
La flexibilité intrinsèque des bases de données graphes nous permet en effet de représenter les mêmes données de différentes manières, permettant d'optimiser les interrogations subséquentes.

\begin{example}
    Supposons l'existence des prédicats suivants : $getTreatment$, qui associe un patient à un traitement, $forPatho$, qui associe un traitement à la pathologie qu'il concerne et le prédicat $hasPatho$, indiquant qu'un patient a une pathologie.
    Dans cet exemple, nous pouvons considérer les règles suivantes :
    \begin{enumerate}[label=\textbf{$r_\arabic*$ :},ref=$r_\arabic*$]
        \item $getTreatment(x, y) \to forPatho(y, z)$ \label{ex:intro:r1}
        \item $getTreatment(x, y), forPatho(y, z) \to hasPatho(x, z)$ \label{ex:intro:r2}
    \end{enumerate}
    La règle~\ref{ex:intro:r1} stipule qu'un traitement est obligatoirement lié à une pathologie, tandis que la règle~\ref{ex:intro:r2} établit que si un patient $x$ reçoit un traitement pour une pathologie $z$, alors ce patient est atteint de cette pathologie ($hasPatho(x, z)$).
    
    Considérons le graphe suivant représentant l'ensemble de faits ${getTreatment(\text{alice}, \text{résection})}$.
    \begin{center}
        \begin{adjustbox}{max width=\linewidth}
            \begin{tikzpicture}[shorten >=2pt,thick,-Latex,node distance=5cm,on grid]
                \node[draw] (x) {alice};
                \node[draw, right=of x] (y) {résection};

                \path
                (x) edge node[anchor=center, fill=white] {getTreatment} (y);
            \end{tikzpicture}
        \end{adjustbox}
    \end{center}

    Dans ce contexte, le graphe ne satisfait pas la règle \ref{ex:intro:r1}.
    Pour respecter cette règle, il faudrait ajouter le fait $forPatho(\text{résection}, N_1)$.
    Remarquons que $N_1$ est une valeur nulle, ce qui signifie qu'une pathologie existe, mais sa valeur exacte est inconnue.
    Cependant, cet ensemble de faits ne respecterait pas non plus la règle \ref{ex:intro:r2}.
    Pour obtenir un ensemble de faits cohérent avec les règles énoncées, il serait nécessaire de construire l'ensemble ${getTreatment(\text{alice}, \text{résection}), forPatho(\text{résection}, N_1), hasPatho(\text{alice}, N_1)}$, représenté par le graphe suivant :
    \begin{center}
        \begin{adjustbox}{max width=\linewidth}
            \begin{tikzpicture}[shorten >=2pt,thick,-Latex,node distance=5cm,on grid]
                \node[draw] (x) {alice};
                \node[draw, right=of x] (y) {résection};
                \node[draw, right=of y] (z) {$N_1$};

                \path
                (x) edge node[anchor=center, fill=white] {getTreatment} (y)
                (y) edge node[anchor=center, fill=white] {forPatho} (z)
                (x) edge [bend right=20] node[anchor=center, fill=white] {hasPatho} (z);
            \end{tikzpicture}
        \end{adjustbox}
    \end{center}
\end{example}

\paragraph{}
Le deuxième volet du projet de recherche se concentre sur les données textuelles, qui sont largement présentes, notamment dans les descriptions de cas cliniques et dans le domaine de la pharmacovigilance/cosmétovigilance.
Les cas cliniques fournissent un historique détaillé de la santé d'un patient dès son admission à l'hôpital, comprenant ses antécédents médicaux, les symptômes observés, les diagnostics établis et les traitements recommandés.
Quant à la pharmacovigilance, elle consiste à surveiller en continu les effets indésirables des médicaments après leur commercialisation, en se basant principalement sur les retours des patients ou des professionnels de santé.
L'objectif consiste à rendre les informations contenues dans les textes exploitables en facilitant leur interrogation et leur analyse.
Pour cela, il est nécessaire de structurer et d'agréger les données textuelles selon des abstractions générales qui correspondent à des entités dans une base de données.
Dans ce cadre, nous explorons les questions suivantes : Comment repérer et extraire les données pertinentes présentes dans le texte ? Comment structurer ces informations de manière à les rendre exploitables et enregistrable dans une base de données ? Et en particulier, quelles abstractions génériques (intermédiaires) nous permettront de passer d'une vision détaillée et multiforme représentée par le texte à une vision plus globale et uniforme requise par une base de données ?

Dans cette thèse, nous partons du principe que notre instance initiale est constituée d'un ensemble de textes qui respectent une grammaire.
L'objectif est de faire évoluer cette grammaire en la généralisant vers une autre grammaire qui représente des abstractions structurantes correspondant à un schéma de données.
Cette évolution est réalisée par une suite de transformation appliquée à l'instance.

\begin{example}
    \label{ex:runex}
    Prenons en compte le texte suivant, extrait d'un cas clinique du corpus CAS~\cite{grabarCASFrenchCorpus2018} :
    \begin{displayquote}
        Un patient de 78 ans suivi pour cancer prostatique avec métastases ganglionnaires ayant déjà subi une résection endoscopique prostatique avec pulpectomie a été admis en urgence pour insuffisance rénale aiguë obstructive à 330 mmol/l de créatinine avec fièvre et urétérohydronéphrose bilatérale à l'échographie.
    \end{displayquote}

    Notre méthode consiste à extraire les informations du texte, telles que les entités nommées.
    Par exemple, nous pouvons identifier des symptômes tels que \enquote{insuffisance rénale}, \enquote{fièvre} ou encore \enquote{urétérohydronéphrose}.
    En supposant que les informations sont déjà structurées par le texte lui-même, notre objectif est de les restructurer afin d'obtenir une représentation uniforme.
    Par exemple, à partir du texte \enquote{Un patient de 78 ans}, nous pourrions construire le fait $Patient(N_1, \text{homme}, 78)$.
    En continuant avec l'extrait de texte précédent, nous pourrions construire l'ensemble de faits suivants :
    \begin{multicols}{2}
        \begin{itemize}
            \item $CasClinique(\textit{c-2-3})$
            \item $hasPatient(\textit{c-2-3}, p_1)$
            \item $Patient(p_1, \text{homme}, 78)$
            \item $hasPatho(p_1, \text{cancer})$
            \item $concernsAnat(\text{cancer}, \text{prostate})$
            \item $leadTo(\text{cancer}, \text{métastases ganglionnaires})$
            \item $getTreatment(p_1, \text{résection endoscopique})$
            \item $concernsAnat(\text{résection endoscopique}, \text{prostate})$
            \item $getTreatment(p_1, \text{pulpectomie})$
            \item $passExam(p_1, e_1)$
            \item $measure(e_1, \text{créatinine}, \SI{330}{\milli\mol\per\L}) $
            \item $reveal(e_1, \text{insuffisance rénale})$
            \item $passExam(p_1, e_2)$
            \item $reveal(e_2, \text{fièvre})$
            \item $passExam(p_1, \textit{echographie})$
            \item $reveal(\textit{echographie}, \textit{urétérohydronéphrose})$
        \end{itemize}
    \end{multicols}

    Ces informations peuvent être représentées par le graphe attribué présenté dans la figure~\ref{fig:runex:graph}.
    Notez que, dans cet exemple, le prédicat $measure$ est promus comme attributs de la relation $exam$.

    \begin{figure}[htb]
        \centering
        \begin{adjustbox}{varwidth=\linewidth,max height=\textheight}
            \small
            \begin{tikzpicture}[shorten >=2pt,thick,-Latex,node distance=3cm and 5cm,on grid]
                \node[labeled node] (patient) {Patient \nodepart{two} name: \emph{$p_1$}\\sex: \emph{homme}\\age: \emph{78}};
                \node[labeled node, left=of patient] (cas) {CasClinique \nodepart{two} doc: \emph{c-2-3}};
                \node[labeled node, above=of patient] (traitement) {Traitement \nodepart{two} name: \emph{résection}\\~~~~~~~~~~\emph{endoscopique}};
                \node[labeled node, left=of traitement] (traitement2) {Traitement \nodepart{two} name: \emph{pulpectomie}};
                \node[labeled node, right=of traitement] (cancer) {Pathologie \nodepart{two} name: \emph{cancer}};
                \node[labeled node, above=of cancer] (prostate) {Anatomie \nodepart{two} name: \emph{prostate}};
                \node[labeled node, below=of cancer] (metastase) {Symptom \nodepart{two} name: \emph{métastases}\\~~~~~~~~~~\emph{ganglionnaires}};

                \node[labeled node, below=3.7cm of cas] (exam1) {Exam \nodepart{two} name: $e_1$};
                \node[labeled node, right=of exam1] (exam2) {Exam \nodepart{two} name: $e_2$};
                \node[labeled node, right=of exam2] (exam3) {Exam \nodepart{two} name: \emph{échographie}};
                \node[labeled node, below=of exam1] (sosy1) {SOSY \nodepart{two} name: \emph{insuffisance}\\~~~~~~~~~~\emph{rénale}};
                \node[labeled node, below=of exam2] (sosy2) {SOSY \nodepart{two} name: \emph{urétérohydronéphrose}\\~~~~~~~~~~\emph{bilatérale}};
                \node[labeled node, below=of exam3] (sosy3) {SOSY \nodepart{two} name: \emph{fièvre}};
    
                \path
                (cas) edge node[labeled edge, anchor=center] {hasPatient} (patient)
                (patient) edge node[labeled edge, anchor=center] {hasPatho} (cancer)
                (patient) edge node[labeled edge, anchor=center] {getTreatment} (traitement)
                (patient) edge node[labeled edge, anchor=center] {getTreatment} (traitement2)
                (patient) edge node[labeled edge, anchor=center] {passExam} (exam1)
                (patient) edge node[labeled property edge, anchor=center] {passExam \nodepart{two} name: \emph{créatinine}\\value: \emph{\SI{330}{\milli\mol\per\L}}} (exam2)
                (patient) edge node[labeled edge, anchor=center] {passExam} (exam3)
                (cancer) edge node[labeled edge, anchor=center] {concernsAnat} (prostate)
                (cancer) edge node[labeled edge, anchor=center] {leadTo} (metastase)
                (traitement) edge node[labeled edge, anchor=center] {forPatho} (cancer)
                (traitement) edge node[labeled edge, anchor=center] {concernsAnat} (prostate)
                (exam1) edge node[labeled edge, anchor=center] {reveal} (sosy1)
                (exam2) edge node[labeled edge, anchor=center] {reveal} (sosy2)
                (exam3) edge node[labeled edge, anchor=center] {reveal} (sosy3)
                ;
            \end{tikzpicture}
        \end{adjustbox}
        \caption{Instance sous forme de graphe}
        \label{fig:runex:graph}
    \end{figure}

    \begin{remark}
        Le graphe proposé ici n'est donné qu'à titre d'exemple et n'est qu'une des nombreuses représentations sous forme de graphe possibles.
        Par exemple, les relations $getTreatment$ peuvent être décomposées en un nœud $Treatment$ et $TreatmentType$.
        Cette représentation permettrait de séparer l'instance de son type et faciliterait la construction de relations autour du type.
        Un raisonnement similaire peut être appliqué pour les examens et les pathologies.
    \end{remark}
\end{example}

Il convient de noter que le choix des outils de \gls{tal} utilisés à cette étape, s'est fait en prenant en considération les contraintes spécifiques à notre domaine d'application :
\begin{itemize}
    \item Les défis liés à la sensibilité des données médicales, qui imposent souvent des restrictions strictes sur leur accès et leur disponibilité.
    Ces restrictions visent à protéger la vie privée des patients et à prévenir tout accès non autorisé ou usage malveillant de ces informations cruciales pour la santé.
    Les modèles d'apprentissage automatique supervisé nécessitent un accès direct aux données pour être entraînés, ce qui peut compromettre la confidentialité des informations médicales des patients.
    Pour pallier cela, des solutions telles que l'apprentissage fédéré ou des approches symboliques plus classiques peuvent être mise en place.

    \item Dans les domaines sensibles comme le domaine médical, la nécessité de construire des solutions explicables est primordiale.
    L'aide à la prise de décision doit être appuyée par des méthodes claires et compréhensibles, surtout étant donné que les décisions prises par les systèmes informatiques peuvent avoir des conséquences directes sur la santé et la vie des patients.
    Une explication claire de la manière dont ces décisions sont prises est essentielle.
    Cela permet aux médecins et aux patients de comprendre pourquoi une recommandation spécifique a été faite.
    De plus, les approches explicables permettent de retracer les décisions et de déterminer la partie du processus qui est responsable en cas d'erreur.
    Favoriser cette transparence contribue à créer une confiance dans le système et à maintenir l'utilisateur au centre du processus décisionnel.
\end{itemize}

\FloatBarrier
\section{Contributions et organisation du manuscrit}

\noindent
Les contributions importantes de nos travaux s'organisent de la manière suivante :

\begin{description}
    \item[\nameref*{part:update}] Présentée dans la partie~\ref{part:update}, la mise à jour cohérente des bases de données revêt une importance capitale pour garantir la qualité des données.
    Le chapitre~\ref{chp:update:intro} offre une vue d'ensemble de la notion de cohérence dans le contexte des bases de données.
    Alors que certaines approches se concentrent sur l'application de règles lors de l'interrogation, nous mettons l'accent sur la cohérence dès les mises à jour.
    Une base de données maintenue cohérente facilite l'analyse et l'interrogation en garantissant un niveau uniforme de qualité des données.
    Nous nous intéressons particulièrement aux bases de données graphes incomplètes, c'est-à-dire, celles comportant des valeurs nulles liées.
    Le chapitre~\ref{chp:update:algos} propose une méthode de mise à jour incrémentale reposant sur un \gls{sgbd} graphe et qui garantit le respect des contraintes tout en assurant qu'il n'y a aucune redondance d'information.
    Ce chapitre présente également une suite d'expérimentations permettant de mesurer les performances de notre solution.

    \item[\nameref*{part:texts}] Nous nous concentrons sur la structuration des données textuelles en vue de l'enregistrement, dans une base de données graphe, des informations essentielles qu'elles apportent.
    En effet, pouvoir structurer et regrouper des informations de manière cohérente pour les maintenir dans un \gls{sgbd} présente plusieurs avantages, notamment l'amélioration des capacités de recherche et la simplification de l'analyse des données.
    Le chapitre~\ref{chp:struct} présente une solution développée pour répondre à la problématique de structuration automatique des données textuelles.
    Plus précisément, nous nous penchons sur la définition d'une structure valide par rapport aux bases de données et sur les méthodes pour la construire automatiquement à partir de données textuelles.
    Pour finir, le chapitre propose une évaluation critique de la solution proposée.

    L'extraction d'information (\acrshort{ie}) est un domaine étroitement lié au \gls{tal}.
    Elle implique généralement l'identification d'entités nommées, leur résolution ou Entity Linking (c'est-à-dire l'association de la mention à un identifiant unique) et l'extraction de relations entre ces entités.
    Dans le chapitre~\ref{chp:tal}, nous examinons plusieurs travaux réalisés en collaboration avec des chercheurs spécialisés dans le \gls{tal}.
    Nous présentons les différentes solutions développées dans cette thèse et mises en œuvre au sein d'\gls{ennov}.
    Il est important de souligner que notre intérêt pour les données textuelles découle de notre partenariat avec \gls{ennov}.
    Le domaine médical englobe une vaste gamme d'informations, allant des images aux modèles 3D, en passant par les rapports textuels sur les cas cliniques ou la pharmacovigilance.
    Cependant, l'écosystème d'\gls{ennov} est principalement axé sur les textes.
    Le chapitre~\ref{chp:tal} termine avec la présentation de deux cas d'application concrets illustrant l'application pratique des méthodes discutées.
\end{description}

\subsection{Publications}

Les travaux présentés dans cette thèse reposent sur une série de collaborations interdisciplinaires, notamment dans le domaine du \gls{tal}.
Ils ont également donné lieu aux publications suivantes :

\begin{bibunit}[alphaurl]
    \nocite{amaviNaturalLanguageQuerying2020}
    \nocite{chabinGraphRewritingRules2020}
    \nocite{minardDOINGDEFTCascade2020}
    \nocite{chabinGraphRewritingRules2021}
    \nocite{hiotDOINGDEFTUtilisation2021}
    \nocite{savaryRelationExtractionClinical2022}
    \nocite{chabinManagingLinkedNulls2023}

    \renewcommand{\bibname}{}
    \renewcommand{\section}[1]{}
    \putbib[these/biblio]
\end{bibunit}

\subsection{Logiciels}

Ces travaux ont abouti à la création de logiciels et de jeux de données pour les expérimentations.
Pour garantir la reproductibilité des résultats, tous les logiciels et jeux de données associés sont rendus disponibles publiquement :

\begin{bibunit}[alphaurl]
    \nocite{chabinSetUpSchemaEvolution2020}
    \nocite{chabinDataFix2023}
    \nocite{chabinArchiTXT2024}

    \renewcommand{\bibname}{}
    \renewcommand{\section}[1]{}
    \putbib[these/biblio]
\end{bibunit}
