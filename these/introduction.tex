\section{Un peu de contexte}

L'action \gls{doing} réunit des chercheurs issus des domaines des bases de données, de l'intelligence artificielle et du traitement du langage naturel.
Son objectif principal est de convertir les données en informations, puis en connaissances.
Plus précisément, elle se concentre sur le stockage des informations dans des bases de données graphes.
Les bases de données graphes sont privilégiées en raison de leur capacité à modéliser de manière efficace les relations complexes entre les données.
Cette modélisation permet une exploitation efficace de ces relations en vue de les transformer en connaissances exploitables.

Les axes de discussion de \gls{doing} se concentrent sur deux problématiques principales :
\begin{itemize}
    \item L'extraction d'informations à partir de données textuelles et leur représentation dans des bases de connaissances.
    \item Le développement de méthodes intelligentes pour le traitement et la maintenance de ces bases de données, incluant de nouvelles formes de requêtes pour l'analyse efficaces, flexibles et sécurisés adaptés à l'utilisateur, ainsi que des garanties de qualité et de protection de la vie privée.
\end{itemize}

\paragraph{}
Les travaux exposés dans cette thèse coïncidente avec l'action \gls{doing}, le projet de recherche ayant étant initié simultanément à la mise en place de cette action.
Les multiples collaborations entreprises dans le cadre de ces travaux sont principalement liées à cette action.
Le projet de recherche se focalise sur les bases de données graphes, mais s'intéresse en particulier aux données issues du domaine médical.
Le projet s'articule autour de deux axes principaux.

Dans un premier temps, il est nécessaire de préciser que les bases de données graphes se distinguent par leur flexibilité intrinsèque.
Cette flexibilité permet de représenter des données à différents niveaux d'abstraction sans imposer de structure fixe.
Dans le cadre de cette thèse, l'accent est mis sur la garantie d'un niveau de qualité suffisant des données.
En particulier, notre intérêt se porte sur les bases de données capables de stocker des données incomplètes.
L'incomplétude des données peut provenir de diverses sources, mais nous explorerons dans cette thèse comment l'exploitation des données textuelles rend nécessaire la prise en compte et la représentation de cette incomplétude.
On s'intéresse alors aux questions suivantes : comment imposer un ensemble de contraintes aux bases de données graphes ? Et comment garantir la cohérence de ces bases tout au long de leur évolution constante ?

Le deuxième volet du projet de recherche se concentre sur les données textuelles, qui sont abondamment présentes, notamment dans les descriptions de cas cliniques et dans le domaine de la pharmacovigilance/cosmétovigilance.
Les cas cliniques fournissent un historique détaillé de la santé d'un patient dès son admission à l'hôpital, comprenant ses antécédents médicaux, les symptômes observés, les diagnostics établis et les traitements recommandés.
La pharmacovigilance consiste quant à elle à surveiller en continu les effets indésirables des médicaments après leur commercialisation, en se basant principalement sur les retours d'expérience des patients ou des professionnels de santé.
Ce volet vise à intégrer ces données textuelles dans notre base de données graphes.
Plus précisément, les questions suivantes sont explorées : comment repérer et extraire les données pertinentes présentes dans le texte ? Et comment structurer ces informations de manière à les rendre exploitables et les sauvegarder dans une base de données ?

De plus, nous nous concentrons sur les contraintes suivantes :
\begin{itemize}
    \item Les défis liés à la sensibilité des données médicales, qui imposent souvent des restrictions strictes sur leur accès et leur disponibilité.
    Ces restrictions visent à protéger la vie privée des patients et à prévenir tout accès non autorisé ou usage malveillant de ces informations cruciales pour la santé.
    Les modèles d'apprentissage automatique supervisé nécessitent un accès direct aux données pour être entraînés, ce qui peut compromettre la confidentialité des informations médicales des patients.
    Pour pallier cela, des solutions telles que l'apprentissage fédéré ou des approches symboliques plus classiques peuvent être mise en place.

    \item Dans les domaines sensibles comme le domaine médical, la nécessité de construire des solutions explicables est primordiale.
    L'aide à la prise de décision doit être appuyée par des méthodes claires et compréhensibles, surtout étant donné que les décisions prises par les systèmes informatiques peuvent avoir des conséquences directes sur la santé et la vie des patients.
    Une explication claire de la manière dont ces décisions sont prises est essentielle.
    Cela permet aux médecins et aux patients de comprendre pourquoi une recommandation spécifique a été faite.
    De plus, les approches explicables permettent de retracer les décisions et de déterminer partie du processus est responsable en cas d'erreur.
    Favoriser cette transparence contribue à créer une confiance dans le système et à maintenir l'utilisateur au centre du processus décisionnel.
\end{itemize}

\paragraph{}
Ce projet de recherche est mené en collaboration avec la société \gls{ennov}, un éditeur de logiciels spécialisé dans le secteur médical.
\gls{ennov} propose une suite logicielle basée sur une \gls{ged} conçue pour aider les entreprises pharmaceutiques et biomédicales à gérer leurs contenus, automatiser leurs processus métier et se conformer aux normes réglementaires internationales.
Cette suite logicielle permet de réduire les risques, faciliter la conformité et améliorer la prise de décision.
\gls{ennov} est un fournisseur majeur de solutions pour le \gls{rim} et l'\gls{idmp}.
Sa suite logicielle comprend des fonctionnalités de gestion de processus et de contenu pour la qualité, les affaires réglementaires, les études cliniques et la pharmacovigilance.
\gls{ennov} est intéressé à offrir une meilleure automatisation des tâches répétitives, une visibilité accrue dans toute l'organisation et une facilité de mise en œuvre. 

\section{Exemple}
\begin{quote}
    Un patient de 78 ans suivi pour cancer prostatique avec métastases ganglionnaires ayant déjà subi une résection endoscopique prostatique avec pulpectomie a été admis en urgence pour insuffisance rénale aiguë obstructive à 330 mmol/l de créatinine avec fièvre et urétérohydronéphrose bilatérale à l'échographie.
\end{quote}

\begin{multicols}{2}%
    \small%
\begin{align}
    CasClinique(\textit{c-2-3}) \\
    hasPatient(\textit{c-2-3}, p1) \\
    Patient(p_1, 78, H) \\
    hasPatho(p_1, cancer) \\
    concernsAnat(cancer, prostate) \\
    Anatomy(prostate) \\
    leadTo(cancer, \textit{métastases ganglionnaires}) \\
    getTreatment(\textit{résection endoscopique}, prostate) \\
    forPatho(\textit{résection endoscopique}, cancer)
\end{align}

\begin{align}
    exam(p_1, e_1, et_1, \textit{01-01-01}) \\
    paramClinique(e_1, creatine, \textit{330mmol/L}) \\
    reveal(e_1, \textit{insuffisance rénale}) \\
    exam(p_1, e_2, et_2, \textit{01-01-01}) \\
    paramClinique(e_2, temperature, t_1) \\
    reveal(e_2, fievre) \\
    exam(p_1, e_3, \textit{echographie}, \textit{01-01-01}) \\
    paramClinique(e_3, param_1, res_1) \\
    reveal(e_3, \textit{urétérohydronéphrose bilatérale})
\end{align}
\end{multicols}

\begin{figure}
    \small
    \centering
    \begin{adjustbox}{varwidth=\linewidth,max height=\textheight}
        \begin{tikzpicture}[shorten >=2pt,thick,-Latex,node distance=3cm and 5cm,on grid]
            \node[labeled node] (patient) {Patient \nodepart{two} name: \emph{$p_1$}\\sex: \emph{H}\\age: \emph{78}};
            \node[labeled node, below=of patient] (exam1) {Exam \nodepart{two} name: \emph{constantes}};
            \node[labeled node, below=of exam1] (param1) {ParamClinique \nodepart{two} name: \emph{créatinine}\\value: \emph{330 mmol/l}};
            \node[labeled node, below=of param1] (sosy1) {SOSY \nodepart{two} name: \emph{insuffisance}\\~~~~~~~~~~\emph{rénale}};
            \node[labeled node, right=of exam1] (exam2) {Exam \nodepart{two} name: \emph{échographie}};
            \node[labeled node, below=of exam2] (param2) {ParamClinique \nodepart{two} name: \emph{unknown}\\ value: \emph{unknown}};
            \node[labeled node, below=of param2] (sosy2) {SOSY \nodepart{two} name: \emph{urétérohydronéphrose}\\~~~~~~~~~~\emph{bilatérale}};
            \node[labeled node, left=of param1] (param3) {ParamClinique \nodepart{two} name: \emph{temperature}\\ value: \emph{unknown}};
            \node[labeled node, below=of param3] (sosy3) {SOSY \nodepart{two} name: \emph{fiévre}};
            \node[labeled node, above=of param3] (cas) {CasClinique \nodepart{two} doc: \emph{c-2-3}};
            \node[labeled node, above=of cas] (testicule) {Anatomie \nodepart{two} name: \emph{testicule}};
            \node[labeled node, above=of patient] (traitement) {Traitement};
            \node[labeled node, above=of traitement] (traitmentType) {TraitmentType \nodepart{two} name: \emph{résection}\\~~~~~~~~~~\emph{endoscopique}};
            \node[labeled node, above=of testicule] (traitement2) {Traitement};
            \node[labeled node, above=of traitement2] (traitmentType2) {TraitmentType \nodepart{two} name: \emph{pulpectomie}};
            \node[labeled node, above=of exam2] (metastase) {Symptom \nodepart{two} name: \emph{métastases}\\~~~~~~~~~~\emph{ganglionnaires}};
            \node[labeled node, above=of metastase] (cancer) {Pathologie \nodepart{two} name: \emph{cancer}};
            \node[labeled node, above=of cancer] (prostate) {Anatomie \nodepart{two} name: \emph{prostate}};

            \path
            (cas) edge node[labeled edge, anchor=center] {hasPatient} (patient)
            (patient) edge node[labeled edge, anchor=center] {hasPatho} (cancer)
            (patient) edge node[labeled edge, anchor=center] {getTreatment} (traitement)
            (cancer) edge node[labeled edge, anchor=center] {concernsAnat} (prostate)
            (patient) edge node[labeled edge, anchor=center] {getTreatment} (traitement2)
            (cancer) edge node[labeled edge, anchor=center] {leadTo} (metastase)
            (traitement) edge node[labeled edge, anchor=center] {forPatho} (cancer)
            (traitement) edge node[labeled edge, anchor=center] {concernsAnat} (prostate)
            (traitement) edge node[labeled edge, anchor=center] {hasType} (traitmentType)
            (traitement2) edge node[labeled edge, anchor=center] {concernsAnat} (testicule)
            (traitement2) edge node[labeled edge, anchor=center] {hasType} (traitmentType2)
            (patient) edge node[labeled edge, anchor=center] {passExam} (exam1)
            (exam1) edge node[labeled edge, anchor=center] {reveal} (param1)
            (param1) edge node[labeled edge, anchor=center] {show} (sosy1)
            (patient) edge node[labeled edge, anchor=center] {passExam} (exam2)
            (exam2) edge node[labeled edge, anchor=center] {reveal} (param2)
            (param2) edge node[labeled edge, anchor=center] {show} (sosy2)
            (exam1) edge node[labeled edge, anchor=center] {reveal} (param3)
            (param3) edge node[labeled edge, anchor=center] {show} (sosy3)
            ;
        \end{tikzpicture}
    \end{adjustbox}
    \caption{Instance sous forme de graphe}
    \label{fig:runex:graph}
\end{figure}

\FloatBarrier

\section{Contributions et organisation du manuscrit}

\noindent
Les contributions importantes de nos travaux s'organisent de la manière suivante :

\begin{description}
    \item[Mise à jour cohérente] Présentée dans la partie~\ref{part:update}, la mise à jour cohérente des bases de données revêt une importance capitale pour garantir la qualité des données.
    Le chapitre~\ref{chp:update:intro} offre une vue d'ensemble de la notion de cohérence dans le contexte des bases de données.
    Alors que certaines approches se concentrent sur l'application de règles lors de l'interrogation, nous mettons l'accent sur la cohérence dès les mises à jour.
    Une base de données maintenue cohérente facilite l'analyse et l'interrogation en garantissant un niveau uniforme de qualité des données.
    Nous nous intéressons particulièrement aux bases de données graphes pouvant inclure des données incomplètes représentées par des valeurs nulles liés.
    Le chapitre~\ref{chp:update:intro} s'attarde en particulier sur deux travaux qui tente de répondre à cette problématique : \acrfull{setup}~\cite{chabinUsingGraphGrammar2019} et \acrfull{gfd}~\cite{fanDependenciesGraphs2019}.

    Bien que des travaux existent sur les données incomplètes et sur les graphes, nous n'avons connaissance d'aucun travail qui combine ces deux aspects en donnant priorité à la mise à jour.
    Le chapitre~\ref{chp:update:algos} présente nos travaux sur cet aspect, qui font suite à des travaux précédents \cite{chabinConsistentUpdatingDatabases2020} concernant des mises à jour en mémoire nécessitant de recalculer l'ensemble de la base pour garantir la cohérence.
    Nous proposons une méthode de mise à jour itérative reposant sur un \gls{sgbd} graphe et qui garantit le respect des contraintes tout en assurant qu'il n'y a aucune redondance d'information.
    Ce chapitre présente également une suite d'expérimentations permettant de mesurer les performances de notre solution.

    \item Le domaine médical repose sur une grande diversité d'information, qu'il s'agisse d'images, de modèles 3D ou encore de rapports textuels sur des cas cliniques ou de pharmacovigilance.
    Notre travail se concentre sur la valorisation des données textuelles abondantes dans l'écosystème d'\gls{ennov}.
    Structurer ces données textuelles présente plusieurs avantages, notamment l'amélioration des capacités de recherche et la simplification de l'analyse des données.
    Nous nous concentrons particulièrement sur la structuration des données contenues dans les textes en vue de leur enregistrement dans une base de données graphes.
    Le chapitre~\ref{chp:struct} présente une solution développée pour répondre à la problématique de structuration automatique des données textuelles.
    Plus précisément, nous nous penchons sur la définition d'une structure valide et sur les méthodes pour la construire automatiquement.
    Pour finir, le chapitre propose une évaluation critique de la solution proposée.

    \item[Extraction d'information] L'extraction d'information (\acrshort{ie}) est un domaine étroitement lié au \gls{tal}.
    Elle implique généralement l'identification d'entités nommées, leur résolution ou Entity Linking (c'est-à-dire l'association de la mention à un identifiant unique) et l'extraction de relations entre ces entités.
    Dans le chapitre~\ref{chp:tal}, nous examinons plusieurs travaux réalisés en collaboration avec des chercheurs spécialisés dans le \gls{tal}.
    Nous présentons les différentes solutions développées dans cette thèse et mises en œuvre au sein d'\gls{ennov}.
    Enfin, ce chapitre se conclura par la présentation de deux cas d'application concrets illustrant l'application pratique des méthodes discutées.
\end{description}

\section{Publications}

Les travaux présentés dans cette thèse reposent sur une série de collaborations interdisciplinaires, notamment dans le domaine du \gls{tal}.
Dans un souci de reproductibilité des résultats, ces travaux ont conduit à la conception de logiciels et de jeux de données accessibles publiquement.
Ils ont également donnés lieux aux publications suivantes :

\begin{bibunit}[alphaurl]
    \nocite{amaviNaturalLanguageQuerying2020}
    \nocite{chabinGraphRewritingRules2020}
    \nocite{minardDOINGDEFTCascade2020}
    \nocite{chabinGraphRewritingRules2021}
    \nocite{hiotDOINGDEFTUtilisation2021}
    \nocite{savaryRelationExtractionClinical2022}
    \nocite{chabinManagingLinkedNulls2023}

    \renewcommand{\bibname}{}
    \renewcommand{\section}[1]{}
    \putbib[these/biblio]
\end{bibunit}
