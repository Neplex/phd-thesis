Dans ce chapitre, nous avons exploré une approche hybride reposant sur la réécriture d'arbre pour résoudre le défi de la structuration automatique de l'information en vue de son enregistrement dans une base de données.
Une des contributions principale est à formalisation d'un schéma de base de données valide à l'aide d'une méta-grammaire S-attribuée.
Nous avons proposé une succession d'étapes clés pour la structuration automatique de données et la conception d'un schéma de base de données valide.
Nous avons également développé une implémentation de ces étapes avec comme critère objectif la minimisation de la grammaire.
Pour atteindre cet objectif, l'implémentation cherche à optimiser la taille des groupes identifiés tout en réduisant la variabilité à l'aide d'une mesure de fréquence.
Enfin, nous avons procédé à une analyse critique de cette implémentation, confirmant sa capacité à atteindre l'objectif fixé tout en évaluant la qualité du schéma de base de données produit.

Bien que ces travaux portent sur la structuration de données textuelles, la polyvalence de la structure intermédiaire utilisée peut donc également être adaptée pour représenter divers formats de données (comme illustré dans l'évaluation et dans \cite{barretAbstraGenericAbstractions2022}).
Cette caractéristique confère au système une capacité d'intégration de données structurées ou semi-structurées, ouvrant ainsi la voie à des applications potentielles dans le domaine de l'intégration de données.
De plus, l'évaluation de l'implémentation proposée a mis en évidence les avantages de l'approche reposant sur une mesure de fréquence, permettant une adaptation continue du processus de structuration à mesure que les données sont corrigées et intégrées dans la base.
Néanmoins, des limitations persistent, notamment en termes d'efficacité pour le traitement de volumes de données importants.
Des travaux supplémentaires sont donc nécessaires pour améliorer la performance de l'implémentation, tout en explorant de nouvelles fonctions objectif.

L'approche développée dans ce chapitre représente une étape significative pour l'intégration structurée de données textuelles dans une base de données.
Sa caractéristique principale réside dans sa capacité à être déployée sans nécessiter de données d'apprentissage, la rendant ainsi facilement adaptable à divers domaines d'application.
De plus, sa transparence permet à l'utilisateur de suivre et de valider chaque étape du processus de structuration, permettant d'assurer la confiance dans le système et la qualité des données produite.
Toutefois, malgré la cohérence générale du schéma obtenu, notre évaluation a révélé la présence d'erreurs, appuyé par le manque de mesures quantitatives pour évaluer la qualité des schémas construits.
Pour améliorer cette qualité, plusieurs pistes de recherche sont envisageables, telles que la prise en compte de règles métier spécifiques ou l'incorporation d'un processus de détection automatique de dépendances fonctionnelles \cite{papenbrockFunctionalDependencyDiscovery2015}.
En outre, l'approche se base uniquement sur les entités extraites depuis le texte et suppose que l'annotation sémantique et la résolution de la co-référence a déjà été effectuée.
Cependant, il serait possible d'incorporer les relations extraites dans le processus.

La structuration automatique de données peut parfois aboutir à des instances incomplètes, où certaines entités peuvent manquer dans l'instance d'un groupe.
Cette incomplétude peut être imputée à divers facteurs tels que des omissions dans le texte ou des erreurs lors du processus d'extraction.
Cette notion d'instances incomplètes est alors représentée en utilisant des valeurs nulles, ce qui fait écho avec le concept de mise à jour de base de données incomplète présenté dans le chapitre~\ref{chp:update:algos}.
Les méthodes mises en œuvre pour garantir la cohérence dans le chapitre~\ref{chp:update:algos} ajoutent ainsi une étape supplémentaire visant à assurer la qualité de l'instance obtenue.
