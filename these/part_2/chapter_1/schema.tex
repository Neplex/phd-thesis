% L'organisation de données non structurées comme du texte présente des défis importants.
% Les bases de données sont des structures complexes dont l'apparence peut différer considérablement.
% Cependant, on retrouve toujours une structure intrinsèque composée des données nommées (ou attributs) organisé en groupe (tuples ou nœuds) eux même regrouper par type (table ou label).
% On identifie aussi certains groupes spéciaux qui connecte des groupes entre eux et qu'on appelle relations.
% C'est le fondement du \gls{mea} \cite{chenEntityrelationshipModelUnified1976}.
% \cite{barretGenericAbstractionsData2021} se base sur le \gls{mea} pour identifier des connexions entre des jeux de données hétérogènes (XML, \gls{rdf}, etc).
% Les auteurs introduisent la notion de \emph{sub-record} pour une donnée, de \emph{record} pour les groupements de données et de \emph{collections} sous forme d'ensemble de \emph{record}. Les jeux de données sont ainsi représenté comme des forets où chaque arbre relie des \emph{sub-record}, \emph{record} et \emph{collection} en respectant des contraintes spécifique : un \emph{record} ne peux avoir que des \emph{sub-record} comme enfants, une collection est hétérogène et ne peux donc pas contenir différents \emph{records}.

% Il en va de même pour les textes qui possèdent plusieurs structures.
% La grammaire du français ou de l'anglais définit comment les phrases doivent être construites.
% L'idée est de pouvoir permettre la transformation de la grammaire du texte vers une autre (celle du \gls{mea}).

% \begin{definition}[Structure de donnée universelle]
%   Si une instance $I$ est un arbre, ce dernier est reconnu par une grammaire $G_S$ qui est le schéma de l'instance.
%   Ici, on traitera $I$ comme un arbre de dérivation de $G_S$, la grammaire est donc une \gls{cfg} et non une grammaire d'arbre.
%   Cependant, un schéma n'est valide que s'il est de la forme \gls{mea} et qu'il respecte les règles sémantiques décrites dans \cite{barretGenericAbstractionsData2021}.
%   C'est-à-dire que l'on peut définir une méta-grammaire $G$ qui reconnait l'ensemble des grammaires de schéma valides.
%   $G$ est défini comme une méta-grammaire S-attribuée et est donnée figure~\ref{table:struct:meta} en utilisant la \gls{bnf}.
%   $\langle X_A \rangle$ est un méta-non-terminaux avec pour attributs l'ensemble $A$, $\textsc{eol}$ est le séparateur entre les règles de la grammaire de destination et on retrouve à droite de chaque règle de production l'ensemble de formule sémantique $\Phi$.
%   Les attributs sont définis comme suit :
%   \begin{itemize}
%     \item $name$, $grpName$, $relName$ sont des attributs qui prennent le nom d'une instance d'un non-terminal. Par exemple, pour le non-terminal $ENT$ on peut obtenir dans la grammaire de destination les instances $ENT_{person}$ et $ENT_{exam}$
%     \item $eL$ est un ensemble de nom d'entités
%     \item $gL$ est un ensemble de nom de groupes
%     \item $cgL$ est un ensemble de nom de collection de groupes
%     \item $rL$ est un ensemble de nom de relations
%     \item $crL$ est un ensemble de nom de collection de relations
%   \end{itemize}
%   Dans la méta-grammaires ces attributs servent (en plus de la validation des formules sémantiques) à garder une trace des règles de production qui ont été construite afin de vérifier que les grammaires produites sont valides, c.-à-d. que tout non-terminal qui apparait dans une règle de production est bien défini dans une autre règle.
% \end{definition}

% La section~\ref{} présente les différentes structuration possible du langage naturel.
% En ce qui concerne la structuration automatique du texte, l'utilisation des arbres de syntaxe parait plus appropriée étant donné leur ressemblance avec \gls{xml} (les données / mots sont les feuilles de l'arbre et la structure est portée par les nœuds intermédiaires).
% Bien que leur structure est plus conforme à la grammaire cible $\mathcal{G}$, cela entraîne une perte d'information par rapport à l'arbre de dépendance qui peut mettre en relation des éléments éloignés.
% Cependant, les textes médicaux se caractérisent par des phrases courtes et factuelles, où les observations démontrent que les éléments liés sémantiquement sont généralement proches dans le texte \cite{savaryRelationExtractionClinical2022}.

Dans cette thèse notre objectif est de transformer des données textuelles en une instance de graphe (ou d'arbre) respectant une grammaire cible $G$.
La construction de cette grammaire cible implique un processus d'extraction de schémas, commençant par la transformation des phrases en arbres syntaxiques.
Le processus consiste en plusieurs itérations et est résumé visuellement dans la Figure~\ref{fig:struct:general}, où, sur l'axe vertical, à chaque étape $i$, nous considérons :
\begin{itemize}
    \item Une instance $I_i$ sous forme d'une forêt d'arbres syntaxiques enrichis.
    \item Un arbre quotient $S_i$ construit à partir des arbres de $I_i$.
    \item Une grammaire d'arbre $G_{i}$ (pour les arbres dans $I_i$) dérivée de l'arbre quotient $S_i$.
\end{itemize}

L'axe horizontal de la figure~\ref{fig:struct:general} représente la progression du processus de l'étape $i$ à l'étape $i+1$. Cette progression est basée sur le raisonnement suivant :
\begin{description}
    \item[Initialisation] Le processus commence par la transformation des phrases en arbres syntaxique, ce qui donne lieu à une instance désignée par $I_0$.

    \item[Enrichissement] Une seconde étape consiste à enrichir les arbres de l'instance $I_0$ en incorporant les informations extraites par une analyse du texte.
          Il peut s'agir d'insérer des entités, des noms de relations ou d'autres informations pertinentes dans les arbres.

    \item[Évolutions successives] :

          \begin{enumerate}
              \item \textbf{Évolution de l'instance.}
                    L'évolution de l'instance $I_i$ vers l'instance $I_{i+1}$ suit un processus de transformation.
                    Au cours de cette étape, les branches de l'arbre sont regroupées ou transformées sur la base de mesures de similarité. Il peut s'agir de réorganiser la structure des arbres afin d'améliorer leur cohérence ou de les aligner plus efficacement sur la représentation souhaitée.

              \item \textbf{Évolution de la grammaire.}
                    L'évolution de $G_i$ vers $G_{i+1}$ est déclenchée en vérifiant si la grammaire $G_i$ est conforme à la méta-grammaire générale $\mathbb{G}$.
                    À l'étape $i$, si $G_i$ n'est pas conforme à $\mathbb{G}$, le processus se poursuit en transformant les structures des arbres de $I_i$ donnant lieu à l'instance $I_{i+1}$ qui engendre un nouveau graphe quotient $S_{i+1}$ et une nouvelle grammaire $G_{i+1}$.
                    Le processus se termine lorsque nous trouvons une grammaire qui satisfait $\mathbb{G}$.
          \end{enumerate}
\end{description}

\begin{figure}[htb]
    \centering
    \begin{tabular}{llllllll|l}
        Données textuelles & $\rightarrow$
                           & $I_0$         & $\rightarrow$ & $ I_1$ & $\rightarrow$ & $ \dots$ & $I_t$    & Instances (arbres)                    \\
                           &               & $\downarrow$  &        & $\downarrow$  &          &          & $\downarrow$                          \\
                           &               & $S_0$         &        & $S_1$         &          & $ \dots$ & $S_t$              & Arbres quotients \\
                           &               & $\downarrow$  &        & $\downarrow$  &          &          & $\downarrow$                          \\
                           &               & $G_0$         &        & $G_1$         &          & $ \dots$ & $G$                & Grammaires       \\
                           &               & $\searrow$    &        & $\downarrow$  &          &          & $\swarrow$                            \\
        \multicolumn{9}{c}{\fbox{Vérification par rapport à $\mathbb{G}$}}
    \end{tabular}
    \caption{Procéssus itératif pour la structuration automatique}
    \label{fig:struct:general}
\end{figure}

L'objectif de la méta-grammaire $\mathbb{G}$ est d'établir la description générale des grammaires qui serviront pour structurer des textes.
Lorsqu'on considère une instance $I$, les non-terminaux d'une grammaire $G$ sont définis pour représenter les concepts suivants, qui seront appliqués aux arbres de $I$ :

\begin{description}
    \item[Arbre d'entité] Un arbre de profondeur $1$ dont la racine a un nom d'entité comme étiquette et dont les enfants sont des terminaux.

    \item[Groupe] Un arbre de profondeur $2$ dont tous les enfants sont des arbres d'entités.

    \item[Relation] Un arbre de profondeur $3$ ayant exactement deux groupes avec des étiquettes distinctes comme enfants.

    \item[Collection] Un arbre de profondeur $3$ n'ayant pour enfants que des groupes équivalents ou un arbre de profondeur $4$ n'ayant pour enfants que des relations équivalentes.
\end{description}

Il est intéressant de remarquer que ces concepts s'alignent sur l'approche présentée dans~\cite{barretAbstraGenericAbstractions2022}.
Les arbres d'entités peuvent être assimilés au concept de \emph{sub-records} qui se réfèrent à une paire attribut-valeur.
Dans $I_t$, l'instance finale de la figure~\ref{fig:struct:general}, les groupes correspondent aux \emph{records} de~\cite{barretAbstraGenericAbstractions2022}, alors que dans les grammaires $G_i$ (obtenues dans le processus de construction de $G$ de la figure~\ref{fig:struct:general}), ils servent de formats intermédiaires.
Le concept \textit{collection} est identique à celui introduit dans~\cite{barretAbstraGenericAbstractions2022}.

\begin{landscape}
    \centering
    \begin{adjustbox}{max width=\linewidth,max height=.95\textheight,valign=c}
        \parbox{\linewidth}{\begin{align}
                \epsilon                                    & ::= \langle root_{eL',gL',cgL',rL',crL'} \rangle ~\textsc{eol}~ \langle ruleList_{eL,gL,cgL,rL,crL} \rangle                                                                                                                                                                                                                                                                                  & [eL' \subseteq eL; gL' \subseteq gL; cgL' \subseteq cgL; rL' \subseteq rL; crL' \subseteq crL]                    \label{meta:start}      \\
                \langle root_{eL,gL,cgL,rL,crL} \rangle     & ::= \lambda \to \langle rootList_{eL,gL,cgL,rL,crL} \rangle                                                                                                                                                                                                                                                                                                           \label{meta:root}                                                                                                                                                  \\[1em]
                % Root list
                \langle rootList_{eL,gL,cgL,rL,crL} \rangle & ::= \epsilon                                                                                                                                                                                                                                                                                                                                                                                 & [eL \gets \emptyset; gL \gets \emptyset; cgL \gets \emptyset; rL \gets \emptyset; crL \gets \emptyset]            \label{meta:rootList:1} \\
                                                            & ~~ \mid ~ ENT_{name}  ~ \langle rootList_{eL',gL,cgL,rL,crL} \rangle                                                                                                                                                                                                                                                                                                                         & [name \notin eL'; eL \gets \{name\} \cup eL']                                                                     \label{meta:rootList:2} \\
                                                            & ~~ \mid ~ GROUP_{name} ~ \langle rootList_{eL,gL',cgL,rL,crL} \rangle                                                                                                                                                                                                                                                                                                                        & [name \notin gL'; gL \gets \{name\} \cup gL']                                                                     \label{meta:rootList:3} \\
                                                            & ~~ \mid ~ REL_{name}  ~ \langle rootList_{eL,gL,cgL,rL',crL} \rangle                                                                                                                                                                                                                                                                                                                         & [name \notin rL'; rL \gets \{name\} \cup rL']                                                                     \label{meta:rootList:4} \\
                                                            & ~~ \mid ~ COLL_{name} ~ \langle rootList_{eL,gL,cgL',rL,crL} \rangle                                                                                                                                                                                                                                                                                                                         & [name \notin cgL'; cgL \gets \{name\} \cup cgL']                                                                  \label{meta:rootList:5} \\
                                                            & ~~ \mid ~ COLL_{name} ~ \langle rootList_{eL,gL,cgL,rL,crL'} \rangle                                                                                                                                                                                                                                                                                                                         & [name \notin crL'; crL \gets \{name\} \cup crL']                                                                  \label{meta:rootList:6} \\[1em]
                % Rules list
                \langle ruleList_{eL,gL,cgL,rL,crL} \rangle & ::= \epsilon                                                                                                                                                                                                                                                                                                                                                                                 & [eL \gets \emptyset; gL \gets \emptyset; cgL \gets \emptyset; rL \gets \emptyset; crL \gets \emptyset]            \label{meta:ruleList:1} \\
                                                            & ~~ \mid ~ \langle entity_{name}          \rangle ~\textsc{eol}~ \langle ruleList_{eL',gL,cgL,rL,crL} \rangle                                                                                                                                                                                                                                                                                 & [name \notin eL'; eL \gets \{name\} \cup eL']                                                                     \label{meta:ruleList:2} \\
                                                            & ~~ \mid ~ \langle group_{name, eL'}      \rangle ~\textsc{eol}~ \langle ruleList_{eL,gL',cgL,rL,crL} \rangle                                                                                                                                                                                                                                                                                 & [name \notin gL' \land eL' \subseteq eL; gL \gets \{name\} \cup gL']                                              \label{meta:ruleList:3} \\
                                                            & ~~ \mid ~ \langle relation_{name, gL'}   \rangle ~\textsc{eol}~ \langle ruleList_{eL,gL,cgL,rL',crL} \rangle                                                                                                                                                                                                                                                                                 & [name \notin rL' \land gL' \subseteq gL; rL \gets \{name\} \cup rL']                                              \label{meta:ruleList:4} \\
                                                            & ~~ \mid ~ \langle collGrp_{name,grpName} \rangle ~\textsc{eol}~ \langle ruleList_{eL,gL,cgL',rL,crL} \rangle                                                                                                                                                                                                                                                                                 & [name \notin cgL' \land grpName \in gL; cgL \gets \{name\} \cup cgL']                                             \label{meta:ruleList:5} \\
                                                            & ~~ \mid ~ \langle collRel_{name,relName} \rangle ~\textsc{eol}~ \langle ruleList_{eL,gL,cgL,rL,crL'} \rangle                                                                                                                                                                                                                                                                                 & [name \notin crL' \land relName \in rL; crL \gets \{name\} \cup crL']                                             \label{meta:ruleList:6} \\[1em]
                % Groups
                \langle group_{name, eL} \rangle            & ::= GROUP_{name} \to \langle entList_{eL} \rangle                                                                                                                                                                                                                                                                                                                     \label{meta:group}                                                                                                                                                 \\
                \langle collGrp_{name,grpName} \rangle      & ::= COLL_{name} \to GROUP_{grpName}^+                                                                                                                                                                                                                                                                                                                                 \label{meta:collGroup}                                                                                                                                             \\[1em]
                % Relations
                \langle relation_{name, gL} \rangle         & ::= REL_{name} \to GROUP_{name1} ~ GROUP_{name2}                                                                                                                                                                                                                                                                                                                                             & [name1 \neq name2; gL \gets \{name1, name2\}]                                                                     \label{meta:rel}        \\
                \langle collRel_{name,relName} \rangle      & ::= COLL_{name} \to REL_{relName}^+                                                                                                                                                                                                                                                                                                                                   \label{meta:collRel}                                                                                                                                               \\[1em]
                % Entities
                \langle entList_{eL} \rangle                & ::= ENT_{name}                                                                                                                                                                                                                                                                                                                                                                               & [eL \gets \{name\}]                                                                                               \label{meta:entList:1}  \\
                                                            & ~~ \mid ~ ENT_{name} ~ \langle entList_{eL'} \rangle                                                                                                                                                                                                                                                                                                                                         & [name \notin eL'; eL \gets \{name\} \cup eL']                                                                     \label{meta:entList:2}  \\
                \langle entity_{name} \rangle               & ::= ENT_{name} \to \langle data \rangle \label{meta:entity}
            \end{align}}
    \end{adjustbox}
    \captionof{table}[Méta-grammaire S-attribuée $\mathbb{G}$ au format BNF]{Méta-grammaire S-attribuée $\mathbb{G}$ au format \glsname{bnf}. Cette grammaire définit des grammaires qui seront vues comme des schémas de base de donnée \label{table:struct:meta}}
\end{landscape}

\paragraph{Méta-grammaire}
La table~\ref{table:struct:meta} présente notre méta-grammaire $\mathbb{G}$.
Il s'agit d'une grammaire attribuée qui définit tous les schémas (ou les structurations) valides pour nos données.
Il est important de rappeler qu'ici un schéma de base de données est représenté par une grammaire hors contexte.
Dans cette méta-grammaire $\mathbb{G}$ les méta-non-terminaux sont représentés par des termes entre chevrons $\langle \cdot \rangle$.
Les règles sémantiques sont présentées dans la partie droite de la table~\ref{table:struct:meta}, entre crochets $[ \cdot ]$.

\paragraph{}
La première méta-règle de production de $\mathbb{G}$ (\ref{meta:start}) indique que la grammaire cible $G$ est définie par une règle initiale suivie par une liste de règles, éventuellement vide.
La règle initiale engendrée par $\mathbb{G}$ (méta-règle~\ref{meta:root}) contient le symbole \emph{ROOT} (non-terminal initial de $G$) du côté gauche.
Son côté droit est défini par les méta-règles \ref{meta:rootList:1}-\ref{meta:rootList:6} qui spécifie la construction d'une série de non-terminaux de $G$.
Ces non-terminaux sont : \emph{ENT}, \emph{GROUP}, \emph{REL} and \emph{COLL}, représentant, respectivement, des entités, des groupes d'entités, des relations entre des groupes et des collections de groupes ou de relations.
Pour distinguer chaque structuration spécifique à une grammaire $G$, on associe un attribut $name$ à chaque non-terminal.
Les attributs de $\mathbb{G}$ sont synthétisés et représentent des listes de noms ($name$) utilisés comme identificateur :
\begin{itemize}
    \item $eL$ (ou $eL'$) : liste contenant les noms des entités,
    \item $gL$ (ou $gL'$) : liste contenant les noms des groupes,
    \item $rL$ (ou $rL'$) : liste contenant les noms des relations,
    \item $cgL$ (ou $cqL'$) : liste contenant les noms des collections de groupes,
    \item $crL$ (ou $crL'$) : liste contenant les noms des collections de relations.
\end{itemize}

Leur initialisation commence du bas vers le haut.
Les règles syntaxiques permettent, notamment de vérifier qu'un nom est unique.
Par exemple, lors de l'application de la méta-règle \ref{meta:entList:1}-\ref{meta:entList:2}, la liste $eL$ reçoit des noms d'entité qui sont uniques.
Cela est aussi le cas pour toutes les autres listes : l'unicité du nom d'un nouveau non-terminal est assurée par les règles sémantiques.
Il est également essentiel de noter que ces règles sémantiques garantissent que chaque non-terminal apparaissant du côté droit d'une règle de production de la grammaire $G$ doit avoir une règle le définissant.
Par exemple, lors de l'application de la méta-règle \ref{meta:start}, tous les éléments de la liste $gL'$ doivent être présents dans $gL$.
Dans l'exemple de la figure~\ref{table:struct:meta:ex}, si $gL' = {A}$ et $gL = {A, B}$, le non-terminal $GROUP_A$ présent dans la règle $ROOT \rightarrow GROUP_A ~ COLL_1$ à gauche de l'arbre de dérivation (voir figure~\ref{table:struct:meta:ex}) est défini par la règle $GROUP_A \rightarrow ENT_1$, comme illustré dans la partie droite de cet arbre de dérivation (voir figure~\ref{table:struct:meta:ex}).

\begin{example}
    \label{ex:struct:meta}
    La figure~\ref{table:struct:meta:ex} illustre la dérivation de la grammaire $G$ via la méta-grammaire $\mathbb{G}$.
    En bleu on note, pour chaque nœud, la valeur de chaque attribut en omettant, pour des raisons de visibilité, $\gamma$ et les attributs ayant pour valeur un ensemble vide.
    Les règles de production de la grammaire $G$ sont les suivantes :
    \begin{align*}
        ROOT    & \to GROUP_A ~ COLL_1                                          \\
        COLL_1  & \to REL_X^+              & REL_X   & \to GROUP_A ~ GROUP_B    \\
        GROUP_A & \to ENT_1                & GROUP_B & \to ENT_1 ~ ENT_2        \\
        ENT_1   & \to \langle data \rangle & ENT_2   & \to \langle data \rangle
    \end{align*}

    La règle de production $ROOT \rightarrow GROUP_A ~ COLL_1$ est obtenue par l'application de la méta-règle~\ref{meta:start}, suivie de \ref{meta:root}.
    Ce qui nous donne $ROOT \rightarrow \langle rootList \rangle$.
    Le côté droit de cette règle est un méta-non-terminal.
    Dans la suite, il est spécifié par l'application de la méta-règle \ref{meta:rootList:3} qui engendre $ROOT \rightarrow GROUP_A ~ \langle rootList \rangle$, puis \ref{meta:rootList:6} nous donne $ROOT \rightarrow GROUP_A ~ COLL_1 ~ \langle rootList \rangle$ et \ref{meta:rootList:1} produit $ROOT \rightarrow GROUP_A ~ COLL_1$.
    Voir la figure~\ref{table:struct:meta:ex} avec l'arbre de dérivation.
    Les méta-règles~\ref{meta:ruleList:1}-\ref{meta:ruleList:6} permettent de définir un ensemble de règles.
    En effet, le méta-non-terminal $\langle ruleList \rangle$ a déjà été utilisé dans \ref{meta:start}.
    Chaque règle produite permet de définir un non-terminal de $G$.
    Dans la suite, nous verrons que les règles sémantiques poseront des contraintes, obligeant que chaque non-terminal de notre grammaire cible $G$ soit définit afin de construire une grammaire valide.
    Dans notre exemple, nous avons $GROUP_A \rightarrow  \langle entList \rangle$ et ensuite $GROUP_A \rightarrow ENT_1$.
\end{example}

\begin{landscape}
    \centering
    \vspace*{\fill}
    \begin{adjustbox}{max width=\linewidth,max height=.95\textheight,valign=c}
        \begin{forest}
            for tree={align=center}
            [\huge{$\epsilon$}
            [{\large{$\langle root \rangle$}\\$\color{blue}crL=\{1\}$\\$\color{blue}gL=\{A\}$}
                    [$\lambda$]
                    [$\to$,before computing xy={s/.average={s}{siblings}}]
                    [{\large{$\langle rootList \rangle$}\\$\color{blue}crL=\{1\}$\\$\color{blue}gL=\{A\}$}
                            [$GROUP_A$]
                            [{\large{$\langle rootList \rangle$}\\$\color{blue}crL=\{1\}$}
                                    [$COLL_1$]
                                    [\large{$\langle rootList \rangle$}
                                        [$\epsilon$]
                                    ]
                            ]
                    ]
            ]
            [\textsc{eol},before computing xy={s/.average={s}{siblings}}]
            [{\large{$\langle ruleList \rangle$}\\$\color{blue}crL=\{1\}$\\$\color{blue}rL=\{X\}$\\$\color{blue}gL=\{A, B\}$\\$\color{blue}eL=\{1, 2\}$}
                    [{\large{$\langle collRel \rangle$}\\$\color{blue}name=1$\\$\color{blue}relName=X$}
                            [$COLL_1$]
                            [$\to$,before computing xy={s/.average={s}{siblings}}]
                            [$REL_X^+$]
                    ]
                    [\textsc{eol},before computing xy={s/.average={s}{siblings}}]
                    [{\large{$\langle ruleList \rangle$}\\$\color{blue}rL=\{X\}$\\$\color{blue}gL=\{A, B\}$\\$\color{blue}eL=\{1, 2\}$}
                            [{\large{$\langle relation \rangle$}\\$\color{blue}name=X$\\$\color{blue}gL=\{A, B\}$}
                                    [$REL_X$]
                                    [$\to$]
                                    [$GROUP_A$]
                                    [$GROUP_B$]
                            ]
                            [\textsc{eol},before computing xy={s/.average={s}{siblings}}]
                            [{\large{$\langle ruleList \rangle$}\\$\color{blue}gL=\{A, B\}$\\$\color{blue}eL=\{1, 2\}$}
                                    [{\large{$\langle group \rangle$}\\$\color{blue}name=A$\\$\color{blue}eL=\{1\}$}
                                            [$GROUP_A$]
                                            [$\to$,before computing xy={s/.average={s}{siblings}}]
                                            [{\large{$\langle entList \rangle$}\\$\color{blue}eL=\{1\}$}
                                                    [$ENT_1$]
                                            ]
                                    ]
                                    [\textsc{eol},before computing xy={s/.average={s}{siblings}}]
                                    [{\large{$\langle ruleList \rangle$}\\$\color{blue}gL=\{B\}$\\$\color{blue}eL=\{1, 2\}$}
                                            [{\large{$\langle group \rangle$}\\$\color{blue}name=B$\\$\color{blue}eL=\{1, 2\}$}
                                                    [$GROUP_B$]
                                                    [$\to$,before computing xy={s/.average={s}{siblings}}]
                                                    [{\large{$\langle entList \rangle$}\\$\color{blue}eL=\{1, 2\}$}
                                                            [$ENT_1$]
                                                            [{\large{$\langle entList \rangle$}\\$\color{blue}eL=\{2\}$}
                                                                    [$ENT_2$]
                                                            ]
                                                    ]
                                            ]
                                            [\textsc{eol},before computing xy={s/.average={s}{siblings}}]
                                            [{\large{$\langle ruleList \rangle$}\\$\color{blue}eL=\{1, 2\}$}
                                                    [{\large{$\langle ent \rangle$}\\$\color{blue}name=1$}
                                                            [$ENT_1$]
                                                            [$\to$]
                                                            [$\langle data \rangle$]
                                                    ]
                                                    [\textsc{eol},before computing xy={s/.average={s}{siblings}}]
                                                    [{\large{$\langle ruleList \rangle$}\\$\color{blue}eL=\{2\}$}
                                                            [{\large{$\langle ent \rangle$}\\$\color{blue}name=2$}
                                                                    [$ENT_2$]
                                                                    [$\to$]
                                                                    [$\langle data \rangle$]
                                                            ]
                                                            [\textsc{eol},before computing xy={s/.average={s}{siblings}}]
                                                            [\large{$\langle ruleList \rangle$}
                                                                [$\epsilon$]
                                                            ]
                                                    ]
                                            ]
                                    ]
                            ]
                    ]
            ]
            ]
        \end{forest}
    \end{adjustbox}
    \captionof{figure}{Exemple de dérivation de la méta-grammaire \label{table:struct:meta:ex}}
    \vspace*{\fill}
\end{landscape}

\FloatBarrier
