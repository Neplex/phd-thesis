Cette section a pour objectif de définir les concepts fondamentaux qui seront utilisés dans la suite de ce chapitre et se décompose en quatre parties distinctes.
La première partie se focalise sur la notion de grammaire.
Une grammaire permet de définir un langage en utilisant un ensemble de règles de production qui dérivent les mots du langage sur un alphabet donné.
La seconde partie de cette section s'attarde sur les arbres, qui sont des structures de données hiérarchiques.
% Chaque nœud d'un arbre, à l'exception de la racine, est lié à un nœud parent et zéro ou plusieurs nœuds enfants.
Dans une troisième partie, nous montrerons comment les arbres jouent un rôle important dans la représentation des structures linguistiques.
Cette section se termine par la définition des opérations de transformations et d'éditions appliquées aux arbres.

\subsection{Grammaires}

\begin{definition}[\Glsentrylong{cfg}]
    \label{def:struct:pre:cfg}
    Une \glsreset{cfg}\gls{cfg} est définie par un quadruplet $(N, T, R, S)$ où $N$ est un ensemble fini de symboles non-terminaux ; $T$ est un ensemble fini de symboles terminaux ; $R$ un ensemble fini de règles de production et $S \in N$ est le symbole initial.
    Une règle de production est définie par la forme $X \to \alpha$, avec $X \in N$ et $\alpha$ est une chaîne de symboles terminaux et non-terminaux.

    Nous introduisons la notion de \gls{cfg} condensée qui admet aussi des règles de production de la forme $X \to \alpha^+$ où $\alpha^+$ est une expression régulière qui répéte une ou plusieurs fois la chaîne $\alpha$.
    Cela est équivalent aux règles $X \to \alpha$ et $X \to \alpha~X$.
\end{definition}

\begin{wrapfigure}[14]{r}{.3\textwidth}
    \centering
    \begin{adjustbox}{max width=\linewidth}
        \begin{forest}
            [\dots [U [X [\dots]][Y [\dots]][Z [\dots]]]]
        \end{forest}
    \end{adjustbox}
    \caption{Extrait d'un arbre de dérivation pour la règle de production $U \to X ~ Y ~ Z$}
    \label{fig:struct:ex-deriv}
\end{wrapfigure}

Un arbre de dérivation, également appelé arbre d'analyse ou arbre de syntaxe, décrit comment le symbole de départ d'une grammaire $G$ dérive un mot du langage.
Si un non-terminal $U$ est associé à une règle de production $U \to X ~ Y ~ Z$, l'arbre de dérivation peut contenir un nœud interne étiqueté par $U$, comportant trois enfants, à savoir $X$, $Y$ et $Z$, disposés de gauche à droite comme représenté dans la figure~\ref{fig:struct:ex-deriv}.
Chaque nœud interne représente un symbole non-terminal de la grammaire $G$, tandis que les feuilles symbolisent les symboles terminaux de $G$.
Les liaisons entre les nœuds illustrent la manière dont les symboles sont dérivés les uns des autres.
La construction de l'arbre de dérivation s'effectue de manière récursive en suivant les règles de production de la grammaire. Elle débute avec un nœud racine correspondant au symbole initial de la grammaire, et à chaque niveau de l'arbre, les nœuds sont remplacés par des symboles conformément aux règles de production.

\begin{definition}[Arbre de dérivation]
    Étant donné une \gls{cfg} $G = (N, T, R, S)$, un arbre de dérivation de la grammaire est un arbre qui respecte les propriétés suivantes :
    \begin{enumerate}
        \item Le nœud racine de l'arbre est étiqueté par le symbole de départ $S$, noté $l(\epsilon) = S$ ;
        \item Chaque feuille de l'arbre $f$ est étiquetée par un symbole terminal, c'est-à-dire que $l(f) \in T$ ;
        \item Chaque nœud interne $n$ de l'arbre est étiqueté par un symbole non-terminal, c'est-à-dire que $l(n) \in N$ ;
        \item Si $U$ est un non-terminal utilisé comme étiquette d'un nœud interne $n$ et $X_1, \dots, X_n$ sont les étiquettes des enfants de $n$ de gauche à droite, alors il existe une règle de production $U \to X_1 ~ \dots ~ X_n$ dans $R$.
        Les étiquettes $X_1, \dots, X_n$ représentent une séquence de symboles terminaux et non-terminaux.
    \end{enumerate}
\end{definition}

\begin{example}
    \label{ex:struct:cfg}
    Soit la grammaire $G = (\{P\}, \{0,1\}, R, P)$ où $R$ est l'ensemble de règles de productions ci-dessous et qui produit des nombres binaires.
    La figure~\ref{fig:struct:parsetree} montre un arbre de dérivation pour $0011$.
    \begin{align*}
        P & \to 0 & P & \to 1 & P & \to P~0 & P & \to P~1
    \end{align*}
\end{example}

\begin{figure}[htb]
    \begin{subfigure}[t]{.45\textwidth}
        \centering
        \begin{adjustbox}{max width=\linewidth}
            \begin{forest}
                where n children=0{tier=word}{}
                [P, for tree={calign=last, s sep=2em}
                    [P
                            [P
                                    [P
                                            [0]
                                    ]
                                    [0]
                            ]
                            [1]
                    ]
                    [1]
                ]
            \end{forest}
        \end{adjustbox}
        \caption{Arbre de dérivation pour la grammaire $G$ définie dans l'exemple~\ref{ex:struct:cfg}}
        \label{fig:struct:parsetree}
    \end{subfigure}
    \hfill
    \begin{subfigure}[t]{.45\textwidth}
        \centering
        \begin{adjustbox}{max width=\linewidth}
            \begin{forest}
                where n children=0{tier=word}{}
                [{P ($val=3$)}, for tree={calign=last, s sep=2em}
                    [{P ($val=1$)}
                            [{P ($val=0$)}
                                    [{P ($val=0$)}
                                            [0]
                                    ]
                                    [0]
                            ]
                            [1]
                    ]
                    [1]
                ]
            \end{forest}
        \end{adjustbox}
        \caption{Arbre de dérivation pour la grammaire $G'$ définie dans l'exemple~\ref{ex:struct:cfg-attr} avec les valeurs de l'attribut synthétisé $val$}
        \label{fig:struct:parsetree-attr}
    \end{subfigure}
    \caption{Exemple de dérivation d'une grammaire pour le nombre $0011$}
\end{figure}

Le principe des grammaires attribuées peut se résumer de la manière suivante : nous considérons une grammaire hors contexte.
À chacun de ses non-terminaux, on attache deux ensembles de symboles :
\begin{enumerate*}[label=(\arabic*)]
    \item les attributs synthétisés, qui transmettent l'information depuis les feuilles d'un arbre de dérivation jusqu'à la racine et
    \item les attributs hérités, qui transportent l'information en sens inverse.
\end{enumerate*}
À chaque production, nous associons un ensemble de règles sémantiques qui spécifient comment calculer les attributs de sortie ; c'est-à-dire les attributs synthétisés du non-terminal membre gauche de la production et les attributs hérités des non-terminaux du membre droit de la production, en fonction des attributs d'entrée de la production (les autres attributs).
La définition suivante formalise cette notion.

\begin{definition}[Grammaire attribuée]
    \label{def:struct:G-attr}
    Une grammaire attribuée \cite{knuthSemanticsContextfreeLanguages1968}, ou grammaire avec attributs, est une \gls{cfg} $G = (N, T, R, S)$ à laquelle on ajoute, pour chaque règle de production $r \in R$ un ensemble de règles sémantiques noté $\Phi_r$ et pour chaque symbole $X \in (N \cup T)$ on associe un ensemble fini d'attributs $A(X)$ constitué de deux sous-ensembles disjoints d'attributs :
    \begin{itemize}
        \item $A_\uparrow(X)$ pour les attributs synthétisés où $\forall X \in T ~ A_\uparrow(X) = \emptyset$
        \item $A_\downarrow(X)$ pour les attributs hérités où, pour le symbole initial $S$, $A_\downarrow(S) = \emptyset$
    \end{itemize}
    Chaque attribut $a \in A(X)$ possède un ensemble (potentiellement infini) de valeurs possibles $V_a$ duquel une valeur est sélectionnée pour chaque apparition de $X$ dans l'arbre de dérivation.
    $r$ est une règle de production de la forme $X_0 \to X_1, \dots, X_n$ où $n \ge 1$, $X_0 \in N$ et $X_i \in (N \cup T)$ pour $1 \le i \le n$.
    Une règle sémantique $\varphi_{a_j} \in \Phi_r$ est une fonction $\varphi_{a_j} : \bigtimes_{i = 1}^n V_{a_i} \to V_{a_j}$ où $a_j \in A_\uparrow(X_j)$ si $j = 0$, $a_j \in A_\downarrow(X_j)$ si $j > 0$ et $a_i \in A(X_i)$.
    En d'autres termes, une règle sémantique sur $r: X_0 \to X_1, \dots, X_n$ est de la forme $a_0 \gets \varphi(a_1, \dots, a_k)$ et calcule la valeur d'un attribut de $X_j$ à partir des attributs des symboles $X_0, \dots, X_n$.
    Si $a_0$ est un attribut de $X_0$, c'est un attribut synthétisé.
    Si $a_0$ est un attribut de $X_j$ ($1 \leq j \leq n$), c'est un attribut hérité.
\end{definition}

\begin{example}
    \label{ex:struct:cfg-attr}
    Soit la grammaire attribuée $G'$ construite à partir de $G$ (exemple~\ref{ex:struct:cfg}).
    La grammaire $G'$ associe la valeur décimale à un nombre binaire.
    Cela est formalisé par les règles de production suivantes en associant un attribut synthétisé $val$ au non terminal $P$ et en fournissant des règles pour calculer la valeur de l'attribut $val$ par rapport à la valeur de l'attribut $val'$ précédemment calculé et associé au côté droit de la règle de production.
    Dans les règles de production, les attributs sont indiqués en indice et les règles sémantiques sont présentées, entre crochet, à droite de la règle de production.
    La figure~\ref{fig:struct:parsetree-attr} montre, pour chaque niveau, la valeur calculé de l'attribut $val$ pour $0011$.
    \begin{minipage}[b]{.45\textwidth}
        \begin{align*}
            P_{val} & \to 0 & [val \gets 0] \\
            P_{val} & \to 1 & [val \gets 1]
        \end{align*}
    \end{minipage}
    \quad
    \begin{minipage}[b]{.45\textwidth}
        \begin{align*}
            P_{val} & \to P_{val'}~0 & [val \gets 2 * val'] \\
            P_{val} & \to P_{val'}~1 & [val \gets 2 * val' + 1]
        \end{align*}
    \end{minipage}
\end{example}

Les grammaires \emph{S-attribuées} sont des grammaires attribuées qui contiennent uniquement des attributs synthétisés.
Elles ont l'avantage d'être plus simple et plus facilement vérifiable en utilisant uniquement une propagation du bas vers le haut.
Elles sont souvent plus adaptées pour des tâches de compilation ou d'analyse sémantique bien que leur pouvoir d'expression soit réduit par rapport aux autres grammaires attribuées qui acceptent les attributs hérités.

\begin{definition}[Méta-grammaire]
    Une méta-grammaire $\mathbb{G} = (N, T, R, S)$ est une grammaire S-attribuée qui définit la syntaxe des règles de production d'un ensemble de \gls{cfg} condensées.
    Autrement dit, chaque mot du langage reconnu par $\mathbb{G}$ sera la liste des règles de productions d'une \gls{cfg} condensées.
    $N$ est l'ensemble fini de méta-non-terminaux, $T$ l'ensemble fini de méta-terminaux, $R$ l'ensemble fini des règles de productions et $S \in N$ le symbole initial.
    On définit l'attribut synthétisé spécial $\gamma$ qui permet de vérifier à chaque niveau de l'arbre de dérivation que la dérivation produit bien une \gls{cfg} condensée valide.
    Si $S_\gamma = \top$, alors la dérivation de $\mathbb{G}$ est considérée valide, inversement, si $S_\gamma = \bot$ la dérivation de $\mathbb{G}$ est considérée invalide.
    On distingue alors deux types de règles sémantiques dans une règle de production $r \in R$ :
    \begin{itemize}
        \item Les règles de la forme $a \gets \alpha$ où $a$ est l'attribut qui est évalué et $\alpha$ est une formule sur les attributs dans $r$ ;
        \item Les règles de la forme $\gamma \gets \beta$ où $\beta$ est une formule logique sur les attributs dans $r$.
    \end{itemize}
    Pour des raisons de lisibilité, on omettra la partie $\gamma \gets$ et les règles de la forme $a \gets a$ dans la suite de ce chapitre.
\end{definition}

\subsection{Arbres}

\begin{definition}[Arbre ordonné]
    \label{def:struct:tree}
    Formellement, un arbre $T$ est une paire $T = (D, l)$ où $D$ est le domaine de l'arbre et $l$ est une fonction de d'étiquetage telle que $l : D \to \Sigma \cup \{\lambda\}$ où $\Sigma$ est l'ensemble des labels de l'arbre et $\lambda$ un symbole spécial.
    Le domaine $D$ d'un arbre est un sous ensemble de $(\mathbb{N})^*$ (c.-à-d. un ensemble de séquence d'entier de la forme $x.y.z$) qui respecte les propriétés suivantes :
    \begin{enumerate}
        \item $D$ est clos sur les préfixes, c'est-à-dire que pour $u, u' \in (\mathbb{N})^*$ si $u$ est un préfixe de $u'$ et $u' \in D$, alors $u \in D$, et
        \item Pour tout $u \in (\mathbb{N})*$ et $j \in \mathbb{N}$ si $u.j \in D$ alors pour tous $i \in \mathbb{N}$ tel que $0 \leq i < j$ on a $u.i \in D$.
    \end{enumerate}
    Chaque élément de $D$ est appelé \emph{position}.
    Pour un nœud $n$ à la position $p$, $|p|$ définit la longueur de la séquence, aussi appelée \emph{profondeur} de $n$.
    La racine d'un arbre est à la position $\epsilon$ et est représentée par le symbole spécial $\lambda$ tel que $l(\epsilon) = \lambda$.
    Un arbre vide est donc défini par  $T = (\{\epsilon\}, \langle \epsilon \mapsto \lambda \rangle)$.
    On note $v \prec u$ si $\exists i \in \mathbb{N}$ tel que $v = u.i$ et on appelle le nœud à la position $v$ \emph{parent} du nœud à la position $u$ inversement $u$ est \emph{enfant} de $v$.  
    %Étant donné un nœud à la position $u.i$ on appelle \emph{parent} le nœud à la position $u$ et \emph{enfants} les nœuds aux positions $u.i.j ~ \forall j \in \mathbb{N}$ tel que $u.i.j \in D$.
    On note $v \prec^* u$ si $\exists v'$ tel que $u = v.v'$, on dit que $v$ est \emph{préfixe} de $u$ et on appelle un nœud à la position $u$ \emph{descendant} du nœud à la position $v$.
    %Un nœud $n$ à la position $u$ est un \emph{descendant} d'un nœud $m$ la position $v$ si et seulement si $v$ est un préfixe direct de $u$ (noté $v \preceq u$) ou indirect (noté $v \preceq^* u$), inversement $m$ est un \emph{ancêtre} de $n$.
    On appelle \emph{feuille} un nœud à une position $u$ tel que $u.0 \notin D$, en d'autres termes, c'est un nœud qui n'a pas d'enfants.
    % $\nexists u.i \in D ~ \forall i \in \mathbb{N}$
\end{definition}

\begin{definition}[Sous-arbre]
    Étant donné un arbre $T = (D, l)$, un sous-arbre de $T$ à la position $u \in D$ est noté $T|_u = (D', l')$ et respecte les propriétés suivantes :
    \begin{enumerate}
        \item $D' \subseteq D$ tel que $\forall v \in D' ~ u \preceq^* v$ et
        \item $l' = \langle v \mapsto l(v) \mid v \in D' \rangle $.
    \end{enumerate}
    Soit un sous-arbre $t = T|_u$, on note $t' = P_i^t$ le $i$-éme sous-arbre parent de $t$ tel que $t' = T|_v$,  $u = vw$ et $|w| = i$.
\end{definition}

\begin{figure}[htb]
    \centering
    \begin{adjustbox}{max width=\linewidth}
        \begin{forest}
            for tree={s sep=5em}
            [{parent ($\epsilon$)}
                    [{child1 ($0$)}]
                    [{child2 ($1$)}
                            [{grandchild1 ($1.0$)}]
                            [{grandchild2 ($1.1$)}]
                    ]
            ]
        \end{forest}
    \end{adjustbox}
    \caption{Exemple d'un arbre ordonné}
    \label{fig:struct:tree-ex}
\end{figure}

\begin{example}
    La figure~\ref{fig:struct:tree-ex} illustre un arbre $T = (D, l)$ ordonné où $D=\{\epsilon, 0, 1, 1.0, 1.1\}$.
    Dans la figure, les positions sont représentées entre parenthèses.
    Nous avons par exemple $l(0) = child1$.
    $T|_{1} = (D', l')$ est un sous-arbre de $T$.
    Remarquez que $T|_{1}$ n'est pas un arbre, car $D'= \{1, 1.0, 1.1\}$ ne respecte pas les conditions requises (définition~\ref{def:struct:tree}).
    Ici, $P_1^{T|_{1}}$ coïncide avec $T$.
\end{example}

\subsection{Textes et arbres}

Les textes possèdent une structure complexe, cependant il est possible de les représenter comme des arbres.
Il existe deux structures très utilisées dans la littérature :

\begin{description}
    \item[Arbre syntaxique]
          (Figure~\ref{fig:struct:ex-tree:syx})
          Aussi appelé arbre en constituant, est l'arbre de dérivation de la grammaire du texte.
          En linguistique, il représente la structure syntaxique d'une phrase.
          Il met en évidence les relations hiérarchiques entre les mots ou groupe de mots et les catégories auxquelles ils appartiennent où \gls{pos} en anglais : nom (NN), verbe (VBD), adjectif (ADJ), déterminent (DT), etc.
          Les feuilles de l'arbre sont les unités lexicales (les mots) et les nœuds intermédiaires représente les structures abstraites comme une phrase verbale (VP) ou un groupe nominal (NP).

    \item[Arbre de dépendance]
          (Figure~\ref{fig:struct:ex-tree:dep})
          Cette représentation met en évidence des liens syntaxique direct entre les unités lexicales représentant une forme de sémantique.
          Dans l'arbre de dépendance, chaque unité lexicale est un nœud de l'arbre et les arêtes sont labellisés par des relations syntaxiques comme \emph{sujet} ou \emph{objet} (entre un nom et un verbe), ou encore \emph{modifier} entre un adjectif et un nom.
          Cet arbre est souvent utilisé dans les taches d'extraction d'information pour récupérer une sémantique entre les mots notamment, car des liens peuvent exister entre des mots très éloignés ce qui est moins évidents dans un arbre syntaxique.
\end{description}

Ces arbres sont obtenus à l'aide d'apprentissages automatiques entrainés à partir d'un corpus appelé banques d'arbres.
L'ensemble des labels utilisés dépend des corpus utilisés et de la langue.
Pour les arbres de dépendance, \gls{ud}\footnote{\url{https://universaldependencies.org/}} est un guide d'annotation très répandu.

\begin{figure}[htb]
    \centering
    \begin{subfigure}{\textwidth}
        \centering
        \begin{adjustbox}{max width=\linewidth}
            \begin{forest}
                where n children=0{tier=word}{}
                [S
                    [NP
                            [DT [la]]
                            [NP,before computing xy={s/.average={s}{siblings}} [fréquence]]
                            [NP [cardiaque]]
                    ]
                    [VP
                            [VBD [était]]
                            [PP
                                    [IN [à]]
                                    [NP
                                            [CD [100]]
                                            [NN [b/min]]
                                    ]
                            ]
                    ]
                ]
            \end{forest}
        \end{adjustbox}
        \caption{Arbre syntaxique}
        \label{fig:struct:ex-tree:syx}
    \end{subfigure}
    \begin{subfigure}{\textwidth}
        \centering
        \begin{adjustbox}{max width=\linewidth}
            \begin{dependency}
                \begin{deptext}[column sep=2em]
                    la \& fréquence \& cardiaque \& était \& à \& 100 \& b/min \\
                \end{deptext}
                \deproot{7}{root}
                \depedge{2}{1}{det}
                \depedge{2}{3}{amd}
                \depedge{7}{4}{cop}
                \depedge{7}{6}{nummod}
                \depedge{6}{5}{case}
                \depedge{7}{2}{nsubj}
            \end{dependency}
        \end{adjustbox}
        \caption{Arbre de dépendance}
        \label{fig:struct:ex-tree:dep}
    \end{subfigure}
    \caption[Comparaison entre l'arbre syntaxique et l'arbre de dépendance]{Comparaison entre l'arbre syntaxique (obtenu avec \gls{corenlp}) et l'arbre de dépendance (obtenu avec \gls{spacy}) pour la phrase \enquote{\textelp{} la fréquence cardiaque était à 100 b/min} issue du corpus CAS \cite{grabarCASFrenchCorpus2018}}
    \label{fig:struct:ex-tree}
\end{figure}

\subsection{Réécriture d'arbres}
\label{sec:struct:pre:tree-rewritting}

\begin{definition}[Haie]
    Une haie est définie comme une séquence d'arbres, qui peut éventuellement être vide, représentée sous la forme $h = [t_0, \dots , t_n]$ où $|h|$ représente le nombre d'arbres contenus dans cette haie (ici $|h| = n + 1$).
    Une substitution, notée $\sigma$, est une application bijective d'un ensemble de variables $V$ vers un ensemble de haies et d'un ensemble de labels vers un ensemble de sous-arbres, étendue de manière homomorphique aux arbres.
\end{definition}

\begin{definition}[Règle de réécriture]
    Une règle de réécriture sur un arbre décrit comment un arbre $t$ peut être réécrit en $t'$ pour une position $u$ donnée.
    Elle est constituée d'une partie gauche, appelée \gls{lhs}, représentant un motif et d'une partie droite, appelée \gls{rhs}, représentant la transformation et s'écrit $\gls*{lhs} \to \gls*{rhs}$.
    Le motif est un sous-arbre construit sur l'ensemble $\Sigma \cup V \cup \{\lambda\}$ où $\Sigma$ est l'ensemble des labels de $t$, $V$ est un ensemble de variables et $\lambda$ est le symbole racine.
    Il existe un morphisme entre les variables de \gls{lhs} et \gls{rhs} permettant la transformation.

    La transformation s'applique sur une substitution $\sigma$ du motif de la partie gauche par un sous-arbre de $t$ à une position donnée $u$.
    En d'autres termes, il existe une correspondance entre les éléments du motif (\gls{lhs}) et un sous-arbre de $t$.
    On note l'application de la règle sur l'arbre $t$ donnant $t'$ avec la substitution $\sigma$ à la position $u$ : $t \mapsto_{[u, \gls*{lhs} \to \gls*{rhs}, \sigma]} t'$ tel que $t|_u = \sigma(\gls*{lhs})$ et $t'|_u = \sigma(\gls*{rhs})$.
    Une règle de réécriture peut contenir un ensemble de conditions d'application en plus du motif.
    Ces conditions spécifient les circonstances dans lesquelles la règle de réécriture peut être appliquée.
    Les conditions peuvent inclure des contraintes sur les attributs des nœuds ou des arêtes, des contraintes topologiques, etc.
\end{definition}

\begin{example}[Règle de réécriture]
    \label{ex:struct:rewriteRule}
    Étant donné la règle de réécriture $rule(u.i)$ et la contrainte $|\sigma(x)| = i$ donnée figure~\ref{fig:sch:pre:rewritting:ex-rule}.
    $U$ représente le nœud à la position $u$ de l'arbre cible, $x \in V$ est une variable et $\{X, Y, A, B, C, D\} \subseteq \Sigma$ représentent des étiquettes (des non-terminaux) de l'arbre.
    La règle est applicable sur un arbre $T$ s'il existe un sous-arbre $t = T|_u$ avec une substitution $\sigma$ tel que $\sigma(x) = [t|_0, \dots, t|_{i-1}]$ et $l(u.i) = A$.
    %L'arbre de destination \gls{rhs} à une correspondance avec le modèle \gls{lhs}, ici intuitivement claire : $U \mapsto U$ et $x \mapsto x$.
    En cas d'application de la règle, le sous-arbre $A$ est supprimé de $T$.

    \begin{figure}[htb]
        \centering
        \begin{subfigure}[c]{0.4\textwidth}
            \centering
            \begin{adjustbox}{valign=c, max width=\textwidth}
                \begin{forest}
                    for tree={s sep=5em}
                    [$U$ [$x$] [$A$]]
                \end{forest}
            \end{adjustbox}
            \caption*{\glsfirst*{lhs}}
        \end{subfigure}
        \hfill
        \begin{subfigure}[c]{0.1\textwidth}
            \centering
            \Large{$\longrightarrow$}
        \end{subfigure}
        \hfill
        \begin{subfigure}[c]{0.4\textwidth}
            \centering
            \begin{adjustbox}{valign=c, max width=\textwidth}
                \begin{forest}
                    for tree={s sep=20mm}
                    [$U$ [$x$]]
                \end{forest}
            \end{adjustbox}
            \caption*{\glsfirst*{rhs}}
        \end{subfigure}
        \caption{Exemple d'une règle de réécriture}
        \label{fig:sch:pre:rewritting:ex-rule}
    \end{figure}

    Si on considère l'arbre $T$ donné figure~\ref{fig:sch:pre:rewritting:ex:origin}, avec l'application de la règle $rule(0.2)$, où $u=0$ et $i=2$, on obtient $\sigma(x) = [C, B]$.
    Le résultat de l'application et l'arbre figure~\ref{fig:sch:pre:rewritting:ex:target}.

    \begin{figure}[htb]
        \centering
        \begin{subfigure}[c]{0.4\textwidth}
            \centering
            \begin{adjustbox}{valign=c, max width=\textwidth}
                \begin{forest}
                    for tree={s sep=5em}
                    [$\lambda$ [$X$ [$C$] [$B$] [$A$]] [$Y$ [$D$]]]
                \end{forest}
            \end{adjustbox}
            \caption{Arbre $T$}
            \label{fig:sch:pre:rewritting:ex:origin}
        \end{subfigure}
        \hfill
        \begin{subfigure}[c]{0.1\textwidth}
            \centering
            \huge{$\Longrightarrow$}
        \end{subfigure}
        \hfill
        \begin{subfigure}[c]{0.4\textwidth}
            \centering
            \begin{adjustbox}{valign=c, max width=\textwidth}
                \begin{forest}
                    for tree={s sep=20mm}
                    [$\lambda$ [$X$ [$C$] [$B$]] [$Y$ [$D$]]]
                \end{forest}
            \end{adjustbox}
            \caption{Arbre réécrit}
            \label{fig:sch:pre:rewritting:ex:target}
        \end{subfigure}
        \caption{Exemple d'application d'une règle de réécriture}
        \label{fig:sch:pre:rewritting:ex}
    \end{figure}
\end{example}

On introduit les opérations de réécriture d'arbre élémentaires qui seront utilisé dans ce chapitre.
La première règle, $\textsf{ins\_elem}$ (figure~\ref{fig:sch:op:insElem}), permet l'insertion d'un nœud d'étiquette $A$ dans l'arbre $T$ à la position $u.i$.
Comme discuté dans l'exemple~\ref{ex:struct:rewriteRule}, la condition $|\sigma(x)| = i$ assure qu'il y a $i$ éléments dans la haie $\sigma(x)$ impliquant que le nœud $A$ se trouve bien à la position $u.i$.
Il n'y a aucunes contraintes sur $y$ qui a pour rôle de déplacer les éléments à droite de $A$.

Les figures~\ref{fig:sch:op:delElemRoot} et \ref{fig:sch:op:delElem} présente la règle de réécriture $\textsf{del\_elem}$ utilisée pour supprimer un sous-arbre dans un arbre $T$ à une position donnée $u.i$.
La règle est récursive : si l'on supprime l'unique enfant $a$ d'un nœud $b$, c'est le nœud $b$ qui est supprimé (figure~\ref{fig:sch:op:delElem}).
Pour cela, on utilise la condition $|\sigma(x)| + |\sigma(y)| \neq 0$ pour vérifier si la substitution est non-vide.
Si l'on se retrouve à devoir supprimer l'arbre au complet (figure~\ref{fig:sch:op:delElemRoot}), on le remplace par un arbre vide contenant uniquement le symbole racine $\lambda$.

\begin{figure}[htb]
    \centering
    \begin{subfigure}[c]{0.4\textwidth}
        \centering
        \begin{adjustbox}{valign=c, max width=\textwidth}
            \begin{forest}
                for tree={s sep=4em}
                [U
                    [$x$]
                    [$y$]
                ]
            \end{forest}
        \end{adjustbox}
        \caption*{\glsname*{lhs}}
    \end{subfigure}
    \begin{subfigure}[c]{0.1\textwidth}
        \centering
        \Large{$\longrightarrow$}
    \end{subfigure}
    \begin{subfigure}[c]{0.4\textwidth}
        \centering
        \begin{adjustbox}{valign=c, max width=\textwidth}
            \begin{forest}
                for tree={s sep=2em}
                [U
                    [$x$]
                    [A]
                    [$y$]
                ]
            \end{forest}
        \end{adjustbox}
        \caption*{\glsname*{rhs}}
    \end{subfigure}
    \caption[Règle $\textsf{ins\_elem}(T, A, u.i)$]{$\textsf{ins\_elem}(T, A, u.i)$ où $A \in \Sigma, |\sigma(x)| = i$}
    \label{fig:sch:op:insElem}
\end{figure}

\begin{figure}[htb]
    \centering
    \begin{minipage}[c]{.35\textwidth}
        \centering
        \begin{subfigure}[c]{0.4\textwidth}
            \centering
            \begin{adjustbox}{valign=c, max width=\textwidth}
                \begin{forest}
                    [T]
                \end{forest}
            \end{adjustbox}
            \caption*{\glsname*{lhs}}
        \end{subfigure}
        \begin{subfigure}[c]{0.1\textwidth}
            \centering
            \Large{$\longrightarrow$}
        \end{subfigure}
        \begin{subfigure}[c]{0.4\textwidth}
            \centering
            \begin{adjustbox}{valign=c, max width=\textwidth}
                \begin{forest}
                    [$\lambda$]
                \end{forest}
            \end{adjustbox}
            \caption*{\glsname*{rhs}}
        \end{subfigure}
        \caption[Règle $\textsf{del\_elem}(T, \epsilon)$]{$\textsf{del\_elem}(T, \epsilon)$}
        \label{fig:sch:op:delElemRoot}
    \end{minipage}%
    \hfill
    \begin{minipage}[c]{.55\textwidth}
        \centering
        \begin{subfigure}[c]{0.4\textwidth}
            \centering
            \begin{adjustbox}{valign=c, max width=\textwidth}
                \begin{forest}
                    for tree={s sep=2em}
                    [U
                        [$x$]
                        [A]
                        [$y$]
                    ]
                \end{forest}
            \end{adjustbox}
            \caption*{\glsname*{lhs}}
        \end{subfigure}
        \begin{subfigure}[c]{0.1\textwidth}
            \centering
            \Large{$\longrightarrow$}
        \end{subfigure}
        \begin{subfigure}[c]{0.4\textwidth}
            \centering
            \begin{adjustbox}{valign=c, max width=\textwidth}
                \begin{forest}
                    for tree={s sep=4em}
                    [U
                        [$x$]
                        [$y$]
                    ]
                \end{forest}
            \end{adjustbox}
            \caption*{\glsname*{rhs}}
        \end{subfigure}
        \caption[Règle $\textsf{del\_elem}(T, u.i)$]{$\textsf{del\_elem}(T, u.i)$ où $|\sigma(x)| = i$ \\si $|\sigma(x)| + |\sigma(y)| \neq 0$, sinon $\textsf{del\_elem}(X, u)$}
        \label{fig:sch:op:delElem}
    \end{minipage}%
\end{figure}

\FloatBarrier
