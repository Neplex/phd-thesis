Cette section à pour objectif de définir les concepts fondamentaux qui seront utilisés dans la suite de ce chapitre et se décompose en deux parties distinctes.
La première partie se focalise sur la définition des grammaires formelles et d'outils indispensables.
En théorie du langage, une grammaire représente un langage par un ensemble de règles de production qui produit par dérivation les mots du langage.
La seconde partie de cette section s'attarde sur les arbres, qui sont des structures de données hiérarchiques composées de nœuds.
Chaque nœud, à l'exception de la racine, est lié à un nœud parent et peut posséder zéro ou plusieurs nœuds enfants.
Les arbres jouent un rôle important dans la représentation graphique des structures linguistiques.
De plus, cette seconde partie s'intéresse aussi à la définition des opérations de transformations et d'éditions appliquées aux arbres.

\subsection{Grammaires}

\begin{definition}[\Glsentrylong{cfg}]
    Une \glsfirst{cfg} est un quadruplet $(N, T, R, S)$ où $N$ est un ensemble fini de symboles non-terminaux ; $T$ est un ensemble fini de symboles terminaux ; $R$ un ensemble fini de règles de production et $S \in N$ est le symbole initial.
    Une règle de production est définie par la forme $n_i \to \alpha$, avec $n_i \in N$ un symbole non-terminal et $\alpha$ est une chaîne de symboles terminaux et non-terminaux.
\end{definition}

\begin{definition}[Grammaire régulière]
    Une grammaire $G = (N, T, R, S)$ est dite régulière si $\alpha$ est une expression régulière sur l'ensemble $N \cup T$ de la forme $\alpha ::= \epsilon \mid nt_i \mid t_i \mid \alpha|\alpha \mid \alpha.\alpha \mid \alpha^+ \mid \alpha^* \mid \alpha^?$ où $nt_i \in N$ et $t_i \in T$.
\end{definition}

\begin{definition}[Arbre de dérivation]
    Un arbre de dérivation, arbre d'analyse ou arbre de syntaxe décrit une dérivation d'une grammaire $G$.
    Chaque nœud intermédiaire représente un symbole non-terminal de la grammaire G, tandis que les feuilles représentent les symboles terminaux de $G$.
    Les liaisons entre les nœuds illustrent comment les symboles sont dérivés les uns des autres.

    L'arbre de dérivation est construit récursivement en suivant les règles de production de la grammaire.
    Il commence avec un nœud racine qui correspond au symbole initial de la grammaire, puis à chaque niveau de l'arbre, les nœuds sont remplacés par des symboles conformément aux règles de production.
\end{definition}

\begin{definition}[Grammaire attribuée]
    Une grammaire attribuée \cite{knuthSemanticsContextfreeLanguages1968}, ou grammaire avec attributs, est une grammaire formelle $G = (N, T, R, S)$ où $P$ est un ensemble de production de la forme $n_i^A \to \alpha \{\phi\}$, avec $n_i \in N$, $A$ un ensemble d'attributs et $\Phi$ un ensemble de fonction sémantiques sur les attributs.
    Les attributs sont des valeurs liées aux symboles et sont généralement calculés à l'aide de fonctions sémantiques qui décrivent comment les déduire à partir des valeurs d'attributs des symboles contenus dans la production.
    Les fonctions sémantiques sont utilisé pour valider une dérivation de la grammaire.
    On distingue deux catégories d'attributs :
    \begin{description}
        \item[Attributs synthétisés] Les attributs synthétisés sont calculé à partir des attributs enfants dans l'arbre de dérivation.
              La propagation est donc du bas vers le haut (bottom-up).
        \item[Attributs hérités] Les attributs hérités sont calculé à partir des attributs parents ou frères dans l'arbre de dérivation.
              La propagation peut être effectuée de différente manière.
    \end{description}

    Les grammaires \emph{S-attribuées} sont des grammaires attribuées qui contiennent uniquement des attributs synthétisés.
    Elles ont l'avantage d'êtres plus simple est plus facilement vérifiable en utilisant uniquement une propagation du bas vers le haut.
    Elles sont souvent plus adaptées pour des tâches de compilation ou d'analyse sémantique bien que leur pouvoir d'expression soit reduit.
\end{definition}

\begin{definition}[Méta-grammaire]
    Une méta-grammaire est une grammaire formelle qui définit la syntaxe des règles de production d'un ensemble de grammaires.
\end{definition}

\subsection{Arbres}

\begin{definition}[Arbre ordonné]
    Formellement, un arbre $T$ est une paire $T = (D, l)$ où $D$ est le domaine de l'arbre (l'ensemble des positions) et $l$ est une fonction de labellisation
    tel que $l : D \to \Sigma \cup \{\lambda\}$.
    Le domaine $D$ d'un arbre est un sous ensemble de $\mathbb{N}^*$ qui respecte les propriétés suivantes :
    \begin{enumerate}
        \item $D$ est clos sur les préfixes, c'est-à-dire que si $u, u' \in \mathbb{N}^*$ et $u$ est un préfixe de $u'$ (noté $u \preceq u'$) et $u' \in D$, alors $u \in D$, et
        \item $\forall u, j \in \mathbb{N}$ $u.j \in D \implies \forall i \in \mathbb{N}$ tel que $0 \leq i \leq j$ $ u.i \in D$.
    \end{enumerate}
    Chaque élément de $D$ est appelé \emph{position}.
    Pour un nœud $n$ à la position $p$, $|p|$ définit la longueur de la séquence, aussi appelée profondeur de $n$.
    La racine d'un arbre est à la position $\epsilon$ et est représenté par le symbole spécial $\lambda$ tel que $t(\epsilon) = \lambda$.
    Un arbre vide est donc défini par  $T = (\epsilon, \langle \epsilon \to \lambda \rangle)$.
    Étant donné un nœud à la position $u.i$ le parent noté est le nœud à la position $u$ et les enfants sont les nœuds aux positions $u.i.j ~ \forall j \in \mathbb{N}$ tel que $u.i.j \in D$.
    Un nœud $n$ à la position $p$ est un descendant d'un nœud $m$ la position $q$ si et seulement si $q$ est un préfixe de $p$ (noté $q \preceq p$).
    Une feuille est un nœud en bas de l'arbre à une position $p$ tel que $\nexists p.i \in D ~ \forall i \in \mathbb{N}$, en d'autres termes, c'est un nœud qui n'a pas d'enfants.
\end{definition}

\begin{definition}[Sous arbre]
    Étant donné un arbre $T = (D, l)$, un sous arbre de $T$ à la position $u \in D$ est noté $T|_u = (D', l')$ et respecte les propriétés suivantes :
    \begin{enumerate}
        \item $D' = \{v \mid u.v \in D\}$ et
        \item $l' = \{(v, l(u.v)) \mid v \in D'\}$.
    \end{enumerate}
    Soit un sous arbre $T$, on note $T' = P_i^T$ le $i$-éme sous arbre parent tel que $\exists v ~ T'|_v = T, |v| = i$.
\end{definition}

\begin{definition}[Haie]
    Une haie est définie comme une séquence d'arbres, qui peut éventuellement être vide, représentée sous la forme $h = [t_0, \dots , t_n]$.
    $|h|$ représente le nombre d'arbres contenus dans cette haie, ici $|h| = n + 1$.
    Une substitution, notée $\sigma$, est une application bijective d'un ensemble de variables $V$ vers un ensemble de haies et d'un ensemble de labels vers un ensemble de sous arbres, étendue de manière homomorphique aux arbres.
\end{definition}

\begin{definition}[Règle de réécriture]
    Une règle de réécriture sur un arbre décrit comment un arbre $t$ peut être réécrit en $t'$.
    Elle est constituée d'une partie gauche, appelée \gls{lhs}, représentant un modèle et d'une partie droite, appelée \gls{rhs}, représentant la transformation et s'écrit $\gls*{lhs} \to \gls*{rhs}$.
    Le modèle est un sous arbre construit sur l'ensemble $\Sigma \cup V \cup \{\lambda\}$ où $\Sigma$ est l'ensemble des labels de $t$, $V$ est un ensemble de variables et $\lambda$ est le symbole racine.
    Il existe un morphisme entre les variables de \gls{lhs} et \gls{rhs} permettant la transformation.

    La transformation s'applique sur une substitution $\sigma$ du modèle de la partie gauche par un sous-arbre de $t$ à une position donnée $u$.
    En d'autres termes, il existe une correspondance entre les éléments du modèle (\gls{lhs}) et un sous-arbre de $t$.
    On note l'application de la règle sur l'arbre $t$ donnant $t'$ avec la substitution $\sigma$ à la position $u$ : $t \mapsto_{[u, \gls*{lhs} \to \gls*{rhs}, \sigma]} t'$ tel que $t|_u = \sigma(\gls*{lhs})$ et $t'|_u = \sigma(\gls*{rhs})$.
    Une règle de réécriture peut contenir un ensemble de conditions d'application en plus du modèle.
    Ces conditions spécifient les circonstances dans lesquelles la règle de réécriture peut être appliquée.
    Les conditions peuvent inclure des contraintes sur les attributs des nœuds ou des arêtes, des contraintes topologiques, etc.
\end{definition}


\begin{example}[Règle de réécriture]
    Soit un arbre $T$ et la règle de réécriture $rule(u.i)$ et la contrainte $|\sigma(x)| = i$ donnée figure~\ref{fig:sch:pre:rewritting:ex}.
    $U$ représente le nœud à la position $u$, $x \in V$ est une variable et $A \in \Sigma$ représente un label de sous arbre.
    La règle est applicable s'il existe un sous arbre $t = T|_u$ avec une substitution $\sigma$ tel que $\forall j, 0 < j < i$ : $t|_j = \sigma(T|_{u.j})$ et $l(u.i) = A$.
    L'arbre de destination \gls{rhs} à une correspondance avec le modèle \gls{lhs}, ici intuitivement claire : $U \mapsto U$ et $x \mapsto x$.
    En cas d'application de la règle, le sous arbre $A$ est supprimé de $T$.

    \begin{figure}[H]
        \centering
        \begin{subfigure}{0.4\textwidth}
            \centering
            \begin{adjustbox}{valign=c, max width=\textwidth}
                \begin{forest}
                    for tree={s sep=5em}
                    [$U$ [$x$] [$A$]]
                \end{forest}
            \end{adjustbox}
            \caption{\glsfirst*{lhs}}
        \end{subfigure}
        \begin{subfigure}{0.1\textwidth}
            \centering
            \huge{$\Rightarrow$}
        \end{subfigure}
        \begin{subfigure}{0.4\textwidth}
            \centering
            \begin{adjustbox}{valign=c, max width=\textwidth}
                \begin{forest}
                    for tree={s sep=20mm}
                    [$U$ [$x$]]
                \end{forest}
            \end{adjustbox}
            \caption{\glsfirst*{rhs}}
        \end{subfigure}
        \caption{Exemple d'une règle de réécriture}
        \label{fig:sch:pre:rewritting:ex}
    \end{figure}
\end{example}
