\begin{wrapfigure}[26]{r}{.4\textwidth}
    \centering
    \tikzstyle{flow-node} = [text centered, minimum height=2em, minimum width=5em, text width=5em, draw]
    \tikzstyle{flow-start} = [rectangle, rounded corners, flow-node]
    \tikzstyle{flow-process} = [rectangle, text centered, minimum height=2em, text width=10em, draw]
    \tikzstyle{flow-data} = [trapezium, trapezium left angle=70, trapezium right angle=110, flow-node]
    \tikzstyle{flow-decision} = [diamond, aspect=2.5, flow-node]
    \tikzstyle{flow-arrow} = [thick, ->, >=latex, rounded corners]

    \begin{adjustbox}{max width=\linewidth,max height=.95\textheight,valign=c}
        \begin{tikzpicture}[node distance=2em and 3em]
            \node (start) [flow-start] {Debut};
            \node (tal) [flow-data, below=of start] {Textes analysés};
            \node (enriched) [flow-process, below=of tal] {Enrichissement des arbres};
            \node (grammar) [flow-process, below=of enriched] {Extraction de la grammaire $G_i$};
            \node (check) [flow-decision, below=of grammar] {$G_i = G_T$};
            \node (end) [flow-start, left=of check] {Fin};
            \node (equiv) [flow-process, below=of check] {Construction des classes d'équivalence};
            \node (rewrite) [flow-process, below=of equiv] {Réécriture des arbres};

            \draw [flow-arrow] (start) -- (tal);
            \draw [flow-arrow] (tal) -- (enriched);
            \draw [flow-arrow] (enriched) -- (grammar);
            \draw [flow-arrow] (grammar) -- (check);
            \draw [flow-arrow] (check) -- node[above] {Oui} (end);
            \draw [flow-arrow] (check) -- node[left] {Non} (equiv);
            \draw [flow-arrow] (equiv) -- (rewrite);
            \draw [flow-arrow] (rewrite) |- + (3, -1) |- (grammar);
        \end{tikzpicture}
    \end{adjustbox}

    \caption{Processus de structuration sous forme de diagramme de flux}
    \label{fig:struct:flow}
\end{wrapfigure}

Les arbres syntaxiques constituent la fondation de la structuration automatique de l'information.
L'objectif final est d'obtenir des arbres contenant uniquement les entités hiérarchisées conformément à la structure de l'arbre de syntaxe et valide selon la méta-grammaire $\mathbb{G}$.
Plusieurs traitements sont nécessaires pour la préparation des textes à la suite de l'analyse syntaxique.
L'objectif est de simplifier les arbres sur lesquels nous travaillons et d'incorporer les informations extraites, telles que les entités et les relations, au sein de l'arbre de syntaxe.
En effet, dans le domaine du traitement automatique du langage, la tâche d'extraction d'information inclue généralement une étape d'extraction d'entités nommées suivie par une étape d'extraction de relations entre ces entités.
Une entité nommée est une expression linguistique référentielle, souvent associée aux noms propres et aux descriptions définies.
Elle est souvent définie comme une représentation d'un objet ou d'un concept du monde physique.
Ainsi, les entités nommées peuvent être considérées comme des instances d'une classe.
Elles se composent donc d'un type (la classe) et d'une valeur (représentant l'instance).
Par exemple, \textquote{Paris} peut représenter une entité de type \textquote{Ville} avec la valeur \textquote{Paris}.
Lors de l'extraction des informations dans un texte, une fois que les entités sont identifiées, l'extraction de relations est une tâche qui consiste à détecter si deux entités ou plus sont liées et à classer le lien.
Les entités représentent les unités d'information, les données, qui seront structurées en vue de leur enregistrement en base de données.

%Par conséquent, toute partie de l'arbre qui ne fait pas référence à une entité peut être éliminée, car elle n'est pas porteuse d'information.

\begin{definition}[Entité nommée]
    \label{def:struct:entity}
    Une entité nommée est définie comme un tuple $E = (entityName,$ $startToken,$ $endToken)$ où $entityName$ est le nom ou le type de l'entité nommée, $startToken$ est l'indice du token qui marque le début de l'entité, et $endToken$ est l'indice du token qui marque la fin de l'entité.
    $L_{tokens}(E)$ correspond à la séquence de tokens faisant partie de l'entité, définie comme $L_{tokens}(E) = [startToken, \dots, endToken]$.
\end{definition}

La figure~\ref{fig:struct:flow} illustre, à l'aide d'un diagramme de flux, les différentes étapes mise en œuvres pour structurer les données textuelles, c'est-à-dire pour obtenir une grammaire représentant le schéma de nos données.
La première étape constitue une phase de préparation des arbres syntaxiques afin d'incorporer les entités et les relations au sein de l'arbre syntaxique.
Cette étape vise aussi la simplification de ses arbres en supprimant les structures redondantes.
Comme indiqué dans la section~\ref{sec:struct:meta}, une seconde étape cruciale consiste à vérifier que nous sommes en mesure de construire un schéma valide qui couvre l'ensemble de nos arbres.
Pour cela, il est impératif de pouvoir construire automatiquement la grammaire de schéma à partir des arbres d'instances.
Pour finir, si le schéma construit n'est pas valide, il est nécessaire de réécrire les arbres.
Pour cela, on commence, dans la troisieme étape par calculer l'ensemble des classes d'équivalences pour les non-terminaux.
Dans la quatrieme étape, la réécriture cherche à unifier les sous-arbres équivalents et à leur donner une structure valide par rapport à la méta-grammaire $\mathbb{G}$.

Les différents traitements proposés dans notre approche reposent sur des régles de réécriture d'arbre (section~\ref{sec:struct:pre:tree-rewritting}).
Dans la suite, nous presentons de façon en detailée les différentes étapes presentées sur la figure~\ref{fig:struct:flow} (sections~\ref{sec:struct:auto:text} a \ref{sec:struct:auto:rewrite}).

\subsection{Enrichissement des arbres}
\label{sec:struct:auto:text}
L'étape de préparation des textes constitue une phase importante.
Les arbres de syntaxe enrichie et simplifié obtenue apres cette phase sont la base de la structuration automatique.
L'ensemble des opérations définie pour cette étapes ont été conçus de maniére empirique.
La representation des arbres d'entités et les simplification effectuée sont guidée par la structure que l'on souhaite obtenir.

\begin{procedure}[htb]
    \caption{enrich\_tree($T$ = ($D$, $l$), $Ents$)}
    \label{algo:struct:enrichTree}

    \SetKwFunction{delelem}{del\_elem}
    \SetKwFunction{fixcoord}{fix\_coord}
    \SetKwFunction{fixconj}{fix\_conj}
    \SetKwFunction{reduce}{reduce}
    \SetKwFunction{unnest}{unnest\_ent}
    \SetKwFunction{insent}{\ref{algo:struct:insent}}
    \SetKwFunction{insrel}{\ref{algo:struct:insrel}}

    \tcp{Réécriture des coordinations}
    \ForEach{$p \in D$\label{algo:struct:inslist:coord}}{
        \fixcoord{$T$, $p$}\;
        \fixconj{$T$, $p$}\;
    }

    \tcp{Enrichissement des arbres}
    Trier l'ensemble $Ents$ par longueur ($|E|$) de façon decroisante\label{algo:struct:inslist:sort}\;
    \ForEach{$E \in Ents$}{
        \insent{$T$, $E$}\label{algo:struct:inslist:ent}\;
    }
    \ForEach{entité englobante de $T$ à la position $u.i$}{
        \unnest{$T$, $u.i$}\label{algo:struct:inslist:unnest}\;
    }

    \tcp{Simplification des arbres}
    \While{$\exists p \in D$ tel que $\textsf{height}(p) = 1$ et $l(p) \neq \textsf{ENT}$}{
        \delelem{$T$, $p$}\label{algo:struct:inslist:del}\;
    }
    \ForEach{$p \in D$}{
        \reduce{$T$, $p$, $\{ENT\}$}\label{algo:struct:inslist:reduce}\;
    }
\end{procedure}

La procèdure~\ref{algo:struct:enrichTree} résume les transformations nécessaires au nettoyage et à l'insertion des ensembles d'entités ($Ents$) au sein d'un arbre syntaxiques ($T$) afin de construire un arbre enrichis.
La procédure commence par réécrire les arbres en simplifiant les sous-arbres de coordination (ligne~\ref{algo:struct:inslist:coord}).
Le tri à la ligne~\ref{algo:struct:inslist:sort} permet de traiter correctement les entités imbriquées afin que leur insertion se fasse du haut vers le bas de l'arbre.
La ligne~\ref{algo:struct:inslist:ent} se charge d'ajouter les entités dans l'arbre et la ligne~\ref{algo:struct:inslist:unnest} convertie les entités imbriquées.
Après l'enrichissement des arbres syntaxiques, l'algorithme s'occupe de la simplification en commençant par supprimer les branches ne contenant pas d'entités (ligne~\ref{algo:struct:inslist:del}) avant de réduire l'arbre (ligne~\ref{algo:struct:inslist:reduce}).

Dans la suite, cette sous-section présentera de façon détaillée les différentes opérations n'nécessaire à l'enrichissement des arbres syntaxique.

\subsubsection{Conjonctions de coordinations}
La structure des arbres syntaxiques est dépendante de la langue étudiée.
Plus particulièrement, on s'intéresse à la représentation des conjonctions de coordination, lesquelles se révèlent être d'excellents candidats pour la représentation des collections.
La figure~\ref{fig:struct:conj} illustre la disparité structurelle entre le français (\ref{fig:struct:conj:fr}) et l'anglais (\ref{fig:struct:conj:en}) dans laes banques d'arbres utilisées.
Le français est annoté avec une structure en profondeur, tandis que l'anglais présente une structure à plat qui sépare bien les composants de la coordination en les mettant au même niveau.
Ainsi, la structure de l'anglais se rapproche davantage de la représentation des collections, lesquels sont représentées par un sous-arbre où chaque enfant constitue un élément de la collection.
Une première étape dans la préparation des textes implique par conséquent la transformation (pour le français) des conjonctions vers la structure, plus adaptée, de l'anglais.

\begin{figure}[ht]
    \centering
    \begin{subfigure}[b]{.48\textwidth}
        \centering
        \begin{adjustbox}{max width=\linewidth}
            \begin{forest}
                where n children=0{tier=word}{}
                [NP
                    [DT [une]]
                    [NN [echographie]]
                    [ADJ [renale]]
                    [COORD
                            [CCONJ [et]]
                            [NP
                                    [DT [une]]
                                    [NN,before computing xy={s/.average={s}{siblings}} [cystographie]]
                                    [ADJ [retrograde]]
                            ]
                    ]
                ]
            \end{forest}
        \end{adjustbox}
        \caption{Conjonction en français}
        \label{fig:struct:conj:fr}
    \end{subfigure}
    \hfill
    \begin{subfigure}[b]{.47\textwidth}
        \centering
        \begin{adjustbox}{max width=\linewidth}
            \begin{forest}
                where n children=0{tier=word}{}
                [COORD
                    [NP
                            [DT [a]]
                            [ADJ,before computing xy={s/.average={s}{siblings}} [renal]]
                            [NN [ultrasound]]
                    ]
                    [CCONJ,before computing xy={s/.average={s}{siblings}} [and]]
                    [NP
                            [DT [a]]
                            [ADJ,before computing xy={s/.average={s}{siblings}} [retrograde]]
                            [NN [cystography]]
                    ]
                ]
            \end{forest}
        \end{adjustbox}
        \caption{Conjonction en anglais}
        \label{fig:struct:conj:en}
    \end{subfigure}
    \caption{Comparaison entre les conjonctions en français et en anglais}
    \label{fig:struct:conj}
\end{figure}

Pour ce faire, on introduit deux règle de réécriture : \textsf{fix\_coord} (figure~\ref{fig:sch:op:fixCoord}) et \textsf{fix\_conj} (figure~\ref{fig:sch:op:fixConj}).
\textsf{fix\_coord} est chargée de mettre a plat la structure imbriquée des sous-arbres de conjonction.
La régle \textsf{fix\_conj} est utilisée pour fusionner plusieurs conjonction imbriqués.
Par exemple, la phrase \textquote{A et B et C} sera representé par un sous-arbre de conjonction $T_1$ entre A et $T_2$ où $T_2$ est un sous-arbre de conjonction entre B et C.
Comme une collection est representée à plat, cette régle à pour objectif de transformer cette double conjonction en un seul sous-arbre ayant pour enfants A, B et C.

\begin{figure}[htb]
    \centering
    \begin{subfigure}[c]{0.4\textwidth}
        \centering
        \begin{adjustbox}{valign=c, max width=\textwidth}
            \begin{forest}
                for tree={s sep=4em}
                [U [$x$] [$y_1$[ $y_2$] [$x_{coord}$ [$x_{cconj}$] [$y_3$]]] [$z$]]
            \end{forest}
        \end{adjustbox}
        \caption*{\glsname*{lhs}}
    \end{subfigure}
    \begin{subfigure}[c]{0.1\textwidth}
        \centering
        \Large{$\longrightarrow$}
    \end{subfigure}
    \begin{subfigure}[c]{0.4\textwidth}
        \centering
        \begin{adjustbox}{valign=c, max width=\textwidth}
            \begin{forest}
                for tree={s sep=4em}
                [U [$x$] [CONJ [$y_1$ [$y_2$]] [$y_3$]] [$z$]]
            \end{forest}
        \end{adjustbox}
        \caption*{\glsname*{rhs}}
    \end{subfigure}
    \caption[Règle $\textsf{fix\_coord}(T, u.i)$]{$\textsf{fix\_coord}(T, u.i)$ où $|\sigma(x)| = i$, $|\sigma(x_{coord})| = 1$, $|\sigma(x_{cconj})| = 1$, $t(\sigma(x_{coord})) = `COORD`$ et $t(\sigma(x_{cconj})) = `CCONJ`$}
    \label{fig:sch:op:fixCoord}
\end{figure}

\begin{figure}[htb]
    \centering
    \begin{subfigure}[c]{0.4\textwidth}
        \centering
        \begin{adjustbox}{valign=c, max width=\textwidth}
            \begin{forest}
                for tree={s sep=4em}
                [U [$x$] [$conj_1$ [$y_1$] [$conj_2$ [$y_2$]] [$y_3$]] [$z$]]
            \end{forest}
        \end{adjustbox}
        \caption*{\glsname*{lhs}}
    \end{subfigure}
    \begin{subfigure}[c]{0.1\textwidth}
        \centering
        \Large{$\longrightarrow$}
    \end{subfigure}
    \begin{subfigure}[c]{0.4\textwidth}
        \centering
        \begin{adjustbox}{valign=c, max width=\textwidth}
            \begin{forest}
                for tree={s sep=4em}
                [U [$x$] [CONJ [$y_1$] [$y_2$] [$y_3$]] [$z$]]
            \end{forest}
        \end{adjustbox}
        \caption*{\glsname*{rhs}}
    \end{subfigure}
    \caption[Règle $\ProcFnt{fix\_conj}(T, u.i)$]{$\ProcFnt{fix\_conj}(T, u.i)$ où $t(\sigma(conj_1)) = `CONJ`$, $t(\sigma(conj_2)) = `CONJ`$ and $|\sigma(conj_2)| = 1$}
    \label{fig:sch:op:fixConj}
\end{figure}

\subsubsection{Enrichissement des arbres}
L'intégration d'une entité dans un arbre syntaxique se concrétise par la création d'un sous-arbre d'entité avec pour racine un nœud étiqueté \emph{ENT} auquel est attribué le type de l'entité (Personne, Pays, Maladie, etc.) et des feuilles représentant les tokens qui composent l'entité.
L'enrichissement des arbres syntaxiques consiste à intégrer l'ensemble des sous-arbres d'entité.

\begin{definition}[Sous-arbre ordonné d'entité]
    \label{def:struct:entity-tree}
    Soit un arbre syntaxique $T = (D, l)$, une entité nommée $E = (entityName,$ $startToken,$ $endToken)$ et $L_{tokens}(E)$ la liste des indices des tokens de $E$ donnée dans la définition~\ref{def:struct:entity} (page~\pageref{def:struct:entity}).
    Un sous-arbre ordonné d'entité est défini par un tuple $E_T = (entityName, L_{tree}(T, E))$.
    $L_{tree}(T, E)$ est la séquence des positions des tokens de $E$ dans l'arbre $T$ tel que $L_{tree}(T, E) = [u.b,$ $\dots,$ $u.e]= [treePos(T, startToken),$ $\dots,$ $treePos(T, endToken)]$ où $u$ est la position commune à tous les tokens de $E$ et $treePos : \mathbb{T} \times \mathbb{N} \to D$ est une fonction qui associe pour chaque indice d'un token sa position dans l'arbre.
    $|E|$ est la taille de l'entité (ou nombre de tokens) telle que :
    \[
        |E| = |L_{tokens}(E)| = |L_{tree}(T, E)| = (endToken - startToken) + 1
    \]
\end{definition}

Dans certaines situations, les entités peuvent être imbriquées, c'est-à-dire qu'elles peuvent être contenues dans une autre, resultant en deux nœuds entités d'ont l'un est ancêtre de l'autre.
Cependant, cette structure est considérée comme invalide selon la méta-grammaire $\mathbb{G}$, où il est spécifié que les nœuds entités ne peuvent pas avoir une entité comme parent.
Ainsi, pour s'aligner sur la méta-grammaire, il a été choisi de représenter cette imbrication sous la forme d'une relation entre l'entité englobante et la collection des entités contenues (c.-à-d. un arbre ayant deux enfants : l'entité englobante et l'entité englobée).

\begin{definition}
    \label{def:struct:nestedEnt}
    Lorsqu'un nœud du type entité a pour ancêtre un nœud qui est aussi du type entité, la règle de réécriture donnée par la figure~\ref{fig:sch:op:unnestEnt} est appliquée pour introduire un nouveau nœud \emph{ER\_NODE} à la place de l'entité englobante.
    Ce nœud symbolise la relation entre l'entité englobante et les entités englobées.
    Ce sous-arbre a pour enfants :
    \begin{enumerate}[label=(\roman*)]
        \item l'arbre d'entité englobante, qui a pour feuille le texte complet couvert par l'entité englobante et
        \item un nœud \emph{EC\_NODE}, symbolisant une collection et possédant comme enfants l'ensemble des arbres d'entités englobées.
    \end{enumerate}
\end{definition}

% Si une entité englobe une autre dans le texte, la procédure~\ref{algo:struct:insent} construit un arbre d'entité pour l'entité englobante qui contient, comme enfant, l'entité englobée.
% L'arbre d'entité de l'entité englobante n'est donc pas valide.
% Les entités imbriquées sont alors réécrites en accord avec la description faite précédemment.
% Pour ce faire, on introduit un nouveau nœud \emph{ER\_NODE} faisant le lien entre les deux entités qui se retrouve alors au même niveau.
% Cependant, si plusieurs entités se retrouvent englobées, elles seront d'abord regroupées dans un nœud \emph{EC\_NODE}.
% Cela permet de préserver la proximité des entités englobées tout en mettant en avant une relation entre l'entité englobante et les entités englobée.
% La règle donnée dans la figure~\ref{fig:sch:op:unnestEnt} permet cette transformation.

\begin{figure}[htb]
    \centering
    \begin{subfigure}[c]{0.4\textwidth}
        \centering
        \begin{adjustbox}{valign=c, max width=\textwidth}
            \begin{forest}
                % for tree={s sep=3em}
                [U
                        [$x_0$]
                        [ENT
                                [$z_0$]
                                [ENT$_0$ [$y_0$]]
                                [$z_1$]
                                [$\dots$]
                                [ENT$_n$ [$y_n$]]
                                [$z_{n+1}$]
                        ]
                        [$x_1$]
                ]
            \end{forest}
        \end{adjustbox}
        \caption*{\glsname*{lhs}}
    \end{subfigure}
    \begin{subfigure}[c]{0.1\textwidth}
        \centering
        \Large{$\longrightarrow$}
    \end{subfigure}
    \begin{subfigure}[c]{0.4\textwidth}
        \centering
        \begin{adjustbox}{valign=c, max width=\textwidth}
            \begin{forest}
                %for tree={s sep=5em}
                [U
                        [$x_0$]
                        [ER\_NODE
                            [ENT [$z_0$] [$y_0$] [$z_1$] [$\dots$] [$y_n$] [$z_{n+1}$]]
                            [EC\_NODE
                                [ENT$_0$ [$y_0$]]
                                [$\dots$]
                                [ENT$_n$ [$y_n$]]
                            ]
                        ]
                        [$x_1$]
                ]
            \end{forest}
        \end{adjustbox}
        \caption*{\glsname*{rhs}}
    \end{subfigure}
    \caption[Règle $\ProcFnt{unnest\_ent}(T, u.i)$]{$\textsf{unnest\_ent}(T, u.i)$ où $|\sigma(x_0)| = i$}
    \label{fig:sch:op:unnestEnt}
\end{figure}

% L'intégration d'une entité dans un arbre syntaxique se concrétise par la création d'un nœud \emph{ENT}, auquel est attribué le type de l'entité (Personne, Pays, Maladie, etc.).
% Ce nœud agit en tant que parent englobant l'ensemble des mots qui composent l'entité, correspondant au nœud dont la position représente le plus grand préfixe commun à tous les termes de l'entité.
% Par exemple, dans la figure~\ref{fig:struct:enrichissement:2} le nœud \emph{ENT\_PERSON} à la position $00$ de l'arbre et a pour enfants les mots \emph{John} À la posiion $0000$ et \emph{Smith} à la position $0010$.
% Dans certaines situations, les entités peuvent être imbriquées, c'est-à-dire qu'elles peuvent être contenues dans une autre.
% Dans l'arbre syntaxique d'origine, lorsqu'une entité $A$ englobe une entité $B$, le nœud représentant l'entité $A$ est parent du nœud de l'entité $B$, ou occupe la même position que ce dernier.
% Dans l'arbre enrichi, ils seront representé par deux noeud distincts.
% Cependant, cette structure est considérée comme invalide selon la méta-grammaire $\mathbb{G}$, où il est spécifié que les nœuds entités ne peuvent pas avoir une entité comme parent.
% Pour s'aligner sur la métagrammaire, il a été choisi de représenter cette imbrication sous la forme d'une relation entre l'entité englobante et la liste des entités contenues.
% Pour ce faire, un nouveau nœud \emph{ER\_NODE} est introduit pour symboliser cette relation et a comme enfants :
% \begin{enumerate*}[label=(\roman*)]
%     \item l'arbre d'entité englobant, qui a pour feuille le texte complet couvert par l'entité englobante, et
%     \item un nœud \emph{EC\_NODE}, symbolisant la collection et possédant comme enfants l'ensemble des arbres d'entités contenues.
% \end{enumerate*}
% L'exemple suivant illustre la procédure.

\begin{example}
    La figure~\ref{fig:struct:enrichissement} illustre un exemple d'enrichissement d'un arbre syntaxique pour la phrase : \enquote{John Smith takes 500mg of Paracetamol}.
    Nous considérons l'arbre ordonné présenté dans la figure~\ref{fig:struct:enrichissement:1} ainsi que les entités suivantes :
    \begin{multicols}{3}
        \begin{enumerate}
            \item \enquote{John Smith} (Person) \label{ex:struct:enrich-ent:person}
            \item \enquote{500 mg} (Dosage) \label{ex:struct:enrich-ent:dosage}
            \item \enquote{500} (Value) \label{ex:struct:enrich-ent:val}
            \item \enquote{mg} (Unit) \label{ex:struct:enrich-ent:unit}
            \item \enquote{Paracetamol} (Drug) \label{ex:struct:enrich-ent:drug}
        \end{enumerate}
    \end{multicols}

    Pour la première entité, \enquote{John} est situé à la position $000$ dans l'arbre de syntaxe et \enquote{Smith} à la position $010$.
    Le nœud représentant cette entité doit donc être inséré comme enfant du nœud à la position $0$, qui correspond au préfixe commun à \enquote{John} et \enquote{Smith} (respectivement aux positions $000$ et $010$).
    Les entités \ref{ex:struct:enrich-ent:val} et \ref{ex:struct:enrich-ent:unit} sont des entités imbriquées dans l'entité \ref{ex:struct:enrich-ent:dosage}, comme illustrée dans la figure~\ref{fig:struct:enrichissement:2}.
    Par conséquent, il est nécessaire de corriger l'arbre de la figure~\ref{fig:struct:enrichissement:2} en s'introduisant la structure de relation présentée dans la définition~\ref{def:struct:nestedEnt}.

    On introduit alors le nœud \emph{ER\_NODE} à la position de l'entité englobante, c'est-à-dire, comme enfant du nœud \emph{NP} situé à la position $110$ dans l'arbre de syntaxe (figure~\ref{fig:struct:enrichissement:1}).
    Ainsi, dans la figure~\ref{fig:struct:enrichissement:3}, \emph{ER\_NODE} est racine d'un sous-arbre qui contient:
    \begin{enumerate}
        \item comme premier fils, le sous-arbre ayant l'entité \emph{ENT\_DOSAGE} comme racine et deux feuilles: \enquote{500} à la position $110000$ et \enquote{mg} à la position $110001$ ;
        \item comme deuxième fils, le sous-arbre ayant l'entité \emph{EC\_NODE} comme racine et comme fils :
              \begin{enumerate}
                  \item l'entité \emph{ENT\_VALUE} ayant \enquote{500} comme feuille et
                  \item l'entité \emph{ENT\_UNIT} ayant \enquote{mg} comme feuille.
              \end{enumerate}
    \end{enumerate}

    % Pour la première entité, \emph{John} est situé à la position $000$ dans l'arbre de syntaxe et \emph{Smith} à la position $010$.
    % Le nœud représentant cette entité doit donc être inséré comme enfant du nœud à la position $0$, qui correspond au préfixe commun à \emph{John} et \emph{Smith} (respectivement aux positions $000$ et $010$).
    % Les entités \ref{ex:struct:enrich-ent:val} et \ref{ex:struct:enrich-ent:unit} sont des entités imbriquées dans l'entité \ref{ex:struct:enrich-ent:dosage} comme illustré par la figure~\ref{fig:struct:enrichissement:2}.
    % Par conséquent, il est nécessaire de corriger l'arbre~\ref{fig:struct:enrichissement:2} en introduisant la structure de relation conformément à la définition~\ref{}.
    % On introduit alors le nœud \emph{ER\_NODE} à la position de l'entité englobante, c'est-à-dire, comme enfant du nœud \emph{NP} situé à la position $110$ dans l'arbre de syntaxe.
    % Ainsi, dans la figure~\ref{fig:struct:enrichissement:3}, \emph{ER\_NODE} est racine d'un sous-arbre qui contient les mots \emph{500} à la position $11000$ et \emph{mg} à la position $11010$.
    % À noter que $110$ est le plus grand préfixe commun entre $11000$ et  $11010$.
    % Sous le nœud \emph{ER\_NODE}, on ajoute l'arbre d'entité \emph{ENT\_DOSAGE} (correspondant à l'entité \ref{ex:struct:enrich-ent:dosage}, \emph{dosage}) et le nœud \emph{EC\_NODE}.
    % Sous le nœud \emph{EC\_NODE}, on conserve le sous-arbre qui était initialement positionné sous la position $110$.
    % Ce sous-arbre contient les entitiés imbriquées \emph{ENT\_VALUE} et \emph{ENT\_UNIT}.
    % Ensuite, le reste des entités peuvent être annotés de la même manière que la première entité.
    % Le résultat de cette opération d'enrichissement est illustré dans la figure~\ref{fig:struct:enrichissement:2}.
    % Intuitivement, le nœud \emph{ER\_NODE} représente une relation entre l'entité englobante et une collection d'entité imbriquée, matérialisée par le nœud \emph{EC\_NODE}.
\end{example}

\begin{figure}[htb]
    \centering
    \begin{subfigure}[b]{.4\textwidth}
        \centering
        \begin{adjustbox}{max width=\linewidth}
            \begin{forest}
                where n children=0{tier=word}{}
                [S
                    [NP
                            [NNP [John]]
                            [NNP [Smith]]
                    ]
                    [VP
                            [VBZ [takes]]
                            [NP
                                    [NP
                                            [CD [500]]
                                            [NN [mg]]
                                    ]
                                    [PP
                                            [IN [of]]
                                            [NPP [NN [Paracetamol]]]
                                    ]
                            ]
                    ]
                ]
            \end{forest}
        \end{adjustbox}
        \caption{Arbre de syntaxe}
        \label{fig:struct:enrichissement:1}
    \end{subfigure}
    \hfill
    \begin{subfigure}[b]{.55\textwidth}
        \centering
        \begin{adjustbox}{max width=\linewidth}
            \begin{forest}
                where n children=0{tier=word}{}
                [S
                    [NP [\textbf{ENT\_PERSON}
                                [John]
                                [Smith]
                            ]]
                    [VP
                            [VBZ [takes]]
                            [NP
                                    [NP [\textbf{ENT\_DOSAGE}
                                                [\textbf{ENT\_VALUE} [500]]
                                                [\textbf{ENT\_UNIT} [mg]]
                                            ]]
                                    [PP
                                            [IN [of]]
                                            [NP [NNP [\textbf{ENT\_DRUG} [Paracetamol]]]]
                                    ]
                            ]
                    ]
                ]
            \end{forest}
        \end{adjustbox}
        \caption{Arbre en construction, avec entité englobante}
        \label{fig:struct:enrichissement:2}
    \end{subfigure}
    \hfill
    \begin{subfigure}[b]{.65\textwidth}
        \centering
        \begin{adjustbox}{max width=\linewidth}
            \begin{forest}
                where n children=0{tier=word}{}
                [S
                    [NP [\textbf{ENT\_PERSON}
                                [John]
                                [Smith]
                            ]]
                    [VP
                            [VBZ [takes]]
                            [NP
                                    [NP [\textbf{ER\_NODE}
                                                [\textbf{ENT\_DOSAGE} [500] [mg]]
                                                [\textbf{EC\_NODE}
                                                    [\textbf{ENT\_VALUE} [500]]
                                                    [\textbf{ENT\_UNIT} [mg]]
                                                ]
                                            ]]
                                    [PP
                                            [IN [of]]
                                            [NP [NNP [\textbf{ENT\_DRUG} [Paracetamol]]]]
                                    ]
                            ]
                    ]
                ]
            \end{forest}
        \end{adjustbox}
        \caption{Arbre enrichi}
        \label{fig:struct:enrichissement:3}
    \end{subfigure}

    \caption{Exemple d'enrichissement d'arbre de syntaxe}
    \label{fig:struct:enrichissement}
\end{figure}

% Pour formaliser l'operation d'enrichissement de l'arbre de syntaxique, on commence par définir les sous-arbres d'entités.
% Ces sous-arbres sont la représentation d'une entité nommée dans un arbre de syntaxe.

% \begin{definition}[Sous-arbre ordonné d'entité]
%     \label{def:struct:entity-tree}
%     Soit un arbre syntaxique $T = (D, l)$, une entité nommée $E = (entityName,$ $startToken,$ $endToken)$ et $L_{tokens}(E)$ la liste des indices des tokens de $E$ donnée dans la définition~\ref{def:struct:entity} (page~\pageref{def:struct:entity}).
%     Un sous-arbre ordonné d'entité est défini par un tuple $E_T = (entityName, L_{tree}(T, E))$.
%     $L_{tree}(T, E)$ est la séquence des positions des tokens de $E$ dans l'arbre $T$ tel que $L_{tree}(T, E) = [u.b,$ $\dots,$ $u.e]= [treePos(T, startToken),$ $\dots,$ $treePos(T, endToken)]$ où $u$ est la position commune à tous les tokens de $E$ et $treePos : \mathbb{T} \times \mathbb{N} \to D$ est une fonction qui associe pour chaque indice d'un token sa position dans l'arbre.
%     $|E|$ est la taille de l'entité (ou nombre de tokens) telle que :
%     \[
%         |E| = |L_{tokens}(E)| = |L_{tree}(T, E)| = (endToken - startToken) + 1
%     \]
% \end{definition}

% La procédure pour enrichir les arbres de syntaxe est donnée par la procédure~\ref{algo:struct:insent} (page~\pageref{algo:struct:insent}) donnée ci-après.
% La ligne~\ref{algo:struct:insent:sum} permet de vérifier si le premier token de l'entité n'a pas de frère à gauche sous la position $u$ qui ne ferait alors pas partie de l'entité.
% Si c'est le cas, il est nécessaire de laisser la partie gauche à sa place et d'introduire un nouveau sous-arbre pour représenter l'entité, sinon on peut simplement convertir le sous-arbre à la position $u.x_0$.
% Ici, on est obligé de garder le nœud $u.x_0$ s'il n'y a pas d'enfants à gauche, sinon lors de la suppression des feuilles sa suppression pourrait engendrer la suppression du nœud parent étant donné que la suppression est récursive.
% La boucle tant que (ligne~\ref{algo:struct:insent:while}) collecte les feuilles de l'entité dans $H$ et les supprime de l'arbre.
% À la fin de la boucle, l'ensemble de l'arbre correspondant à l'entité est supprimé.
% Il reste les éléments à gauche et à droite sous la position $u$ qui ne font pas partie de l'entité.
% À la ligne~\ref{algo:struct:insent:add}, on insère un nouveau nœud correspondant à l'entité à la position $u.x'$ entre la partie gauche et droite.
% Pour finir, la boucle à la ligne~\ref{algo:struct:insent:leafs} ajoute les feuilles à l'arbre d'entité.

La procédure~\ref{algo:struct:insent} (page~\pageref{algo:struct:insent}) met en place l'enrichissement d'un arbre par l'insertion d'une entité.
Pour comprendre le rôle de cette procédure, prenons comme exemple l'insertion de l'entité $E_{T} = (ENT_1, [u.1.0, u.1.1, u.2.0, u.2.1])$  dans l'arbre de la figure~\ref{fig:struct:enrich:1}.
Nous rappelons que les labels des nœuds internes sont des non-terminaux.
La question consiste à savoir dans quelle position de l'arbre, nous allons insérer le sous-arbre entité avec une nouvelle étiquette, $ENT1$, pour représenter une entité nommée.
Dans ce cas simple, l'entité se retrouve comme enfant à la position $u.1$ et est le résultat de la fusion des nœuds aux position $u.1$ et $u.2$.
Le résultat est celui de la figure~\ref{fig:struct:enrich:2}.
Néanmoins, dans une situation légèrement différente, nous souhaitons insérer l'entité $E_{T} = (ENT_2, [u.1.1, u.2.0])$ dans l'arbre de la figure~\ref{fig:struct:enrich:1}.
L'idée ici est de l'insérer comme frère à droite de $u.1$, c'est-à-dire, à la position $u.2$.
En faisant cela, nous obtenons l'arbre de la figure~\ref{fig:struct:enrich:3}.

\begin{figure}[htb]
    \centering
    \begin{subfigure}[c]{.9\textwidth}
        \centering
        \begin{adjustbox}{valign=c, max width=\textwidth}
            \begin{forest}
                where n children=0{tier=word}{}
                [$\dots$
                [$N_0$ ($u$)
                        [$N_1$ ($u.0$)
                                [token1 ($u.0.0$)]
                            ]
                            [$N_2$ ($u.1$)
                                [token2 ($u.1.0$)]
                                    [token3 ($u.1.1$)]
                            ]
                            [$N_3$ ($u.2$)
                                [token4 ($u.2.0$)]
                                    [token5 ($u.2.1$)]
                            ]
                            [$N_4$ ($u.3$)
                                [token6 ($u.3.0$)]
                            ]
                    ]
                ]
            \end{forest}
        \end{adjustbox}
        \caption{Arbre avant enrichissement}
        \label{fig:struct:enrich:1}
    \end{subfigure}
    \hfill
    \begin{subfigure}[c]{.9\textwidth}
        \centering
        \begin{adjustbox}{valign=c, max width=\textwidth}
            \begin{forest}
                where n children=0{tier=word}{}
                [$\dots$
                [$N_0$ ($u$)
                        [$N_1$ ($u.0$)
                                [token1 ($u.0.0$)]
                            ]
                            [$ENT_1$ ($u.1$)
                                [token2 ($u.1.0$)]
                                    [token3 ($u.1.1$)]
                                    [token4 ($u.1.2$)]
                                    [token5 ($u.1.3$)]
                            ]
                            [$N_4$ ($u.2$)
                                [token6 ($u.2.0$)]
                            ]
                    ]
                ]
            \end{forest}
        \end{adjustbox}
        \caption{Ajout de l'entité $E_{T} = (ENT_1, [u.1.0, u.1.1, u.2.0, u.2.1])$}
        \label{fig:struct:enrich:2}
    \end{subfigure}
    \hfill
    \begin{subfigure}[c]{.9\textwidth}
        \centering
        \begin{adjustbox}{valign=c, max width=\textwidth}
            \begin{forest}
                where n children=0{tier=word}{}
                [$\dots$
                [$N_0$ ($u$)
                        [$N_1$ ($u.0$)
                                [token1 ($u.0.0$)]
                            ]
                            [$N_2$ ($u.1$)
                                [token2 ($u.1.0$)]
                            ]
                            [$ENT_2$ ($u.2$)
                                [token3 ($u.2.0$)]
                                    [token4 ($u.2.1$)]
                            ]
                            [$N_3$ ($u.3$)
                                [token5 ($u.3.0$)]
                            ]
                            [$N_4$ ($u.4$)
                                [token6 ($u.4.0$)]
                            ]
                    ]
                ]
            \end{forest}
        \end{adjustbox}
        \caption{Ajout de l'entité $E_{T} = (ENT_2, [u.1.1, u.2.0])$}
        \label{fig:struct:enrich:3}
    \end{subfigure}
    \caption{Exemple de placement d'un sous-arbre entité}
    \label{fig:struct:enrich}
\end{figure}

L'objectif est de faire évoluer un arbre $T$ pour qu'il y contienne un sous-arbre où la racine est étiquetée par le nom de l'entité et ses enfants (feuilles) représentent les tokens qui constituent de cette entité.
Autrement dit, il s'agit de modifier $T$ pour y placer un sous-arbre entité.
Sois $u$ le plus grand préfixe commun des positions dans $L_{tree}(T,E)$.
Le sous-arbre entité sera placé comme un fils de la position $u$.
Pour savoir dans quelle position $u.i$ de $T$ nous devons placer le sous-arbre entité, il faut prendre en compte la position $p \in D$ du premier token de $E$.
Dans l'arbre original, soit $T|_{u.x_0}$ le sous-arbre, fils de $u$, qui contient le premier token de $E$.
Nous allons ajouter le sous-arbre entité comme frère à droite de $T|_{u.x_0}$ et supprimer les sous-arbres contenant les positions originales des tokens.
Ainsi, dans le cas de l'entité $ENT_1$, l'insertion du sous-arbre d'entité est faite à la position $u.2$, mais la suppression de $T_{\mid_{u.1.0}}$ entraîne la suppression de $T_{\mid_{u.1}}$ donnant comme résultat l'arbre de la figure~\ref{fig:struct:enrich:2}.
Pour l'entité $ENT_2$, le sous-arbre d'entité est inséré à la position $u.2$ et le sous-arbre $T|_{u.1.1}$ est supprimé.

% \begin{procedure}[htb]
%     \caption{ins\_ent($T$ = ($D$, $l$), $E$)}
%     \label{algo:struct:insent}

%     \SetKwArray{ltree}{$L_{tree}$}
%     \SetKwFunction{inselem}{ins\_elem}
%     \SetKwFunction{delelem}{del\_elem}

%     $[u.b, \dots, u.e] \gets L_{tree}(T, E)$\;
%     $u.x_0.x_1 \dots x_n \gets u.b$ \tcp*{On decompose la position de début en composants $x_0, x_1, \dots$}
%     \tcp{Vérifier si $u.x_0$ a des enfants qui ne sont pas dans l'entité et qui doivent être laissés en place}
%     \uIf{$\sum_{i=1}^{n} (x_i) \neq 0$\label{algo:struct:insent:sum}}{
%         $x' \gets x_0 + 1$\;
%     } \Else {
%         $x' \gets x_0$\;
%     }
%     $j \gets |E| - 1$\;
%     $H \gets []$\;
%     \While{$j \geq 0$\label{algo:struct:insent:while}}{
%         $pos \gets \ltree{j}$\;
%         $H \gets H.\textsf{append}(T|_{pos})$\;
%         \delelem{$T$, $pos$, $u$}\;
%         $j \gets j-1$\;
%     }
%     \inselem{$T$, entityName, $u.x'$}\label{algo:struct:insent:add}\;
%     \For{$leaf \in H$}{
%         \inselem{$T$, leaf, $u.x'.0$}\label{algo:struct:insent:leafs}\;
%     }
% \end{procedure}

\begin{procedure}[htb]
    \caption{ins\_ent($T$ , $E$)}
    \label{algo:struct:insent}

    \SetKwFunction{inselem}{ins\_elem}
    \SetKwFunction{delelem}{del\_elem}

    $[u.b, \dots, u.e] \gets L_{tree}(T, E)$ \tcp*{$u$ est le plus grand préfixe commun des positions des tokens de $E$ dans l'arbre $T$}
    $u.x_0.x_1 \dots x_n \gets u.b$ \tcp*{Pour $u.b$, nous explicitons chaque composant de $b$:  $x_0, x_1, \dots$}
    \tcp{$u.x_0$ est la racine du sous-arbre, fils de $u$,  qui contient le premier token de $E$}
    \inselem{$T$, entityName, $u.(x_0+1)$}\label{algo:struct:insent:add} \tcp*{On ajoute à droite de la position $u. x_0$ un nouveau noeud étiqueté avec le nom de l'entité à ajouter. C'est donc la racine du sous-arbre entité.}
    $[u.b', \dots, u.e'] \gets L_{tree}(T, E)$ \label{algo:struct:insent:newPos}\tcp*{On recalcule les positions des tokens, éventuellement modifiées par l'insertion précédente. On obtient ainsi une nouvelle liste de positions}
    $j \gets |E| - 1$\;
    \While{$j \geq 0$\label{algo:struct:insent:while}}{
        \tcp{On parcours toutes les positions de la liste $[u.b', \dots, u.e']$ en partant de la fin }
        $pos \gets [u.b', \dots, u.e'] [j]$\tcp*{$pos$ est la position d'un token}
        \inselem{$T$,$ l(pos)$, $u.(x_0+1).0$}\tcp*{On insère le token comme feuille du sous-arbre entité}
        \delelem{$T$, $pos$} \tcp*{On supprime la feuille qui originalement correspondait au token qui vient d'être rajouté} \label{algo:struct:insent:del}
    }
\end{procedure}

La procédure~\ref{algo:struct:insent} a pour but d'effectuer l'évolution de $T$ pour y placer le sous-arbre entité.
Elle commence par trouver le plus grand préfixe commun des positions de tous les tokens d'une entité $E$.
La première position $u.b$ sert de guide pour le placement du sous-arbre entité : ce sous-arbre sera placé à la position $u.(x_0+1)$ de l'arbre $T$ (ligne~\ref{algo:struct:insent:add}).
Cet ajout peut impacter les positions à droite de $u.(x_0+1)$, affectant ainsi les positions des autres tokens.
A la ligne~\ref{algo:struct:insent:newPos} les nouvelles positions des tokens sont calculées.
La boucle de la ligne~\ref{algo:struct:insent:while} nous permet de parcourir les tokens de $E$ en les ajoutant comme feuilles du sous-arbre entité et, en même temps, de les supprimer de leur position originale.
Cette série d'insertions et suppressions se fait de droite à gauche, pour éviter de recalculer les positions des tokens à chaque tour de boucle.

\subsubsection{Simplifications}
% La simplification des arbres consiste en la suppression de toutes les branches qui ne contiennent pas d'entité.
% Dans les arbres de syntaxe, il est aussi possible de trouver des structures redondantes où des nœuds intermédiaires ne contiennent qu'un seul enfant.
% Bien que porteur d'information dans l'arbre syntaxique, ces nœuds n'ajoutent pas d'information pour la structuration des entités au sein de l'arbre.
% Une seconde étape de la simplification permet la réduction de ces structures.
% Pour ce faire, on utilise successivement la règle de réécriture \textsf{reduce} présentée dans la figure~\ref{fig:sch:op:reduce}.
% Pour finir, on s'assure que pour chaque entité une seule feuille est présente, représentant la concaténation de l'ensemble des token de l'entité.

La simplification des arbres est faite en deux étapes:
\begin{enumerate}
    \item On supprime les sous-arbres qui ne contiennent pas d'entité.
          Ces sous-arbres sont repérables en regardant les parents de toutes les feuilles de $T$.
          S'ils ne sont pas labellisés par le nom d'une entité alors ils doivent être supprimés.
          Par exemple, dans la figure~\ref{fig:struct:enrichissement:3}, le sous-arbre $T|_{1.1.1.0}$.

    \item On supprime les nœuds qui ne sont pas étiquetés comme entité et qui ont un seul fils.
          Par exemple, dans la figure~\ref{fig:struct:enrichissement:3}, le noeud à la position $0$.
          Pour ce faire, on utilise successivement la règle de réécriture \textsf{reduce} présentée dans la figure~\ref{fig:sch:op:reduce}.
\end{enumerate}

\begin{figure}[htb]
    \centering
    \begin{minipage}[c]{.4\textwidth}
        \centering
        \begin{subfigure}[t]{0.4\textwidth}
            \centering
            \begin{adjustbox}{valign=c, max width=\textwidth}
                \begin{forest}
                    for tree={s sep=2em}
                    [U [$x$] [$y_1$ [$y_2$]] [$z$]]
                \end{forest}
            \end{adjustbox}
            \caption*{\glsname*{lhs}}
        \end{subfigure}
        \begin{subfigure}[c]{0.1\textwidth}
            \centering
            \Large{$\longrightarrow$}
        \end{subfigure}
        \begin{subfigure}[t]{0.4\textwidth}
            \centering
            \begin{adjustbox}{valign=c, max width=\textwidth}
                \begin{forest}
                    for tree={s sep=2em}
                    [U [$x$] [$y_2$] [$z$]]
                \end{forest}
            \end{adjustbox}
            \caption*{\glsname*{rhs}}
        \end{subfigure}
        \caption[Règle $\textsf{reduce}(T, u.i, S_{labels})$]{$\textsf{reduce}(T, u.i, S_{labels})$ où $|\sigma(x)| = i$ et $|\sigma(y_1)| = 1$ avec $|\sigma(y_2)| = 1$ si $S_{label} = \emptyset$ ; $t(u.i) \notin S_{labels}$ si $S_{label} \neq \emptyset$}
        \label{fig:sch:op:reduce}
    \end{minipage}%
    \hfill
    \begin{minipage}[c]{.58\textwidth}
        \centering
        \begin{adjustbox}{max width=.8\linewidth}
            \begin{forest}
                where n children=0{tier=word}{}
                [S
                    [\textbf{ENT\_PERSON}
                        [John] [Smith]
                    ]
                    [NP
                            [\textbf{ER\_NODE}
                                [\textbf{ENT\_DOSAGE} [500] [mg]]
                                [\textbf{EC\_NODE}
                                    [\textbf{ENT\_VALUE} [500]]
                                    [\textbf{ENT\_UNIT} [mg]]
                                ]
                            ]
                            [\textbf{ENT\_DRUG} [Paracetamol]]
                    ]
                ]
            \end{forest}
        \end{adjustbox}
        \caption{Exemple d'arbre enrichie simplifié}
        \label{fig:struct:simplification}
    \end{minipage}%
\end{figure}

\begin{example}
    Dans l'arbre donné dans la figure~\ref{fig:struct:enrichissement:3}, on remarque que pour le nœud \emph{ENT\_PERSON} le nœud parent \emph{NP} peut être supprimé.
    Le sous-arbre du nœud \emph{VBZ} ne contient aucune entité et peut donc être supprimé.
    Sa suppression rend le nœud parent \emph{VP} supprimable à son tour, car il ne possède alors plus qu'un seul enfant.
    En appliquant le raisonnement à tout l'arbre, on obtient l'arbre donné dans la figure~\ref{fig:struct:simplification}.
\end{example}

\subsection{Extraction de la grammaire}
Pour réaliser la migration de l'instance d'une grammaire vers une autre, il est indispensable de pouvoir récupérer la grammaire à partir de l'instance afin de pouvoir la vérifier à l'aide de la méta-grammaire $\mathbb{G}$.
Étant donné une instance $I$ reconnue par une grammaire $G$, il est possible de récupérer la grammaire $G$ à partir de $I$ en calculant l'arbre quotient $S$.
Dans la suite, nous introduisons cette notion.
%Le quotient d'un graphe est une construction ou les nœuds sont regroupées en suivant une relation d'équivalences.
%Chaque nœud du graphe quotient représente alors une classe d'équivalence et il existe une arête entre deux nœuds s'il existe au moins une arête entre deux individus de chaque classe.
Rappelons d'abord qu'en mathématiques, une \emph{partition} d'un ensemble $X$ est un regroupement de ses éléments en sous-ensembles non-vides et mutuellement disjoints.
Toute \emph{relation d'équivalence} sur un ensemble définit une partition de cet ensemble, et toute partition définit une relation d'équivalence.
Une \emph{famille d'ensembles} $F$ est une partition d'un ensemble $X$ si et seulement si toutes les conditions suivantes sont remplies :
\begin{enumerate*}[label=(\roman*)]
    \item $\emptyset \not\in F$ ;
    \item $\bigcup_{A \in F}A = X$ et
    \item $\forall A, B  \in F~ (A \neq B) \Rightarrow ( A \cap B = \emptyset)$.
\end{enumerate*}
Les ensembles de $F$ sont appelés \textit{blocs}.
En théorie des graphes, un graphe quotient $Q$ d'un graphe $G$ est un graphe dont les sommets sont des blocs d'une partition des sommets de $G$ et où un bloc $A$ est adjacent a un bloc $B$ si au moins un sommet de $A$ est adjacent à un sommet de $B$ par rapport à l'ensemble des arêtes de $G$.
Pour construire la grammaire à partir d'un arbre, on introduit la définition d'\textit{arbre quotient}, ce qui correspond à un graphe quotient sans cycles et sans sommets ayant des multiples parents.
Dans la construction d'un arbre quotient la définition de la fonction \textsf{Succ} ci-dessous est nécessaire.

%Pour construire la grammaire à partir d'un arbre, on définit un nouveau quotient qui maintient la structure d'arbre (c.-à-d. sans cycles et sans nœuds ayant plusieurs parents).
%Le quotient est calculé à partir d'une relation d'équivalence sur les labels.
%Pour capturer la structure hiérarchique de ces classes et construire l'arbre quotient, nous introduisons la fonction \textsf{Succ}, qui permet d'obtenir les successeurs de chaque classe d'équivalence.
%
% \begin{definition}[Quotient d'un graphe]
%   Étant donné un graphe $G = (V, E)$ où $V$ est l'ensemble des sommets du graphe et $E$ l'ensemble des arêtes de la forme $\{u, v\} \in E$ où $u, v \in V$.
%   Le quotient du graphe $G$ est un graphe distinct noté $Q_G = (Q_V, Q_E)$ qui respecte les propriétés suivantes :
%   \begin{enumerate}
%     \item \label{struct:quotient:prop1} $Q_V = V / \equiv$
%     \item \label{struct:quotient:prop2} $\{u_Q, v_Q\} \in Q_E \iff \exists u \in u_Q, v \in v_Q$ tel que $\{u, v\} \in V$
%   \end{enumerate}
%   La propriété \ref{struct:quotient:prop1} signifie que chaque sommet de $Q_G$ correspond à une partition de $G$ par la relation d'équivalence $\equiv$.
%   De plus, il existe une arête entre deux nœuds de $Q_G$ si et seulement si, il existe une arête entre au moins une paire d'élément de chaque partition dans $G$ comme définit par la propriété \ref{struct:quotient:prop2}.
% \end{definition}

% \begin{definition}[Quotient d'un arbre ordonné]
%   Étant donné un arbre ordonné $T = (D, l)$, le quotient de l'arbre est un arbre distinct noté $Q_T = (Q_D, Q_l)$ avec une fonction injective de projection $\pi : (\mathbb{N})^* \to \mathcal{P}(D)$ où $\mathcal{P}(D)$ est l'ensemble des parties de $D$ (\emph{power set} en anglais, c.-à-d. l'ensemble des sous ensemble de $D$).
%   $\pi$ respecte les propriétés suivantes :
%   \begin{enumerate}
%     \item \label{struct:quotient-tree:prop1} $\forall u_Q \in Q_D ~ \exists V \in D / \equiv$ tel que $\pi(u_Q) = V$.
%     \item \label{struct:quotient-tree:prop2} $\forall u_Q \in Q_D ~ \exists v_Q \in Q_D$ tel que $v_Q \preceq u_Q \iff \exists u \in \pi(u_Q), v \in \pi(v_Q)$ tel que $v \prec u$
%   \end{enumerate}
%   La propriété \ref{struct:quotient-tree:prop1} signifie que chaque nœud de $Q_T$ correspond à une partition de $T$ par la relation d'équivalence $\equiv$.
%   De plus, il existe une arête entre deux nœuds de $Q_T$ si et seulement si, il existe une arête entre au moins une paire d'élément de chaque partition dans $T$ comme définit par la propriété \ref{struct:quotient-tree:prop2}.
%   La fonction de labellisation $Q_l$ est définie par : $\langle u \mapsto l(\pi(u)) \mid u \in Q_D \rangle$.
%   De plus, pour toute position $u_Q \in Q_D$ s'il existe un $v \in D$ tel que $|\{u \mid u \in \pi(u_Q), v \prec u\}| > 1$ alors $Q_l(u_Q) = Q_l(u_Q)^+$
% \end{definition}

\begin{definition}
    \label{def:struct:quotient:succ}
    Étant donné un arbre $T = (D, l)$ et une relation d'équivalence $R$.
    Soit $D/R = \{C_0, \dots, C_n\}$ l'ensemble des classes d'équivalences de $T$, on définit la fonction \textsf{Succ} tel que :
    \begin{equation}
        \textsf{Succ}(C) = \{ C' \mid \exists u \in C, v \in C' \text{ tq. } u \prec v \}
    \end{equation}
    % \begin{equation}
    %   \textsf{Succ}(C_i) = \begin{cases*}
    %     \{ \epsilon \}                           & si $C_i  = \emptyset$ \\
    %     \{ C_j \mid \exists u \in C_i, p \in C_j \text{ tq. } u \prec p \} & si $C_i \neq \emptyset$
    %   \end{cases*}
    % \end{equation}
\end{definition}

Pour réaliser l'extraction des grammaires $G_i$ nous utilisons la relation d'équivalence $R_l$ entre les labels d'un arbre $T$, définie par $(\forall x \in D, y \in D) ~ x R_l y \iff l(x) = l(y)$.
Intuitivement, on obtient alors comme classes d'équivalences un ensemble de positions pour chaque étiquette présente dans l'arbre $T$.
Par exemple, dans la figure~\ref{fig:struct:quotient:ex:tree}, $C_{\lambda} = \{\epsilon\}$ et $C_{X} = \{0,1\}$ sont les classes d'équivalence pour les étiquettes $\lambda$ et $X$ respectivement.
La fonction \textsf{Succ} permet de calculer la hiérarchie des classes d'équivalence où pour une classe donnée $C$, la fonction retourne l'ensemble des classes d'équivalences qui ont au moins un élément enfant d'un élément de $C$.
Nous pouvons également écrire que $\textsf{Succ}(C) = \bigcup_{u \in C} \{ C' \mid  \exists  i ~u.i \in C'\}$.
L'exemple suivant illustre cette fonction.

\begin{figure}[H]
    \centering
    \begin{subfigure}{0.45\textwidth}
        \centering
        \begin{adjustbox}{valign=c, max width=\textwidth}
            \begin{forest}
                for tree={s sep=5em}
                [$\lambda$
                [$X$ [$a$] [$b$]]
                    [$X$,before computing xy={s/.average={s}{siblings}} [$b$] [$c$]]
                    [$Y$ [$a$]]
                ]
            \end{forest}
        \end{adjustbox}
        \caption{Arbre $T$}
        \label{fig:struct:quotient:ex:tree}
    \end{subfigure}
    \hfill
    \begin{subfigure}{0.45\textwidth}
        \centering
        \begin{adjustbox}{valign=c, max width=\textwidth}
            \begin{forest}
                for tree={s sep=20mm}
                [$\lambda$
                [$X^+$ [$a$] [$b$] [$c$]]
                    [$Y$ [$a$]]
                ]
            \end{forest}
        \end{adjustbox}
        \caption{Arbre quotient $Q_T$}
        \label{fig:struct:quotient:ex:quotient}
    \end{subfigure}
    \caption{Exemple du quotient d'un arbre}
    \label{fig:struct:quotient:ex}
\end{figure}

\begin{example}
    \label{ex:struct:quotient:1}
    Étant donné l'arbre $T = (D, l)$ donné en figure~\ref{fig:struct:quotient:ex:tree}.
    La première étape de la construction de l'arbre quotient $Q_T$ commence par la récupération des classes d'équivalences de $D$ données par la relation $R_l$.
    On obtient alors les classes suivantes :
    \begin{align*}
        C_\lambda & = \{\epsilon\} & C_X & = \{0, 1\}   & C_Y & = \{2\}  \\
        C_a       & = \{00, 20\}   & C_b & = \{01, 10\} & C_c & = \{11\}
    \end{align*}

    La construction de $Q_T$ implique la récupération de la hiérarchie des ensembles équivalents.
    L'algorithme commence par la classe $C_\lambda$ et parcours récursivement ses successeurs.
    On a, pour chaque classe, les successeurs suivants :
    \begin{align*}
        \textsf{Succ}(C_\lambda) & = \{C_X, C_Y\} & \textsf{Succ}(C_X) & = \{C_a, C_b, C_c\} & \textsf{Succ}(C_Y) & = \{C_a\}   \\
        \textsf{Succ}(C_a)       & = \emptyset    & \textsf{Succ}(C_b) & = \emptyset         & \textsf{Succ}(C_c) & = \emptyset
    \end{align*}
\end{example}

Dans l'exemple~\ref{ex:struct:quotient:1}, la figure~\ref{fig:struct:quotient:ex:tree} représente une instance des données.
L'arbre quotient permet de construire la \gls{cfg} qui reconnaît l'instance.

Avec les classes d'équivalence, la construction de l'arbre quotient implique désormais l'association de chaque classe à une position d'un arbre ordonné.
Nous introduisons la fonction~\ref{algo:struct:quotient:QDom} qui permet de créer le domaine de cet arbre quotient, en utilisant la fonction \textsf{Succ}.
Elle a pour objectif d'associer à chaque classe successeur de $C$ une position relative $u.p$ dans l'arbre quotient, où $u$ est la position de la classe $C$.

\begin{function}[htb]
    \caption{QDom($C_i$, $u$)}
    \label{algo:struct:quotient:QDom}
    \SetKwFunction{succ}{Succ}

    $S \gets \emptyset$\;
    $p \gets 0$\;
    \ForEach{$C_j \in$ \succ{$C_i$}}{
        $S \gets S \cup \{(C_j, u.p)\}$\;
        $p \gets p+1$\;
    }
    \Return{$S$}
\end{function}

\begin{remark}
    Une classe d'équivalence peut apparaître comme successeur de différentes classes.
    Dans l'exemple~\ref{ex:struct:quotient:1}, $C_a$ apparaît comme successeur de $C_Y$ et de $C_X$.
    Dans ce cas, nous aurons différentes positions associées à $C_a$.
\end{remark}

Étant donné un arbre $T$, la fonction~\ref{algo:struct:quotient:QT} construit l'arbre quotient, noté $Q_T$.
Cette construction commence par la racine qui représente la classe d'équivalence $C_{\lambda}$.
Un ensemble $classes$ contenant des couples de la forme \textit{(classe d'équivalence, position dans $Q_T$)} guide la construction (ligne~\ref{algo:struct:quotient:QT:classes}).
Ainsi, initialement, $classes$ contient le couple $(C_{\lambda}, \epsilon)$ et, par conséquent, l'arbre quotient est un nœud à la positon $\epsilon$ avec l'étiquette $\lambda$.

Ensuite, pour chaque couple $(CL, u)$ extrait de $classes$, la fonction~\ref{algo:struct:quotient:QDom} calcule l'ensemble de ses couples successeurs, l'ajoute à l'ensemble $classes$ à la ligne~\ref{QT-ligne:updClasses}.
La nouvelle position $u$ est ajoutée au domaine de $Q_T$.
Dans les lignes~\ref{algo:struct:quotient:QT:plus} à~\ref{algo:struct:quotient:QT:fin} la fonction vérifie si la classe d'équivalence $CL$ contient plusieurs positions ayant le même père.
C'est le cas, dans l'exemple~\ref{ex:struct:quotient:1}, de $C_X=\{0, 1\}$ qui contient deux positions dont le père est à la position $\epsilon$.
Dans ce cas, l'étiquette associée à la positon $u$ sera $l(u)$ décorée du symbole $+$ (ici, $X^+$).

Comme on cherche à construire une \gls{cfg} condensée, si dans un arbre de dérivation, des nœuds frères sont étiquetés par le même non-terminal, cela doit être indiqué dans l'arbre quotient de manière condensée, à l'aide du symbole ${}^+$.
Pour cela, il est nécessaire de vérifier si la classe d'équivalence d'un non-terminal $N$ possède plusieurs positions ayant le même parent.
Cette démarche implique également une \textit{fusion} des éventuelles dérivations distinctes de $N$.
Comme le non-terminal $N$ peut être associé à des règles de production différentes, en fusionnant ces dérivations, on accepte des arbres de dérivation incomplet.
L'exemple suivant illustre la construction de l'arbre quotient.

\begin{function}[htb]
    \caption{Quotient($T$ = ($D$, $l$))}
    \label{algo:struct:quotient:QT}
    \SetKwFunction{succ}{Succ}
    \SetKwFunction{qdom}{QDom}

    $Q_D \gets \{\epsilon\}$ \tcp*[l]{Initialisation du domaine le l'arbre quotient avec la position de la racine}
    $Q_l \gets \langle \epsilon \mapsto \lambda \rangle$ \tcp*[l]{L'étiquette $\lambda$ associée à la racine}

    \BlankLine
    $classes \gets \{(\{\epsilon\}, \epsilon)\}$\label{algo:struct:quotient:QT:classes} \tcp*[l]{ensemble contenant les couples classe-position qu'il reste a traiter}
    \While{$classes \neq \emptyset$\label{algo:struct:quotient:QT:while}}{
        Prendre un couple $(CL, u)$ de l'ensemble $classes$\;
        \tcp{L'ensemble est mis à jour : $classes \gets classes \setminus \{(CL, u)\}$}

        $successeurs \gets$ \qdom{$C$, $u$} \label{algo:struct:quotient:QT:newpos} \tcp*[l]{ensemble des couples classe-position qui succèdent à $CL$}
        $classes \gets classes \cup successeurs$\label{algo:struct:quotient:QT:updClasses}\;

        \BlankLine
        $Q_D = Q_D \cup \{u\}$\;
        % \tcp{Si un nœud a plusieurs successeurs on ajoute le symbole `${}^+$' au label}
        % \eIf{$\lvert\{v \mid \forall (C', v) \in successeurs$ tq. $u \prec v\}\rvert > 1$\label{algo:struct:quotient:QT:plus}}{
        %     $Q_l(u) = l(u)~+~$`${}^+$'\;
        % }{
        %     $Q_l(u) = l(u)$\;
        % }
        \If{$u \neq \epsilon$\label{algo:struct:quotient:QT:plus}}{  
        	Soit $u' \prec u$ \tcp*[l]{$u'$ est la position du père de $u$}
        	$FILS \gets \{v \mid \forall v \in CL, u' \prec v\}$ \tcp*[l]{Ensemble des positions dans $CL$ ayant $u'$ comme père}
            \eIf{$\lvert FILS \rvert> 1$}{
                $Q_l(u) =l(u)~+~$`${}^+$'\;
            }{
                $Q_l(u) = l(u)$\;
            }
        }

        \label{algo:struct:quotient:QT:fin}
    }

    \Return{$Q_T = (Q_D, Q_l)$}
\end{function}

\begin{example}%{Arbre quotient et grammaire}
    \label{ex:struct:quotient:2}
    À partir de la hiérarchie obtenue dans l'exemple~\ref{ex:struct:quotient:1}, on construit le domaine de l'arbre en associant, a chaque successeur, une position relative a son parent.
    La fonction~\ref{algo:struct:quotient:QDom} associe à chaque classe qui succèdent $CL$ une position qui représente un enfant de $u$ et permet de calculer le domaine de l'arbre quotient.

    Pour calculer les positions à l'aide de la fonction~\ref{algo:struct:quotient:QDom}, on commence à la position $\epsilon$ et on obtient $\ref{algo:struct:quotient:QDom}(C_\lambda, \epsilon) = \{(C_X, 0), (C_Y, 1)\}$.
    Après des applications successives de la fonction, le résultat est le suivant :
    \begin{align*}
        \textsf{QDom}(C_X, 0)  & = \{(C_a, 00), (C_b, 01), (C_c, 02)\} & \textsf{QDom}(C_Y, 1)  & = \{(C_a, 10)\} \\
        \textsf{QDom}(C_a, 00) & = \emptyset                           & \textsf{QDom}(C_a, 10) & = \emptyset     \\
        \textsf{QDom}(C_b, 01) & = \emptyset                           & \textsf{QDom}(C_c, 02) & = \emptyset
    \end{align*}

    En construisant le domaine de $Q_T$ il est facile de se rendre compte que le nœud correspondant à la classe $C_a$ sera dupliqué dans l'arbre, car il est lié à deux positions : $00$ et $10$.
    Avec le domaine construit par $QDom$ on peut construire l'arbre donné en figure~\ref{fig:struct:quotient:ex:quotient}.
    Dans l'arbre de la figure~\ref{fig:struct:quotient:ex:tree} le non-terminal $X$ apparaît dans l'arbre $T$ dans deux positions distinctes et indique des dérivations différentes.
    Le nœud correspondant à la classe d'équivalence $C_X$ dans $Q_T$ se voit donc attribué le symbole `${}^+$' signifiant que le symbole $X$ peut être répété plusieurs fois comme enfant de $\lambda$ (voir la définition~\ref{def:struct:pre:cfg} des \gls{cfg} condensée).
    Cette information est déduite de l'ensemble des positions de $C_X = \{0, 1\}$ (ligne~\ref{algo:struct:quotient:QT:plus} de l'algorithme~\ref{algo:struct:quotient:QT}) où deux positions partages le même préfixe (ici, $\epsilon$).
    
    En particulier, dans la fonction~\ref{algo:struct:quotient:QT}, après un premier tour de la boucle \KwSty{tant que} à la ligne~\ref{algo:struct:quotient:QT:while}, nous avons $classes= \{(C_X, 0), (C_Y,1)\}$ et nous considérons le couple $(C_X, 0)$.
    Dans ce cadre, nous aurons $successeurs = \{(C_a, 00), (C_b,01), (C_c,02)\}$ (ligne~\ref{algo:struct:quotient:QT:newpos}).
    À la ligne~\ref{algo:struct:quotient:QT:plus}, comme $u = 0$, la condition est vérifiée et $u'$ est $\epsilon$.
    Nous avons donc $FILS = \{0,1\}$, car les deux positions, $0$ et $1$, dans $C_X$ ont le même père.
    $Q_l$ est donc $\langle 0 \mapsto X^+\rangle$ car $l(0) = X$ dans $T$ et $\lvert FILS \rvert > 1$.
\end{example}

À partir de l'arbre quotient $Q_T$, il est possible de reconstruire la \gls{cfg} condensée $G_T$ qui, selon la définition~\ref{def:struct:pre:cfg}, peut contenir des non-terminaux annotés par le symbole ${}^+$, où $T$ représente une dérivation incomplète de $G_T$ (c'est-à-dire que certains symboles non-terminaux ne sont pas présents).
Cette incomplétude met en évidence un ensemble de données manquantes.
En effet, si une règle de production de $G_T$ est $X \to a~b~c$, l'arbre $T$ de la figure~\ref{fig:struct:quotient:ex:tree} sera accepté comme une dérivation valable seulement si nous considérons que $c$ ou $a$ peuvent être des valeurs manquantes.
Il est important de rappeler que les arbres enrichis proviennent de l'analyse syntaxique des textes en langue naturelle.
Le manque d'informations peut être dû à une omission ou une erreur lors de l'analyse et peut être représenté par des valeurs nulles.
La définition suivante établie de manière formelle la transformation d'un arbre quotient en grammaire.

\begin{remark}
    Nous avons examiné dans la première partie de cette thèse comment nous pouvons gérer de telles valeurs au sein d'une base de données.
    Nous reviendrons en détails sur ce point dans la conclusion.
\end{remark}

\begin{definition}[Grammaire quotient]
    Étant donné un arbre $T$ et son quotient $Q_T = (Q_D, Q_l)$, la \gls{cfg} condensée $G_T = (N, T, P, \lambda)$ qui reconnaît $T$ est obtenue via $Q_T$ où :
    \begin{itemize}
        \item $\lambda$ est le symbole de départ ;
        \item l'ensemble de non-terminaux $N$, éventuellement décorés par ${}^+$, est l'ensemble d'étiquettes $Q_l(u)$ pour toute position $u \in Q_D$ qui n'est pas une feuille ;
        \item l'ensemble de terminaux $T$ est l'ensemble d'étiquettes $Q_l(u)$ pour toute position $u \in Q_D$ qui est une feuille ;
        \item l'ensemble $P$ de règles de production contient, pour toute position $u \in Q_D$ qui n'est pas une feuille, les règles de la forme $Q_l(u) \to Q_l(u.0), \dots Q_l(u.i)$.
    \end{itemize}
\end{definition}

\begin{example}
    \label{ex:struct:quotient:3}
    Etant donné l'arbre $Q_T$ de l'exemple~\ref{ex:struct:quotient:2} donné en figure~\ref{fig:struct:quotient:ex:quotient}, on obtient la grammaire $G_T$ suivante :
    \begin{align*}
        \lambda & \to X^+ ~ Y & X & \to a ~ b ~ c & Y & \to a
    \end{align*}
\end{example}

\subsection{Construction des classes d'équivalence}
\label{sec:struct:equiv-classes}
Rappelons maintenant le concept d'arbre d'entités donné par la définition~\ref{def:struct:entity-tree} (page~\pageref{def:struct:entity-tree}).
Les arbres d'entités représentent l'information et sont donc la base de comparaison des sous-arbres.
La représentation textuelle d'un objet du monde réel peut avoir plusieurs formes de surface matérialisée par différents arbres d'entité.
De plus, la langue naturelle contient souvent de l'information implicite ou manquante.
Ainsi, deux sous-arbres peuvent contenir un ensemble différent d'arbres d'entités, mais quand même représenter le même objet.
Par exemple, un \emph{traitement} ne contient pas toujours une fréquence ou un dosage (\textquote{Le patient prend 500 mg de Paracétamol} or \textquote{Le patient prend du Paracétamol tous les jours}).
Les arbres syntaxiques de ces phrases ne seront pas identiques.
Néanmoins, les deux représentent un \emph{traitement}.
Dans notre approche, nous allons considérer qu'ils sont équivalents.

Néanmoins, pour déterminer l'équivalence entre deux sous-arbres, considérer uniquement l'ensemble des arbres d'entités présent dans chacun des sous-arbres les arbres d'entités n'est pas suffisant.
La langue naturelle est ambiguë et une même forme de surface (et donc un même arbre d'entité) peux représenter deux objets différents.
Il est donc nécessaire de prendre aussi en compte le contexte.
Pour ce faire nous utilisons le concept d'équivalence régulière comme introduite par~\cite{whiteGraphSemigroupHomomorphisms1983} pour les graphes.
L'idée est que deux sommets d'un graphe sont jugés équivalents si leur voisinage est équivalent.
Par exemple, deux personnes peuvent être jugées \emph{équivalentes} (comme catégorie d'objet représenté, par exemple un patient) si elles sont toutes les deux liées à des sommets \emph{équivalents} comme une maladie, un traitement, etc sans pour autant que ces nœuds soient les mêmes.

Afin de définir la relation d'équivalence, nous définissons une mesure de similarité entre deux sous-arbres.
On note $\mathbb{T}$ l'ensemble de tous les arbres et $\mathbb{ST}$ l'ensemble de tous les sous-arbres.
Une mesure de similarité est une fonction symétrique $f : \mathbb{ST} \times \mathbb{ST} \to [0,1]$ où $f(x, x) = 1$ pour tout $x \in \mathbb{ST}$.
Plusieurs mesures $f$ de similarité peuvent être utilisés comme Jaccard, Levenshtein, Jaro, ou encore la distance d'edition pour les arbres \cite{zhangSimpleFastAlgorithms1989}.
La similarité contextuelle d'arbres est notée $sim_f(x, y)$, entre deux sous-arbres enrichis $x = T|_u$ et $y = T'|v$, est définie comme une moyenne pondérée des similarités donnée récursivement par la fonction $f$ pour chaque sous-arbre parent.
Les poids sont inversement proportionnels à la distance entre les sous-arbres parents et les sous-arbres $x$ et $y$.
L'idée sous-jacente est que plus un sous-arbre est distant d'un autre, moins il a de valeur dans son contexte.
La mesure $sim_f(x, y)$ est donnée par l'équation~\ref{eq:struct:sim} où $P^x_i$ (respectivement $P^y_i$) est le $i$-éme sous-arbre parent de $x$ (respectivement $y$) et $depth_{min}$ est la profondeur minimum entre les sous-arbres $x$ et $y$ tel que $depth_{min} = \min\{|u|, |v|\}$.

\begin{equation}
    sim_f(x, y) = \frac{\sum_{i=0}^{depth_{min}} \frac{1}{i + 1} \cdot f(P^x_i, P^y_i)}{\sum_{j=0}^{depth_{min}} \frac{1}{j + 1}} \label{eq:struct:sim}
\end{equation}

\begin{axiom}
    Comme la fonction $sim_f$ est une moyenne pondérée de $f$ elle respecte alors ses propriétés :
    $sim_f$ est symétrique, bornée par l'intervalle $[0, 1]$ et pour tout $x \in \mathbb{ST}$, $sim_f(x, x) = 1$.
\end{axiom}

\begin{example}
    Soit l'arbre donné figure~\ref{fig:struct:sim:ex}.
    On commence par definir une fonction de similaritée.
    Dans cette exemple on utilisera simplement l'indice de Jaccard \cite{jaccardDistributionFloreAlpine1901} sur les nom d'entité present dans le sous-arbre.
    L'indice de Jaccard est definit comme suit :
    \[
        J(X, Y) = \frac{\lvert X \cap Y \rvert}{\lvert X \cup Y \rvert}
    \]
    Il est facile de voir que les deux sous-arbres ER\_NODE$_1$ et ER\_NODE$_2$ sont identique on obtient alors $f(\text{ER\_NODE}_1, \text{ER\_NODE}_2) = 1$.
    Cependant, comme leur contexte est différent : l'un est associé à un médicament (Paracetamol) et l'autre à une frequence (every day) on s'attend à une similarité différente de $1$.
    La fonction $sim_f$ (équation~\ref{eq:struct:sim}) permet de calculer cette différence.
    On commence par calculer la similarité entre les sous-arbres NP$_1$ et NP$_2$ et on obtient $f(\text{NP}_1, \text{NP}_2) = 0.75$.
    Puis ont continue recursivement jusqu'à atteindre la racine.
    A partir du noeud CONJ, les sous-arbres sont identiques avec une similarité de $1$.
    On obtient alors :
    \[
        sim_f(\text{ER\_NODE}_1, \text{ER\_NODE}_2) = \frac{1 \times 1 + 0.5 \times 0.75 + 0.33 \times 1 + 0.25 \times 1}{1 + 0.5 + 0.33 + 0.25} = \frac{1.955}{2.08} \simeq 0.94
    \]

    Ce qui permet de montrer que les deux sous-arbres sont proches sans pour autant etre equivalents car il ne sont pas mis en relation avec les mêmes objets.

    % Dans cet exemple, on utilise la distance de Jaccard noté $J(X, Y)$ où $X$ et $Y$ sont deux liste triée de nom d'entitiés pouvant contenir des redondances.
    % La distance de Jaccard est definie comme suit :
    % \[
    %     J(X, Y) = \frac{\lvert X \cap Y \rvert}{\lvert X \cup Y \rvert}
    % \]
    % Où $\lvert X \cap Y \rvert$ est la concatenation des séquances minimale pour chaque nom d'entité, par exemple $[A, A, B] \cap [A, B, C] = [A, B]$.
    % $\lvert X \cup Y \rvert$ est la concatenation des séquances maximale pour chaque nom d'entité, par exemple $[A, A, B] \cup [A, B, C] = [A, A, B, C]$.

    \begin{figure}[H]
        \centering
        \begin{adjustbox}{valign=c, max width=.9\textwidth}
            \begin{forest}
                where n children=0{tier=word}{}
                [S
                    [\textbf{ENT\_PERSON}
                        [John Smith]
                    ]
                    [CONJ
                            [NP$_1$
                                [\textbf{ER\_NODE}$_1$
                                        [\textbf{ENT\_DOSAGE} [500 mg]]
                                            [\textbf{EC\_NODE}
                                                [\textbf{ENT\_VALUE} [500]]
                                                [\textbf{ENT\_UNIT} [mg]]
                                            ]
                                    ]
                                    [\textbf{ENT\_DRUG} [Paracetamol]]
                            ]
                            [NP$_2$
                                [\textbf{ER\_NODE}$_2$
                                        [\textbf{ENT\_DOSAGE} [200 mg]]
                                            [\textbf{EC\_NODE}
                                                [\textbf{ENT\_VALUE} [200]]
                                                [\textbf{ENT\_UNIT} [mg]]
                                            ]
                                    ]
                                    [\textbf{ENT\_FREQ} [every day]]
                            ]
                    ]
                ]
            \end{forest}
        \end{adjustbox}
        \caption{Exemple d'arbre enrichi}
        \label{fig:struct:sim:ex}
    \end{figure}
\end{example}

Avec cette mesure de similarité, il est possible de définir la relation de similarité entre deux sous-arbres enrichis.
On dira alors que deux sous-arbres sont similaires si leur similarité est supérieure à un taux donné en paramètre.
La décision de ce taux n'est pas arbitraire et il doit être déterminé pour chaque ensemble de données.

\begin{definition}[Similarité de sous-arbres]
    \label{def:struct:sim}
    Soient deux arbres enrichis $T_1$ et $T_2$ avec deux sous-arbres respectifs $st_1 = T_1|_u$ et $st_2 = T_2|_v$ et un seuil noté $\tau \in [0, 1]$, on dit que $st_1$ et $st_2$ sont $\tau$-similaires, noté $st_1 \sim_\tau st_2$, si et seulement si $sim_f(st_1, st_2) \geq \tau$.
\end{definition}

\begin{proposition}
    La $\tau$-similarité est une relation de similarité \emph{réflexive} et \emph{symétrique}.
\end{proposition}

\begin{proof}
    Pour tout $x$, on a $sim_f(x, x) = 1$ et $1 \ge \tau$ pour toute valeur possible de $\tau$, alors, on peut déduire que la relation de $\tau$-similarité est \emph{réflexive}.
    De plus, si on a $x \sim_\tau y$, alors $f(x, y) \ge \tau$.
    Comme la fonction $sim_f$ est symétrique ($f(x, y) = f(y, x)$) on a donc $f(y, x) \ge \tau$ et par extension $y \sim_\tau x$.
    Nous pouvons conclure que la relation de $\tau$-similarité est \emph{symétrique}.
\end{proof}

\begin{definition}[Equivalence de sous-arbres]
    \label{def:struct:equiv}
    Étant donné deux arbres enrichis $T_1$ et $T_2$ avec deux sous-arbres respectifs $st_1 = T_1|_u$ et $st_2 = T_2|_v$ et la relation de $\tau$-similarité définit précédemment (\ref{def:struct:sim}).
    On définit la relation de $\tau$-équivalence entre $st_1$ et $st_2$ notée $st_1 \equiv_\tau st_2$ tel que $st_1$ est $\tau$-équivalent à $st_2$ si $st_1$ est $\tau$-similaires à $st_2$ ou si $st_1$ est  $\tau$-équivalent à un sous-arbre $st_3$ lui même $\tau$-équivalent à $st_2$.
    La proposition est synthétisée par l'équation~\ref{eq:struct:equiv}.
    \begin{equation}
        (\forall x \in \mathbb{ST}, y \in \mathbb{ST}) ~ x \equiv_\tau y \iff x \sim_\tau y \lor (\exists z \in \mathbb{ST}) ~ x \equiv_\tau z \land y \equiv_\tau z \label{eq:struct:equiv}
    \end{equation}
\end{definition}

\begin{proposition}
    La $\tau$-équivalence est une relation d'équivalence et est par conséquent \emph{réflexive}, \emph{symétrique} et \emph{transitive}.
\end{proposition}

\begin{proof}
    Soit $st_1$ un sous-arbre enrichi.
    La réflexivité de la $\tau$-équivalence découle directement de la réflexivité de celle de la $\tau$-similarité, car $st_1 \equiv_\tau st_1$ implique $st_1 \sim_\tau st_1$ qui est toujours vrais.
    $st_1 \equiv_\tau z \land st_1 \equiv_\tau z$ est vrai pour $z = st_1$ étant donné que $st_1 \equiv_\tau st_1$ par hypothèse.
    La $\tau$-équivalence est donc \emph{réflexive}.

    Soient $st_1$ et $st_2$ deux sous-arbres enrichis et supposons que $st_1 \equiv_\tau st_2$.
    On a alors deux cas :
    \begin{itemize}
        \item $st_1 \sim_\tau st_2$ : qui implique, par symétrie, $st_2 \sim_\tau st_1$ et donc $st_2 \equiv_\tau st_1$.
        \item $\exists z (st_1 \equiv_\tau z \land st_2 \equiv_\tau z)$ : qui peut se réécrire en $\exists z (st_2 \equiv_\tau z \land st_1 \equiv_\tau z)$ et qui donne, par définition, $st_2 \equiv_\tau st_1$.
    \end{itemize}
    On peut donc en déduire que la $\tau$-équivalence est \emph{symétrique}.

    Soient $st_1$, $st_2$ et $st_3$ trois sous-arbres enrichis et supposons que $st_1 \equiv_\tau st_2$ et $st_2 \equiv_\tau st_3$.
    Par symétrie, nous avons $st_1 \equiv_\tau st_2$ et $st_3 \equiv_\tau st_2$.
    En utilisant la définition de la $\tau$-équivalence (\ref{eq:struct:equiv}), nous pouvons remplacer $z$ par $st_2$.
    On obtient alors, par définition, $st_1 \equiv_\tau st_3$.
    On peut conclure que $(st_1 \equiv_\tau st_2 \land st_2 \equiv_\tau st_3) \implies st_1 \equiv_\tau st_3$.
    La $\tau$-équivalence est donc \emph{transitive}.

    La $\tau$-équivalence respecte les trois propriétés d'une relation d'équivalence : \emph{réflexive}, \emph{symétrique} et \emph{transitive}.
    % Supposons que $st_1 \equiv_\tau st_2$, on a donc $st_1 \sim_\tau st_2$ ou il existe un $z$ qui vérifie $st_1 \equiv_\tau z \implies st_2 \equiv_\tau z$.
    % Dans le premier cas on peut déduire que $st_2 \sim_\tau st_1$ par réflectivité et par conséquent $st_2 \equiv_\tau st_1$.
    % Dans le second cas, si on prend $z = st_1$, on obtient $st_1 \equiv_\tau st_1 \implies st_2 \sim_\tau st_1$.
    % Or, $st_1 \equiv_\tau st_1$ est vrai par réflexivité et on obtient alors $st_2 \equiv_\tau st_1$
    % On peut donc en déduire que $st_1 \equiv_\tau st_2 = st_2 \equiv_\tau st_1$ ce qui implique que la $\tau$-équivalence est \emph{symétrique}.
    % Prenons l'hypothèse que $st_1 \equiv_\tau st_2$ et $st_2 \equiv_\tau st_3$.
    % Cela signifie que $st_2 \sim_\tau st_3$ ou $st_2 \equiv_\tau z \implies st_3 \equiv_\tau z$.
    % Or, comme $st_2 \equiv_\tau st_1$ par symétrie, en prenant $z = st_1$, on a $st_3 \equiv_\tau st_1$ et donc $st_1 \equiv_\tau st_3$ par symétrie.
    % La $\tau$-équivalence est donc \emph{transitive}.
\end{proof}

\begin{definition}[Classe d'équivalences]
    On note $[x]_\tau$ l'ensemble $\tau$-équivalent de $x$ (partition ou classe d'équivalence) donné par la clôture transitive de $\equiv_\tau$ tel que $y \in [x]_\tau \iff y \equiv_\tau x$.
    $S/\equiv_\tau$ est l'ensemble quotient (ou partitionnement) de $S$ par $\equiv_\tau$ tel que $S/\equiv_\tau = \{[x] \mid x \in S\}$, c-à-d. que $S/\equiv_\tau$ est l'ensemble de tous les ensembles $\tau$-équivalent de $S$.
\end{definition}

Le partitionnement d'un ensemble, basé sur une distance, peut être calculé à l'aide d'un regroupement hiérarchique en lien simple et la mesure de similarité $sim_f$.
\cite{carlssonCharacterizationStabilityConvergence2010} montre qu'un regroupement hiérarchique en lien simple correspond a un partitionnement selon une relation d'équivalence.
Le regroupement hiérarchique implique la construction d'une hiérarchie des éléments selon une mesure de similarité.
Initialement, chaque individu forme une classe distincte.
À chaque étape, les classes \emph{proches} sont fusionnées deux à deux en fonction de la mesure de similarité (ou de disimilarité).
Une hiérarchie est qualifiée \emph{en lien simple} lorsque la similarité entre deux classes est évaluée comme étant le maximum de la similarité (ou le minimum de la dissimilarité) entre chaque paire d'éléments de ces classes respectives.
Cette hiérarchie est fréquemment représentée graphiquement à l'aide d'un dendrogramme (cf. figure~\ref{fig:struct:quotient:dendrogramme}).
Il s'agit d'un arbre binaire décrivant les fusions de classes, où les classes sont disposées sur l'axe des abscisses et l'axe des ordonnées représente la dissimilarité entre les classes (plus la valeur est élevée, plus les classes sont éloignées).

Une classification par un taux $\tau$ consiste à fusionner des classes tant que la similarité demeure supérieure à $\tau$.
Visuellement, il est possible de tracer une droite sur le dendrogramme et de conserver toutes les classes situées en dessous de cette droite.
Dans la figure~\ref{fig:struct:quotient:dendrogramme}, en fixant $\tau = 0.3$, deux classes émergent, à savoir $\{a, b\}$ (en bleu) et $\{c, d, e\}$ (en orange).
Si le seuil est augmenté à $\tau = 0.5$, trois classes distinctes apparaissent, à savoir $\{a\}$, $\{b\}$ et $\{c, d, e\}$.
En continuant d'augmenter le seuil, par exemple à $\tau = 0.7$, quatre classes distinctes seraient obtenues, à savoir $\{a\}$, $\{b\}$, $\{c, d\}$ et $\{e\}$.

\begin{figure}[H]
    \centering
    \begin{adjustbox}{max width=\linewidth}
        \begin{tikzpicture}[sloped]
            \node (a) at (-6,0) {a};
            \node (b) at (-3,0) {b};
            \node (c) at (-0.5,0) {c};
            \node (d) at (0.5,0) {d};
            \node (e) at (2,0) {e};
            \node (ab) at (-4.5,3) {};
            \node (cd) at (0,1) {};
            \node (cde) at (1,2) {};
            \node (all) at (-1.5,4.5) {};

            \draw[thick, blue] (a) |- (ab.center);
            \draw[thick, blue] (b) |- (ab.center);
            \draw[thick, orange] (c) |- (cd.center);
            \draw[thick, orange] (d) |- (cd.center);
            \draw[thick, orange] (e) |- (cde.center);
            \draw[thick, orange] (cd.center) |- (cde.center);
            \draw[thick] (ab.center) |- (all.center);
            \draw[thick] (cde.center) |- (all.center);

            \draw[<-,latex-] (-10,0) -- node[above]{simlaritée} (-10,5);
            \draw[->,-latex] (-9,0) -- node[above]{disimilarité} (-9,5);

            \draw (-7,0) -- (-7,5);

            \foreach \y in {0,1,2,3,4,5}
            \draw[shift={(0,\y)},color=black] (-7,0) -- (-7.1,0);

            \node[left] at (-7.1,0) {$1.0$} ;
            \node[left] at (-7.1,1) {$0.8$} ;
            \node[left] at (-7.1,2) {$0.6$} ;
            \node[left] at (-7.1,3) {$0.4$} ;
            \node[left] at (-7.1,4) {$0.2$} ;
            \node[left] at (-7.1,5) {$0.0$} ;

            \draw[dashed,color=gray] (-7,3.5) -- node[at end, right]{$\tau$} (2.5,3.5);
            \draw[color=gray] (-7,3.5) -- (-7.1,3.5) node[left] {$0.3$};
        \end{tikzpicture}
    \end{adjustbox}
    \caption{Exemple de dendrogramme pour le partitionnement hiérarchique}
    \label{fig:struct:quotient:dendrogramme}
\end{figure}

\subsection{Réécriture des arbres}
\label{sec:struct:auto:rewrite}
La dernière étape du processus de structuration consiste à réécrire les arbres d'instances.
L'objectif est de réécrire l'arbre d'instances de façons à ce que l'instance soit reconnues par un schéma valide selon la méta-grammaire $\mathbb{G}$.
Pour cela, on cherche à identifier les nœuds correspondants à des \emph{groupes}, des \emph{relations} ou les \emph{collections}.
L'ensemble des sous-arbres équivalents peuvent être réécrits pour correspondre à la structure associée à leur catégorie.
Cette réécriture est un processus itératif ou à chaque modification de l'arbre, on vérifie si ce dernier produit un schéma valide ou non.
Comme il n'existe pas de schéma unique, la procédure de réécriture dans cette étape peut varier en fonction des objectifs définis.
Une proposition d'objectifs et une implémentation des procédures correspondantes sont données dans la section suivante.

\FloatBarrier
