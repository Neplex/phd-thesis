L'explosion de la production de données textuelles, alimentée par la numérisation croissante de notre monde, a atteint des proportions phénoménales ces dernières années.
Cette abondance de données textuelles représente une mine d'informations précieuses pour la prise de décision, la recherche et le développement, mais leur véritable valeur réside dans la capacité à les traiter efficacement.
Le traitement de ces données textuelles est essentiel pour extraire des informations significatives, détecter des tendances, automatiser des tâches, et prendre des décisions éclairées, ce qui en fait un enjeu majeur dans le domaine de l'intelligence artificielle et de l'analyse de données.
C'est notamment le cas dans le domaine médical avec les résumés de cas cliniques ou la pharmacovigilance.

Dans ce contexte, le \gls{tal} est un domaine qui joue un rôle fondamental en utilisant des approches syntaxiques, sémantiques ou a base d'apprentissages, pour extraire de l'information dans des textes.
Cela regroupe notamment : la reconnaissance d'entités nommées, l'extraction de relations, les liaisons d'entités nommées avec des individus d'une base de données, l'identification de la coréférence, de la temporalité et de la causalité.
Bien que tout cela constitue une étape majeure, ce n'est pas suffisant pour structurer l'information afin de l'exploiter dans des bases de données traditionnelles.
La structuration des données textuelles présente plusieurs avantages, notamment l'amélioration des capacités de recherche, la facilitation de l'analyse des données, l'amélioration de la qualité des données grâce à une vérification de cohérence, l'intégration et la réutilisation des informations, la promotion de la collaboration au sein des équipes et la facilitation de la visualisation des données à l'aide de graphiques ou de diagrammes.

Il existe plusieurs approches dans littératures qui tentent de répondre à cette problématique.
WORD GRAPH
SEM GRAPH
