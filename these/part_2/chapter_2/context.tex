Une \emph{entité enrichie} est une entité \emph{simple} à laquelle on a ajouté de l'information.
Il peut s'agir de préciser le type, la valeur ou de faire le lien avec un individu dans une base de données.
L'enrichissement est opéré généralement en deux étapes, la première où l'on identifie les entités simples et les marqueurs de contextes et la seconde où l'on identifie le lien entre les entités et les marqueurs.
On identifie les catégories d'entités suivantes :
\begin{description}
    \item[Une entité de référence] est celle qui est choisie pour être enrichie.
          Elle correspond à une valeur d'instance dans une base de données (nom de personnes, titre d'un documents, date, etc.)
    \item[Un opérateur] ajoute des précisions sur la valeur de l'entité.
          Les opérateurs sont des marqueurs de comparaisons \textquote{commence par}, \textquote{contient}, etc ou de négation.
    \item[Un marqueur de contexte] est une entité qui porte l'information de contexte.
\end{description}

\begin{definition}[Entité enrichie]
      Une entité enrichie est définie comme un ensemble d'entités mono-valuées associées à un opérateur tel que :
      $E_e$ est une entité enrichie de la forme $\{(v, \mathcal{T}, op)\}$ où $v$ est une unique valeur d'entité, $\mathcal{T}$ est un ensemble de types associé à la valeur et $op$ est un opérateur de comparaison sur la valeur (par défaut $=$).
      Une entité multi-valuée $E = (\mathcal{V}, \mathcal{T}, m)$ se traduit par une entité enrichie $E_e = \{(v, m(v), =) \mid \forall v \in \mathcal{V}\}$.
\end{definition}

Pour une même valeur, l'ensemble des types associés s'appliquent.
L'ambiguïté n'est représenté que par l'ensemble de tuples dans l'entité enrichie.
Ainsi, l'entité $E_e = \{ (\text{:alice}, \{Person\}, =),$ $(\text{:bob}, \{Person\}, =) \}$ possède une ambiguïté sur sa valeur (\emph{:alice} ou \emph{bob}).
Il est aussi possible de représenter l'ambiguïté sur le type, par exemple, l'entité $E_e = \{ (\text{:paris}, \{Person\}, =),$ $(\text{:paris}, \{City\}, =) \}$ à une ambiguïté sur le type de \emph{Paris} qui peut être une personne ou une ville.

\paragraph{}
Dans la phrase \textquote{écrit par Alice}, \emph{Alice} pourrait représenter une entité de référence.
Elle aurait le type \emph{Personne} et pour valeur, un (ou plusieurs) identifiant dans une base de données.
Ici, \textquote{écrit} pourrait être le marqueur du contexte \emph{Auteur}.
L'enrichissement consisterait alors à ajouter le type correspondant au contexte (ici, \emph{Auteur}) à l'entité de référence (ici, \emph{Alice}).
Cette approche itérative permet de découper le processus d'extraction de l'information.
Dans le cas ou plusieurs valeurs sont possibles pour l'entité \emph{Alice} (par exemple si plusieurs individus de la base partagent le même prénom), l'identification de type plus précis peut permettre de réduire l'ambiguïté.

Un autre facteur important dans les textes sont les conjonctions de coordinations, en particulier les conjonctions \textquote{ou} et \textquote{et}.
Par exemple, dans la phrase \textquote{écrit par Alice et Bob}, la conjonction de coordination \textquote{et} est utilisée pour indiquer que le marqueur du contexte \emph{Auteur} s'applique à la fois à l'entité \emph{Alice} et a l'entité \emph{Bob}.
On obtient alors deux entités distinctes ayant le type \emph{Auteur}.
Si on part de la phrase \textquote{écrit par Alice ou Bob}, la conjonction \textquote{ou} est vu comme un marqueur d'ambiguïté.
On obtient alors une seule entité regroupant les valeurs de l'entité \emph{Alice} et de l'entité \emph{Bob}.

Pour réaliser la contextualisation des entités, nous proposons deux approches :
\begin{itemize}
      \item Une première approche basée sur des règles ;
      \item Une seconde approche basée sur l'idée de la cascade de \glspl{crf} présenté dans la section~\ref{sec:tal:crf}.
\end{itemize}

\subsubsection{Utilisation de règles}
\label{sec:tal:ctx:rule}

En analysant les textes avec lesquels on travaille, on remarque que le contexte est généralement porté par un mot précis et principalement le verbe.
Intuitivement, l'approche basique consiste alors à identifier le marqueur de contexte comme une entité.
Pour cela on utilise les mêmes approches que présenté précédemment, en particulier l'extraction par lexique.
Une entité contexte est associée au type \emph{Contexte} et a pour valeur, l'ensemble de types qu'il représente.
Une fois le marqueur identifié, il faut déterminer sa portée, c'est-à-dire les entités qui sont impactées par ce dernier.
Pour ce faire, on utilise l'arbre de dépendance et on identifie les situations suivantes :
\begin{enumerate}
    \item Il existe un lien direct, vers la droite entre le contexte et l'entité de référence (par exemple si l'entité est l'objet du verbe) ;

    \item Il y a un verbe entre le contexte et l'entité de référence.
          Dans ce cas, le contexte est le sujet et l'objet est l'entité de référence.
          Par exemple, dans la phrase \textquote{l'auteur est Alice}.

    \item L'entité de référence est liée par une conjonction de coordination à un mot qui est lui-même dans la portée du contexte. \label{ctx:rule:1}

    \item Si une entité de référence se trouve dans la portée d'un contexte, elle hérite aussi de tous les contextes conjoints. \label{ctx:rule:2}
\end{enumerate}

On remarque avec les règles~\ref{ctx:rule:1} et \ref{ctx:rule:2} que la portée n'est pas toujours triviale et on se rend compte que la détection de la portée est plus compliquée dès lors qu'on ajoute de multiples contextes, des conjonctions de coordinations.
De plus, cet ensemble de règles ne couvre pas toutes les structures possibles.
Par exemple, dans la phrase \textquote{écrit par Alice et signé ou relus par Bob avant le 10 mars}, on s'attend à avoir les entités $E_1 = (\text{:alice}, \{Auteur\})$, $E_2 = (\text{:bob}, \{Signataire, Relecteur\})$ et $E_3 = (10/03, \{DateSignature, DateRelecture\})$.
Cependant, la règle~\ref{ctx:rule:1} ne permet pas d'associer les contextes \emph{DateSignature} et \emph{DateRelecture} à l'entité date à cause de sa distance avec le marqueur de contexte \textquote{signé ou relus}.

\subsubsection{Cascade de \gls*{crf}}
\label{sec:tal:ctx:crf}

Une autre approche s'inspire de la cascade de \glspl{crf}.
Nous avons vu dans la section~\ref{sec:tal:crf}, que les \glspl{crf} peuvent être utilisés \emph{successivement} en utilisant comme attribut des labels d'entités des étapes précédentes pour extraire des entités plus larges.
Un \gls{crf} est utilisé pour annoter une portion du texte.
L'identification de la portée d'un contexte correspond aussi à détecter la portion du texte impacté.
L'idée est donc de construire un \gls{crf} utilisant comme attributs les entités présentes dans le texte pour identifier la portée des contextes.
Une entité de référence est ensuite associée à un contexte si elle se situe dans la séquence annotée.
Comme le contexte est majoritairement porté par un verbe, on ajoute comme attribut le lemme du premier verbe parent dans l'arbre de dépendance.
Comme il n'est pas possible d'annoter un mot avec différent label quand on utilise un \gls{crf}, il est nécessaire de traiter chaque contexte de façon indépendante.
Cependant, cela ne permet pas de traiter le cas de la conjonction \textquote{ou}.
