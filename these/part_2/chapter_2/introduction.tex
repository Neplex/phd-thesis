L'extraction ou la reconnaissance d'entités nommées, \gls{ner} en anglais, est une tâche commune du \gls{tal} qui consiste à identifier la partie d'un texte non structuré qui représente une entité nommée.
Cette tâche peut être réalisée soit par des techniques basées sur des grammaires, soit par un modèle statistique tel que l'apprentissage automatique (voir \cite{jurafskySpeechLanguageProcessing2009} pour une introduction complète dans ce domaine).
Les approches statistiques sont largement utilisées dans l'industrie, car elles offrent de bons résultats.
Cependant, ces approches nécessitent beaucoup de données pour leur apprentissage, ce qui implique des coûts élevés pour leur récupération, leur annotation et l'apprentissage.
De plus, il n'est pas toujours possible d'avoir accès à de grands volumes de données, en particulier pour le domaine médical à cause des données sensibles qu'ils peuvent contenir.
Les méthodes plus conventionnelles basées sur la grammaire sont très utiles pour traiter de petits ensembles de données.

\paragraph{Extraction itérative d'entités}
L'extraction d'entités peut représenter une tâche complexe en raison de la diversité des types d'entités, de leurs valeurs, de leurs tailles ou de leur imbrication.
Dans cette thèse, nous proposons de décomposer le processus d'extraction en sous-tâches plus simples, permettant d'exploiter une combinaison de diverses méthodes.
Initialement, on se focalise sur l'extraction d'informations basiques à l'aide de grammaires ou de lexiques.
Puis, progressivement, au travers d'une série d'étapes, on s'attelle à la combinaison, la correction, le filtrage et l'enrichissement de ces entités élémentaires afin de parvenir à des entités enrichie sémantiquement.
Ainsi, dans la phrase \textquote{Marie est entrée à l'hôpital}, on commencera par déterminer que \emph{Marie} est une personne avant d'ajouter l'information plus précise que \emph{Marie} est une patiente.
Cette approche permet de tirer parti d'outils génériques pour l'analyse syntaxique et l'extraction d'entités simples, tels que l'extraction de nombres ou de dates.
Dans la suite, il est possible d'appliquer des techniques d'apprentissage, en se basant sur les entités extraites plutôt que sur le texte brut.
Cela permet de construire un processus d'extraction ne nécessitant pas l'accès à de larges ensembles de données du domaine étudié.

\paragraph{Organisation}
Ce chapitre commence par une description (section~\ref{sec:tal:syntax}) du corpus étudié dans cette thèse.
La section~\ref{sec:tal:entity} présente les techniques mise en œuvres pour la reconnaissance d'entités nommées et la section~\ref{sec:tal:ctx} présente la notion de contextualisation des entités.
Enfin, le chapitre se termine sur la présentation de deux cas d'usage direct des techniques présentées dans ce chapitre :
\begin{itemize}
    \item la classification de cas cliniques (section~\ref{sec:tal:classification}) et
    \item les interface d'interrogation en langue naturelle (section~\ref{sec:tal:nl-query}).
\end{itemize}
La section~\ref{sec:tal:conclusion} conclue ce chapitre.
