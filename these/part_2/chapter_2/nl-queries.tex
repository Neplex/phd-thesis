Les systèmes d'interrogation de bases de donnés graphes visant les \emph{experts du domaine} plutôt que les experts dans des bases de données, suscitent un intérêt grandissant, car ce sont ces utilisateurs auxquels profite le plus l'information et qui sont les plus aptes à la transformer en connaissance.
Il est donc crucial de rendre ces systèmes plus conviviaux et accessibles aux utilisateurs sans compétences techniques avancées.
Malgré la puissance des langages d'interrogation comme \gls{sql} \gls{sparql} ou encore \gls{cypher}, leur utilisation nécessite une compréhension de la structure des données et de leurs interactions.
Afin d'améliorer l'accessibilité à l'information sauvegardée dans ces \gls{sgbd}, la recherche sur les \gls{nli} a suscité beaucoup d'intérêt ces dernières années \cite{zhengQuestionAnsweringKnowledge2018,wangCrossdomainNaturalLanguage2019,xuMirrorNaturalLanguage2023,vargas-solarTranslatingDataScience2023,vargas-solarConversationalDataExploration2023}.
L'intérêt fondamental des interfaces en langage naturel réside dans la capacité à permettre aux utilisateurs de se concentrer sur la signification de leurs requêtes, plutôt que sur les mécanismes sous-jacents de recherche d'information.

Dans ce contexte, et en s'appuyant sur les méthodes de reconnaissances d'informations textuelles présentées dans les sections précédentes, on s'intéresse aux requêtes factuelles sur les instances d'une \emph{classe} \gls{rdf} avec deux objectifs principaux :
\begin{enumerate*}[label=(\roman*)]
    \item la facilité d'appliquer le système à divers domaines et
    \item la création de concepts d'interrogation plus complexes.
\end{enumerate*}
L'objectif est de concevoir un système d'interrogation permettant aux utilisateurs d'effectuer des recherches par facettes.
La recherche par facette consiste à filtrer une collection de données basée sur une unique collection d'individus par l'application de différents critères.
Il s'agit d'une requête conjonctive sur une unique classe appelée \emph{classe cible}.

Dans cette section, on présente la transformer d'une requête exprimée en langage naturel (nommée NL-query) en une requête de base de données (dénommée DB-query) \cite{amaviNaturalLanguageQuerying2020} et repose sur le formalisme établi dans la section~\ref{sec:update:pre:db}, page~\pageref{sec:update:pre:db} basé sur \gls{fol}.
En conséquence, ces requêtes peuvent être facilement traduites vers une modélisation graphe ou relationnelle en utilisant les langages de requêtes tels que \gls{sql}, \gls{sparql}, \gls{cypher}, etc.
On définit une formule logique $\phi$ ayant pour atomes :
\begin{enumerate}
    \item $P(t_1, \dots, t_n)$ où $P$ est un symbole de prédicat d'arité $n$ et $t_1, \dots, t_n$ sont des termes pouvant être des constantes ou des variables ;
    \item $\top$ (Vrai) / $\bot$ (Faux) ;
    \item $(t\ op\ \alpha)$ où $t$ est un terme, $\alpha$ est un terme ou un littéral et $op$ est un opérateur de comparaison.
\end{enumerate}
On rappelle qu'un fait est un atome $P(t_1, \dots, t_n)$ où l'ensemble des termes $t_1, \dots, t_n$ sont des constantes.
Par exemple, en \gls{rdf}, le fait $Book(Anatomy)$ signifie que $Anatomy$ est une instance de la classe \emph{Book}.
De manière similaire, $writtenBy(Anatomy, Bob)$, exprime que $Anatomy$ possède la valeur $Bob$ pour la propriété \textit{writtenBy}.
Le schéma de la base de données est représenté par un ensemble de prédicats $\mathcal{S}$ et l'instance de la base de données est représenté par l'ensemble de faits $\mathcal{D}$

\begin{definition}[NL-query]
    Une NL-query est une requête exprimée en langage naturel qui suit le format suivant : \textquote{Trouver des livres qui \dots} (en établissant \textbf{Book} comme \emph{classe cible}), \textquote{Quels médecins \dots} (en utilisant \textbf{Doctor} comme \emph{classe cible}), etc.
    Une requête valide sélectionne exclusivement des instances de  \emph{classe cible} (c'est-à-dire des nœuds du type donné) par l'intermédiaire de propriétés dont le domaine ou la portée est la \emph{classe cible} (c.-à-d. les arcs sortant ou entrant).
\end{definition}

Par exemple, en supposant que \emph{Book} est une \emph{classe cible}, une requête demandant des instances de \textquote{livres édités par le docteur Alice sur la cardiologie parut après 2018} est une requête acceptable.
Cependant, une requête demandant des exemples de \textquote{livres édités par le docteur Alice qui est cardiologue} n'est pas acceptable, car \textquote{est un cardiologue} n'est pas une propriété (ou une arête) de la \emph{classe cible} sélectionnée.
Si l'utilisateur souhaite identifier des médecins ayant rédigé un livre, il doit d'abord modifier la spécification de sa classe cible.
Plus simplement, on ne considère que des requêtes simples qui sont traduites en requêtes conjonctives de base de données identifiant des instances d'une classe particulière, même si elles fournissent également des informations sur les propriétés de ces instances.

\begin{definition}[DB-query]
    Une DB-query $q$ est une requête conjonctive sur un schéma donné $\mathcal{S}$ de la forme $R_0(u_0) \leftarrow \phi$ où $\phi = R_1(u_1) \dots  R_n(u_n), comp_1, \dots, comp_m$ où $n \geq 0$, $R_i$ sont des symboles de prédicats avec $0 \leq i \leq n$, $u_i$ est le tuple de termes de $R_i$ de longueur égale à l'arité de $R_i$ et $com_j$ sont des formules de comparaison qui incluent des variables trouvées dans au moins un tuple de $u_i$ avec $0 \leq j \leq m$.
    $head(q)$ (respectivement $body(q)$) représente la partie gauche de la règle, dénommée tête (respectivement la partie droite, dénommée corps) de $q$.
    $\mathbb{I}_q$ représente l'ensemble de réponses de $q$.

    Les réponses de $q$ sont des instanciations du tuple $u_0$ où l'ensemble des termes $t \in u_0$ sont des constantes.
    C-à-d. qu'il existe, pour chaque instanciation $I \in \mathbb{I}_q$, une surjection $h_I$ de $u_0$ vers $I$ (qui associe des variables à des constantes et une constante à elle-même) tel que : $\{R_1(h_t(u_1)), \dots, R_n(h_t(u_n))\} \subseteq \mathcal{D}$, la conjonction de tous les $h_I(comp_j)$ est évaluée à vrai (selon la sémantique habituelle des opérateurs $op$) et $h_I(u_0)= I$.

    % In this rule-based formalism, the union is expressed by allowing more than one rule with the same head.
    % For instance, 
    % $q(X) \leftarrow writtenBy(X, Bob)$ together with
    % $q(X) \leftarrow editedBy(X, Bob)$
    % express a query looking for documents written or edited by \textit{Bob}.
\end{definition}

\begin{example}
    La requête $NLQ_{run}$ en langage naturel suivante sera utilisée à titre d'exemple :
    \begin{displayquote}
        Livres intitulés \textquote{Principes de médecine}, écrits par Alice et Bob et dont le prix est inférieur à 30 euros.
    \end{displayquote}

    La figure~\ref{fig:nl-query:dep} montre l'arbre de dépendance pour une version simplifiée de $NLQ_{run}$.
    Dans cette requête, \textquote{Livre} représente la \emph{classe cible} : \emph{Book}.
    Cette requête se traduit dans la DB-query $Q_{run}$ suivante où \emph{:alice} et \emph{:bob} sont respectivement les identifiants en base d'Alice et Bob :
    \begin{equation*}
        \begin{split}
            Q(x) \leftarrow & Book(x), hasTitle(x, y_1), writtenBy(x, y_2), Person(y_2), writtenBy(x, y_3), Person(y_3),                      \\
                            & hasPrice(x, y_4), (y_1 = \text{"Principes de médecine"}), (y_2= \text{:alice}), (y_3 = \text{:bob}), (y_4 < 30)
        \end{split}
    \end{equation*}
\end{example}

\begin{figure}[htb]
    \tiny
    \centering
    \begin{adjustbox}{width=\linewidth}
        \begin{dependency}[theme=simple, edge horizontal padding=5ex, edge unit distance=1em, column sep=2em]
            \begin{deptext}
                Livres \& écrits \& par \& Alice \& et \& Bob \& et \& dont \& le \& prix \& est \& inférieur \& à \& 30 \& euros \\
                NOUN \& VERB \& ADP \& PROPN \& CCONJ \& PROPN \& CCONJ \& PROPN \& DET \& NOUN \& AUX \& ADJ \& ADP \& NUM \& NOUN \\
            \end{deptext}

            \deproot[edge height=15ex]{1}{ROOT}
            \depedge{1}{2}{acl}
            \depedge{2}{4}{auoblx}
            \depedge{4}{3}{case}
            \depedge{4}{6}{conj}
            \depedge{6}{5}{cc}
            \depedge[segmented edge, edge height=13ex]{2}{12}{conj}
            \depedge[segmented edge, edge height=10ex]{12}{7}{cc}
            \depedge{12}{10}{nsubj}
            \depedge{12}{11}{cop}
            \depedge{10}{8}{nmod}
            \depedge{10}{9}{det}
            \depedge{12}{15}{obl}
            \depedge{15}{13}{case}
            \depedge{15}{14}{nummod}
        \end{dependency}
    \end{adjustbox}
    \caption{Arbre de dépendance d'une version simplifiée de $NLQ_{run}$ obtenu avec \gls*{spacy}}
    \label{fig:nl-query:dep}
\end{figure}

\subsection{Extraction d'entités}

En s'appuyant sur les sections précédentes, l'ensemble des classes présentes dans la base de données peuvent être représentés par un lexique.
Dans l'exemple de $NLQ_{run}$, on suppose que l'on a un lexique de \emph{Personne} qui à pour valeur les identifiants des individus associés au nom, prénom, etc de chaque personne.
Ainsi, l'ensemble des individus de la base peuvent être reconnus et associé directement à la classe (correspondant au type de l'entité).
Un lexique des classes cible disponible et un lexique des champs/propriété dans la base sont aussi construit.
Ces lexiques nécessitent généralement une intervention humaine pour compléter la liste des synonymes pour ne pas uniquement dépendre du nom de la classe.
Pour identifier les dates et les nombre (ex : 30 euros), on utilise un ensemble de grammaire locale.
Les \emph{texte libres}, comme le titre dans $NLQ_{run}$, sont extrait à l'aide d'un \gls{crf} appris sur un jeu de donnée généré en utilisant un \gls{dsl}.
L'exemple~\ref{ex:nl-query:simpleEnts} montre les entités extraites pour la requête $NLQ_{run}$.

\begin{example}
    \label{ex:nl-query:simpleEnts}
    Ci-dessous nous présentons la requête $NLQ_{run}$ avec les entités extraites dans des rectangles.
    Chaque entité est numérotée.
    Les entités \ref{nl-query:e1}, \ref{nl-query:e3}, \ref{nl-query:e4} et \ref{nl-query:e5} sont extraite par l'intermédiaire de lexiques.
    L'entité \ref{nl-query:e6} est obtenu en utilisant une grammaire locale et \ref{nl-query:e2} et extraite à l'aide d'un \gls{crf}.

    \begin{displayquote}
        \fbox{Livres~\tiny{1}} intitulés \fbox{\textquote{Principes de médecine}~\tiny{2}} écrits par \fbox{Alice~\tiny{3}} et \fbox{Bob~\tiny{4}} et dont le \fbox{prix~\tiny{5}} est \fbox{inférieur à~\tiny{6}} \fbox{30 euros~\tiny{7}}.
    \end{displayquote}

    \noindent
    On a donc les entités suivantes :
    \begin{multicols}{2}
        \begin{enumerate}[label=$E_{\arabic*}$]
            \item \label{nl-query:e1} $= (Book, Class)$
            \item \label{nl-query:e2} $= (\text{Principes de médecine}, Text)$
            \item \label{nl-query:e3} $= (:alice, Person)$
            \item \label{nl-query:e4} $= (:bob, Person)$
            \item \label{nl-query:e5} $= (hasPrice, Attribut)$
            \item \label{nl-query:e6} $= (<, Operator)$
            \item \label{nl-query:e7} $= (30, Number)$
        \end{enumerate}
    \end{multicols}

    Pour des raisons de lecture, on ne représente pas la notation des ensembles de valeurs et de types.
    Ici, chaque entité n'a qu'une seule valeur et chaque valeur est associée à un unique type.
    Dans la pratique, il serait possible d'avoir plusieurs valeurs ou types à cause de l'ambiguïté.
    Ainsi, si $p_1$ et $p_2$ sont deux personnes dans la base qui se prénomme Alice, alors on aurait :
    \begin{equation*}
        E_3 = (\{p_1, p_2\}, \{Personne\}, \langle p_1 \mapsto \{Personne\}, p_2 \mapsto \{Personne\} \rangle)
    \end{equation*}
\end{example}

L'extraction des associations \emph{classe cible} -- attributs (c.-à-d. des arêtes dans le graphe) se fait par contextualisation.
Comme discuté dans la section~\ref{sec:tal:ctx}, cette dernière peut s'opérer par extraction de marqueur de contexte (dans $NLQ_{run}$ on peut identifier les marqueurs \textquote{intitulés} et \textquote{écrits par}) ou en utilisant un \gls{crf}.
Ici, on utilise des \acrshortpl{crf} pour les attributs génériques et l'ensemble de règles donné dans la section~\ref{sec:tal:ctx:rule} pou identifier le lien entre une entité correspondante à un attribut (ex : \ref{nl-query:e5} \textquote{prix}) et l'entité correspondante à la valeur de l'attribut (ex : \ref{nl-query:e6} \textquote{30 euros}).
Dans $NLQ_{run}$, on remarque la présence d'un opérateur \textquote{inférieur à}.
Un opérateur est un modificateur (comme les contextes) qui vient enrichir la valeur (contrairement au type pour les contextes) d'une entité.
Les opérateurs sont identifiés comme les contextes, on utilise un lexique pour le marqueur et la détection des liens avec une entité est effectué par l'ensemble de règles sur l'arbre de dépendances (figure~\ref{fig:nl-query:dep}) utilisées pour la contextualisation et listée dans la section~\ref{sec:tal:ctx:rule}.
Il est possible d'avoir des opérateurs sans valeurs associées et qui peuvent être liée à un attribut de la \emph{classe cible}.
Par exemple, c'est le cas pour les opérateurs \textquote{est vide} et \textquote{est rempli}.

\begin{example}
    \label{ex:nl-query:enrichEnts}
    Pour $NLQ_{run}$, l'association entre les entités \ref{nl-query:e5} \textquote{prix} et \ref{nl-query:e6} \textquote{30 euros} est réalisé en deux temps.
    On commence par associer l'opérateur \textquote{inférieur à} à l'entité \ref{nl-query:e7}.
    On obtient alors l'annotation suivante :
    \begin{displayquote}
        \fbox{Livres~\tiny{1}} intitulés \fbox{\textquote{Principes de médecine}~\tiny{2}} qui ont été écrits par \fbox{Alice~\tiny{3}} et \fbox{Bob~\tiny{4}} et dont le \fbox{prix~\tiny{5}} est \fbox{inférieur à 30 euros~\tiny{7}}.
    \end{displayquote}
    Ensuite, on identifie que l'entité \ref{nl-query:e7} ($<30$) est associé à l'attribut pointé par l'entité \ref{nl-query:e5} ($hasPrice$).
    Seule les types associée au schéma de la base sont gardés ainsi, les types \emph{Text}, \emph{Number} et \emph{Date} sont filtrés.
    On obtient alors l'entité enrichie \ref{nl-query:e7} $= (30, \{Number, hasPrice\}, <)$.
    \begin{multicols}{2}
        \begin{enumerate}[label=$E_{e\arabic*}$]
            \item \label{nl-query:ee1} $= \{(Book, \{Class\}, =)\}$
            \item \label{nl-query:ee2} $= \{(\text{"Principes de médecine"}, \{Title\}, =)\}$
            \item \label{nl-query:ee3} $= \{(\text{:alice}, \{Person, Author\}, =)\}$
            \item \label{nl-query:ee4} $= \{(\text{:bob}, \{Person, Author\}, =)\}$
            \item \label{nl-query:ee5} $= \{(30, \{hasPrice\}, <)\}$
        \end{enumerate}
    \end{multicols}
\end{example}

\subsection{Construction de DB-query à partir d'entités enrichies}

\SetKwFunction{GetNewVar}{getNewVar}
\SetKwFunction{BuildAtom}{buildAtom}
\SetKwFunction{BuildAtomOP}{buildAtomOP}
\SetKwFunction{BuildNewQuery}{buildNewQuery}

Après analyse de la NL-query et l'extraction de l'ensemble $\mathcal{E}$ des entités enrichies, la DB-query est construite à l'aide de la procédure~\ref{algo:nl-query}.
L'algorithme commence par considérer l'entité $E_{e1}$ (ligne~\ref{algo:nl-query:target}) qui a un rôle spécial puisqu'elle spécifie la \emph{classe cible} qui est interrogée.
La requête est initialisée avec un corps composé d'un unique atome sur le prédicat associé à la classe cible.
La procédure \BuildAtom est responsable de la construction d'un atome pour la requête en cours de construction.
Le symbole du prédicat à utiliser dans la construction d'un atome est trouvé via la valeur de l'entité $E_{e1}$.
Dans l'exemple~\ref{ex:nl-query:simpleEnts}, \emph{Book} est la valeur de \ref{nl-query:ee1} et le nom du prédicat unaire associé.
Dans notre cas, l'atome $A_0$ est $Book(x)$.
Remarquons que l'algorithme~\ref{algo:nl-query} ne construit que des requêtes dont les réponses sont des identifiants de livres (c'est-à-dire des instanciations de $x$).
Dans cet exemple, la requête initiale est donc : $q(x) \leftarrow Book(x)$.

\begin{procedure}[htb]
    \caption{entitiesToQueries($\mathcal{E}$)}
    \label{algo:nl-query}

    %\Input{$\mathcal{E}$ an enriched entity set $\{E_{e0}, E_{e1}, \dots\}$}
    %\Output{$\mathcal{Q}$ a set of query rules, i.e., the DB-query with one or more rules}

    $\mathcal{Q} \gets \emptyset$ \;
    \ForEach{entité enrichie $E \in \mathcal{E}$}{
        \eIf{$E$ is $E_{e1}$\label{algo:nl-query:target}}{
            $\{ (value, type, op) \} \gets E$ \;
            $A_0 \gets \BuildAtom(value, x)$ \tcp*[l]{Construit le premier atome du corps de la requête}
            $\mathcal{Q} \gets \{ Q(x) \leftarrow A_0 \}$ \;
        }{
            $Parts \gets \emptyset$ \tcp*[l]{Ensemble des listes d'atomes. Chaque $l \in Parts$ est une liste d'atomes dont la conjonction doit être ajoutée au corps de la requête.}

            \ForEach{$(value, op) \in E$\label{algo:nl-query:ent-start}}{
                \ForEach{$type \in E(value)$\label{algo:nl-query:types}}{
                    $y \gets \GetNewVar$ \;
                    $part \gets \{\BuildAtomOP(value, y, opval), \BuildAtom(type, y)\}$ \;
                    $Parts \gets Parts \cup \{part\}$\label{algo:nl-query:part} \;
                }
            }\label{algo:nl-query:ent-end}

            $\mathcal{Q}' \gets \emptyset$ \;
            \ForEach{requête $q \in \mathcal{Q}$\label{algo:nl-query:queries}}{
                \ForEach{$part \in Parts$}{
                    $q' = \BuildNewQuery(q, part)$ \label{algo:nl-query:query} \;
                    $\mathcal{Q}' \gets \mathcal{Q}' \cup \{q'\}$ \;
                }
            }
            $\mathcal{Q} \gets \mathcal{Q}'$ \;
        }
    }
    \Return $\mathcal{Q}$ \;
\end{procedure}

Les lignes~\ref{algo:nl-query:ent-start} à \ref{algo:nl-query:ent-end} de la procédure~\ref{algo:nl-query} traite les entités $E_e$ enrichies.
On commence par construire pour chaque valeur de l'entité une variable avec la fonction \GetNewVar et on construit la formule de comparaison correspondant à l'opérateur entre la variable et la valeur à l'aide de la fonction \BuildAtomOP.
Pour chaque type associé à la valeur ($E(value)$, ligne~\ref{algo:nl-query:types}), on construit les atomes correspondants.
La fonction \BuildAtom est utilisée pour construire les atomes qui seront ajoutés à $body(q)$.
Notez que \BuildAtom associe à chaque type, le symbole de prédicat correspondant et construit l'atome en prenant en compte l'emplacement de la valeur de l'entité (variable $y$) dans l'atome
Dans les prédicats binaires, $x$ est toujours l'autre variable.
Si, dans $E_e$, il y a plus d'une valeur, alors l'entité est ambiguë, soit parce qu'elle a été extraite par un lexique et que le lexème est ambiguë soit c'est une entité obtenue après prise en compte de la conjonction de coordination \textit{or}.
On matérialise l'ambiguïté par une union de requête.
Pour ce faire, la procédure~\ref{algo:nl-query} divise les valeurs de $E_e$ en sous partie de la requête.
Chaque $l \in Parts$ est un ensemble d'atomes à ajouter au corps de la requête en cours de construction.
Dans la boucle à la ligne~\ref{algo:nl-query:queries} on calcule le produit cartésien entre les requêtes déjà construites et les parties correspondantes à chaque valeur de l'entité $E_e$.
C'est-à-dire que si $E_e$ possède deux valeurs possibles, on aura le double de requêtes qui seront construite.

D'après l'exemple~\ref{ex:nl-query:enrichEnts}, \ref{nl-query:ee3} a deux types : \emph{Person} et \emph{Author}.
Ainsi, sur la ligne~\ref{algo:nl-query:part} on ajoute l'ensemble $\{ Person(y_2),$ $writtenBy(x, y_2),$ $(y_2= \text{:alice}) \}$ à $Parts$ et à la ligne~\ref{algo:nl-query:query} la requête construite est :
\begin{equation*}
    \begin{split}
        q(x) \leftarrow & Book(x), Person(y_2), writtenBy(x, y_2), (y_2= \text{:alice})
    \end{split}
\end{equation*}

Le résultat obtenu avec l'entité \ref{nl-query:ee4} est similaire.
Pour \ref{nl-query:ee2} et \ref{nl-query:ee5}, un seul prédicat est construit, car les valeurs n'ont qu'un seul type associé.
Par exemple, \ref{nl-query:ee2} donne lieu à l'ensemble $\{ hasTitle(x, y_1),$ $(y_1 = \text{"Principes de médecine"}) \}$.
Le traitement de l'entité \ref{nl-query:ee5} construit la comparaison $y_4 < 30$.
Après avoir pris en compte toutes les entités, la procédure~\ref{algo:nl-query} renvoie l'ensemble $\mathcal{Q}$ qui forme la DB-query suivante :
\begin{equation*}
    \begin{split}
        Q(x) \leftarrow & Book(x), hasTitle(x, y_1), writtenBy(x, y_2), Person(y_2), writtenBy(x, y_3), Person(y_3),                      \\
                        & hasPrice(x, y_4), (y_1 = \text{"Principes de médecine"}), (y_2= \text{:alice}), (y_3 = \text{:bob}), (y_4 < 30)
    \end{split}
\end{equation*}

Cependant, considérons qu'il existe une ambiguïté sur l'entité \ref{nl-query:ee3} avec les valeurs \emph{:alice1} et \emph{:alice2} tel que \ref{nl-query:ee3} $= \{(\text{:alice1},$ $\{Person, Author\}, =),$ $(\text{:alice2},$ $\{Person, Author\}, =)\}$.
La procédure~\ref{algo:nl-query} (lignes~\ref{algo:nl-query:ent-start} à~\ref{algo:nl-query:ent-end}) produit deux sous requête à partir de la même entité, à savoir :
\begin{enumerate*}[label=(\roman*)]
    \item $part_1 = \{ Person(y_2),$ $writtenBy(x, y_2),$ $(y_2 = \text{:alice1}) \}$ et
    \item $part_2 = \{ Person(y_2),$ $writtenBy(x, y_2),$ $(y_2 = \text{:alice2}) \}$.
\end{enumerate*}
Ensuite, à la ligne~\ref{algo:nl-query:queries}, chaque sous requête est considérée séparément et la requête $Q$ est remplacée par deux nouvelles requêtes.
À la fin de la procédure, $\mathcal{Q}$ est une requête DB-query composée de deux règles :
\begin{equation*}
    \begin{split}
        Q(x) \leftarrow & Book(x), hasTitle(x, y_1), writtenBy(x, y_2), Person(y_2), writtenBy(x, y_3), Person(y_3),                       \\
                        & hasPrice(x, y_4), (y_1 = \text{"Principes de médecine"}), (y_2= \text{:alice1}), (y_3 = \text{:bob}), (y_4 < 30) \\
        Q(x) \leftarrow & Book(x), hasTitle(x, y_1), writtenBy(x, y_2), Person(y_2), writtenBy(x, y_3), Person(y_3),                       \\
                        & hasPrice(x, y_4), (y_1 = \text{"Principes de médecine"}), (y_2= \text{:alice2}), (y_3 = \text{:bob}), (y_4 < 30)
    \end{split}
\end{equation*}

Enfin, considérons la NL-query \textit{livres édités ou écrits par Alice} comprenant l'entité enrichie $E_{e2} = \{(\text{:alice}, \{Person, Author\}, =), (\text{:alice}, \{Person, Editor\}, =)\}$.
Les ensembles construits à la ligne~\ref{algo:nl-query:part} sont :
\begin{enumerate*}[label=(\roman*)]
    \item $part_1 = \{ Person(y),$ $writtenBy(x, y),$ $(y = \text{:alice}) \}$ et
    \item $part_2 = \{ Person(y),$ $editedBy(x, y),$ $(y = \text{:alice}) \}$.
\end{enumerate*}
On obtient alors la DB-query suivante :
\begin{equation*}
    \begin{split}
        Q(x) \leftarrow & Book(x), Person(y), writtenBy(x, y), (y = \text{:alice}) \\
        Q(x) \leftarrow & Book(x), Person(y), editedBy(x, y), (y = \text{:alice})
    \end{split}
\end{equation*}

\subsection{Évaluation}

On évalue le système sur un cas concret d'une instance de la \glsreset{ged}\gls{ged} d'\gls{ennov}.
L'ensemble des métadonnées des documents sont regroupés et indexés dans des lexiques de façon automatique.
Après l'extraction des diverses entités on ajoute une phase de post-traitement pour supprimer les incohérences.
Par exemple, on vérifie qu'il s'agit bien d'une personne dans un champ personne, sinon on supprime la condition.
Cette phase de post-traitement permet ainsi de corriger les éventuelles erreurs qui ont eu lieu lors de l'extraction et de la contextualisation.

\begin{figure}[htb]
    \centering
    \includegraphics[width=\textwidth]{these/part_2/chapter_2/imgs/ui-nlsearch-compress.png}
    \caption[Interface utilisateur pour la recherche en langage naturel]{Interface utilisateur et exemple de résultat pour la recherche en langage naturel}
    \label{fig:nl-query:ui}
\end{figure}

La figure~\ref{fig:nl-query:ui} montre l'interface utilisateur et un exemple de requête en anglais.
Le résultat de la recherche n'est pas présent sur la figure.
Pour aider l'utilisateur à comprendre la requête qui a été construite, on la représente sous la forme d'un arbre comme présenté sur la figure.
On remarque par exemple l'utilisation de la conjonction \textquote{ou} et l'ambiguïté du prénom \emph{John}.

\begin{table}[htb]
    \centering
    \begin{tabular}{r|cccc}
                         & Précision      & Rappel         & Mesure F1      & Support    \\
        \hline
        \hline
        object type      & \num{0,600764} & \num{0,990551} & \num{0,747919} & \num{ 635} \\
        person           & \num{0,654574} & \num{0,734513} & \num{0,692244} & \num{1130} \\
        field            & \num{0,292157} & \num{0,977049} & \num{0,449811} & \num{ 305} \\
        type             & \num{0,330645} & \num{0,976190} & \num{0,493976} & \num{ 168} \\
        unit             & \num{0,638254} & \num{0,987138} & \num{0,775253} & \num{ 311} \\
        status           & \num{0,575368} & \num{0,970543} & \num{0,722447} & \num{ 645} \\
        \hline
        time             & \num{0,941496} & \num{0,916618} & \num{0,928890} & \num{1703} \\
        title            & \num{0,577215} & \num{0,679920} & \num{0,624372} & \num{1006} \\
        \hline
        application date & \num{0,888889} & \num{0,522105} & \num{0,657825} & \num{ 475} \\
        archive date     & \num{0,899135} & \num{0,715596} & \num{0,796935} & \num{ 436} \\
        creation date    & \num{0,698080} & \num{0,804829} & \num{0,747664} & \num{ 497} \\
        expiration date  & \num{0,814655} & \num{0,675000} & \num{0,738281} & \num{ 280} \\
        issuer           & \num{0,724954} & \num{0,597598} & \num{0,655144} & \num{ 666} \\
        signatories      & \num{0,708738} & \num{0,478166} & \num{0,571056} & \num{ 458} \\
        \hline
        Total            & \num{0,667495} & \num{0,787558} & \num{0,685844} & \num{8715}
    \end{tabular}
    \caption{Score obtenu pour l'extraction d'entités pour la recherche en langage naturel}
    \label{tab:nl-query:result}
\end{table}

La table~\ref{tab:nl-query:result} présente l'évaluation sur le corpus synthétique de test.
Le corpus de test a été généré en utilisant le même \gls{dsl} que pour les données d'entrainement des \acrshortpl{crf}.
La première section (de \emph{object type} à \emph{unit}) sont des entités extraites en utilisant des lexiques.
Les entités \emph{time} sont extraite en utilisant des grammaires locales et \emph{title} à l'aide d'un \gls{crf}.
La section suivante (de \emph{application date} à \emph{signatories}) concerne les entités contextualisées, les dates à partir de l'entité \emph{time} et \emph{issuer/signatories} à partir de l'entité \emph{person}.

De façon générale, on observe que les lexiques donne un très bon rappel (supérieur à \num{0.9}), mais une précision plus faible, en particulier pour les entités \emph{field} et \emph{type}.
Cela s'explique par le manque de filtrage sur les lexiques : dans cette version on ne filtre pas les lexèmes trop génériques.
\emph{field} et \emph{type} sont deux entités particulières où les lexèmes associés à chaque valeur peuvent être long.
Pour l'entité \emph{title}, il est difficile d'annoté correctement les bornes de l'entité à cause de la longueur de la séquence et de la grande variabilité dans les titres.
L'entité \emph{time} est l'entité la mieux annotée avec une mesure F1 de \num{0.93}.
Concernant la contextualisation, on remarque que les dates sont correctement classifiée à \SI{70}{\percent}.
Pour les entités \emph{issuer} et \emph{signatories} on remarque que le système n'arrive pas à classifier correctement les personnes.
Le \gls{crf} associé s'appuie sur les entités \emph{person} extraite précédemment et dépend alors de sa qualité qui est parmi les trois plus faibles avec seulement \num{0.69} pour la mesure F1.

Une des problématiques de cette approche est que les différents composants du système sont entrainé de façon individuelle.
Il pourrait être intéressant d'entrainer les \gls{crf} à partir des annotations que les composants précédents sont capables d'extraire plutôt que depuis les données correctement annotées.

\FloatBarrier
\subsection{Construction de graphique}

\begin{figure}[htb]
    \centering
    \includegraphics[width=\textwidth]{these/part_2/chapter_2/imgs/ui-dashboard-compress.png}
    \caption[Interface utilisateur pour la construction de graphique]{Interface utilisateur et exemple de résultat pour la construction de graphique à partir d'une requête en langage naturel}
    \label{fig:nl-query:dashboard-ui}
\end{figure}

Pour aller plus loin, nous proposons de construire des graphiques à partir d'une requête en langage naturel.
La figure~\ref{fig:nl-query:dashboard-ui} illustre le résultat dans l'interface.
L'idée est qu'un utilisateur puisse construire un graphique simplement avec les données présente dans la base.
Le graphique est construit à partir des champs d'une unique table.
Ici, on recherche l'évolution du nombre de cas de cosmetovigilance par année de réception par l'entreprise et par pays.

Pour cela, on commence par identifier le type de graphique (\emph{chart type}), si on ne le trouve pas, on détermine le type automatiquement en fonction du nombre d'axes sélectionnés.
Ensuite on identifie les différents axes du graphique : \emph{value axis} pour les ordonnées, \emph{x axis} pour l'axe des abscisses et \emph{y axis} pour le second axe utilisé pour construire plusieurs courbes dans l'exemple de la figure~\ref{fig:nl-query:dashboard-ui}.
Pour chaque axe, on identifie l'étiquette (\emph{value label}, \emph{x label} et \emph{y label}) si elle est fournis, le champ de la table utilisé pour l'axe (\emph{field}) la fonction d'agrégation pour les valeurs (\emph{axis aggregate}, moyenne, somme, etc) et le modificateur (\emph{axis modifier}, par exemple, pour une date on peut demander seulement le mois ou l'année).
Pour finir on cherche la table à utiliser, si elle n'est pas spécifiée dans la requête de l'utilisateur, on cherche la table la plus petite qui contient les champs identifiés.
Ici on ne s'intéresse qu'à l'extraction des entités, mais un filtrage est effectué après l'extraction pour s'assurer que le graphique construit est cohérent, permettant de rejeter certaines ambiguïtés.

\begin{table}[htb]
    \centering
    \begin{tabular}{r|cccc}
                       & Précision      & Rappel         & Mesure F1      & Support    \\
        \hline
        \hline
        axis aggregate & \num{0,910714} & \num{0,980769} & \num{0,944444} & \num{ 156} \\
        axis modifier  & \num{0,432500} & \num{0,994253} & \num{0,602787} & \num{ 174} \\
        chart type     & \num{0,936747} & \num{1,000000} & \num{0,967341} & \num{ 311} \\
        table          & \num{0,691983} & \num{0,716157} & \num{0,703863} & \num{ 229} \\
        field          & \num{0,292157} & \num{0,977049} & \num{0,449811} & \num{ 305} \\
        \hline
        value axis     & \num{0,940461} & \num{0,990560} & \num{0,964860} & \num{1483} \\
        value label    & \num{0,894886} & \num{0,875000} & \num{0,884831} & \num{ 360} \\
        x axis         & \num{0,894330} & \num{0,973807} & \num{0,932378} & \num{1069} \\
        x label        & \num{0,915541} & \num{1.000000} & \num{0,955908} & \num{ 271} \\
        y axis         & \num{0,822669} & \num{0,813743} & \num{0,818182} & \num{ 553} \\
        y label        & \num{0,913043} & \num{0,777778} & \num{0,840000} & \num{ 162} \\
        \hline
        Total          & \num{0,785912} & \num{0,918128} & \num{0,824037} & \num{5073}
    \end{tabular}
    \caption[Score obtenu pour l'extraction d'entités pour la construction de graphiques]{Score obtenu pour l'extraction d'entités pour la construction de graphiques à partir d'une requête en langage naturel}
    \label{tab:nl-query:result-dashboard}
\end{table}

La table~\ref{tab:nl-query:result-dashboard} présente les résultats obtenus.
Le premier groupe d'entité (de \emph{axis aggregate} à \emph{tables}) sont extrait à partir de lexiques te le second groupe (de \emph{value axis} à \emph{y label}) sont extrait à partir de \acrshortpl{crf}.

\FloatBarrier
