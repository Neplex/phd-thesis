\section{Movie}

\subsection{Description}

La base de données \emph{Movie} (\url{https://github.com/neo4j-graph-examples/movies}) est fournie par \gls{neo4j} et sert d'exemple pratique pour l'apprentissage de la syntaxe \gls{cypher} et l'exécution d'opérations courantes sur les bases de données graphes telles que l'interrogation, le filtrage et l'exploration des relations.
La base de données contient des informations sur les films, les acteurs, les réalisateurs, les producteurs, les scénaristes, les critiques et leurs relations.
Chaque entité \emph{Person} ou \emph{Movie} comprend plusieurs attributs.
Les acteurs, réalisateurs, producteurs, scénaristes et critiques sont liés aux films par les relations : \emph{ActedIn}, \emph{Directed}, \emph{Produced}, \emph{Wrote} et  \emph{Reviewed}.
Les relations \emph{ActedIn} et \emph{Reviewed} possèdent aussi des attributs.

\begin{figure}[H]
    \centering
    \begin{adjustbox}{max width=.8\linewidth, max height=.5\textheight}
        \begin{tikzpicture}[->,>=stealth',shorten >=1pt,auto,block/.style={rectangle,align=center},node distance=4cm]
            \node[draw] (person) {Person};
            \node[draw, right = of person] (movie) {Movie};

            \path
            (person) edge [bend left=40] node [above] {ActedIn} (movie)
            (person) edge [bend left=20] node [above] {Directed} (movie)
            (person) edge node [above] {Produced} (movie)
            (person) edge [bend right=20] node [above] {Wrote} (movie)
            (person) edge [bend right=40] node [above] {Reviewed} (movie)
            (person) edge [loop left] node {Follows} (person)
            ;
        \end{tikzpicture}
    \end{adjustbox}
    \caption{Schéma de la base \emph{Movie}}
\end{figure}

\subsection{Prédicats et contraintes}

\begin{table}[H]
    \centering
    \begin{adjustbox}{max width=\linewidth}
        \begin{tabular}{l|c|l|l}
            Prédicat   & Arité & Termes                               & Description                                      \\
            \hline
            \hline
            $Movie$    & 4     & $id, year, title, subtitle$          & La description d'un film                         \\
            $Person$   & 3     & $id, birthYear, name$                & La description d'une personne                    \\
            $ActedIn$  & 3     & $actorId, movieId, role$             & Relie un acteur à un rôle dans un film           \\
            $Directed$ & 2     & $directorId, movieId$                & Relie un réalisateur à un film                   \\
            $Produced$ & 2     & $producerId, movieId$                & Relie un producteur à un film                    \\
            $Wrote$    & 2     & $authorId, movieId$                  & Relie un scénariste à un film                    \\
            $Reviewed$ & 4     & $reviewerId, movieId, score, review$ & Une critique sur un film, rattaché à l'auteur    \\
            $Follows$  & 2     & $followerId, personId$               & Décrit une relation de suivi entre les personnes \\
        \end{tabular}
    \end{adjustbox}
    \caption{Description de la base \emph{Movie}}
\end{table}

\begin{figure}[H]
    \centering
    \begin{adjustbox}{max width=\linewidth, max height=.5\textheight}
        \parbox{\linewidth}{\begin{align*}
                Movie(movieId, year, title, subtitle)        & \to  ActedIn(actorId, movieId, role)              \\
                                                             & \to  Directed(directorId, movieId)                \\
                                                             & \to  Produced(producerId, movieId)                \\
                                                             & \to  Reviewed(reviewerId, movieId, score, review) \\
                                                             & \to Wrote(authorId, movieId)                      \\
                ActedIn(actorId, movieId, role)              & \to Person(actorId, birthYear, name)              \\
                                                             & \to Movie(movieId, year, title, subtitle)         \\
                Directed(directorId, movieId)                & \to  Person(directorId, birthYear, name)          \\
                                                             & \to  Movie(movieId, year, title, subtitle)        \\
                Produced(producerId, movieId)                & \to  Person(producerId, birthYear, name)          \\
                                                             & \to  Movie(movieId, year, title, subtitle)        \\
                Wrote(authorId, movieId)                     & \to  Person(authorId, birthYear, name)            \\
                                                             & \to  Movie(movieId, year, title, subtitle)        \\
                Reviewed(reviewerId, movieId, score, review) & \to  Person(reviewerId, birthYear, name)          \\
                                                             & \to  Movie(movieId, year, title, subtitle)        \\
                Follows(followerId, personId)                & \to  Person(followerId, birthYear, name)          \\
                                                             & \to  Person(personId, birthYear, name)
            \end{align*}}
    \end{adjustbox}
    \caption{Liste des contraintes de la base \emph{Movie}}
\end{figure}

\section{GameOfThrone}

\subsection{Description}

La base de données \emph{GOT} \enquote{Game of Thrones} (\url{https://github.com/neo4j-graph-examples/graph-data-science}), fournie par \gls{neo4j}, est un exemple de démonstration pour leur bibliothèque \emph{Graph Data Science}, axée sur la détection des communautés et l'analyse des réseaux basé sur les travaux de \cite{beveridgeNetworkThrones2016}.
Elle modélise les relations au sein du monde fictif de Westeros, tiré de la série de livres \enquote{A Song of Ice and Fire} de George R.R. Martin.
Le graphe contient des nœuds \emph{Person}, représentant les personnages et des relations \emph{Interacts}, représentant les interactions entre les personnages.
Une interaction se produit chaque fois que les noms (ou surnoms) de deux personnages apparaissent à moins de 15 mots l'un de l'autre dans le texte du livre.
Pour chaque personnage, nous avons sons status, la maison auquel il appartient et les livres ou il est mentionné.
Une partie du graphe modélise les batailles avec ses différents acteurs et la zone géographique ou elle prend place.

\begin{figure}[H]
    \centering
    \begin{adjustbox}{max width=.9\linewidth,max height=.5\textheight}
        \begin{tikzpicture}[->,>=stealth',shorten >=1pt,auto,block/.style={rectangle,align=center},node distance=2cm]
            \node[draw] (person) {Person};
            \node[draw, left = 4.5cm of person] (book) {Book};
            \node[draw, right = 4.5cm of person] (battle) {Battle};
            \node[draw, above = of battle] (house) {House};
            \node[draw, below = of battle] (region) {Region};
            \node[draw, right = of region] (location) {Location};
            \node[draw, above = of book] (status) {Status};
            \node[draw, below = of book] (culture) {Culture};

            \path
            (person) edge [bend left=20] node [below] {AppearedIn} (book)
            (person) edge [bend right=20] node [above] {DiedIn} (book)
            (person) edge [bend left] node [above left] {BelongsTo} (house)
            (person) edge [bend right] node [above right] {HasStatus} (status)
            (person) edge [bend left] node [below right] {MemberOfCulture} (culture)
            (person) edge [loop above] node [above] {Interacts} (person)
            (person) edge [loop below] node [below] {RelatedTo} (person)
            (person) edge [bend left=30] node [above] {DefenderCommander} (battle)
            (person) edge [bend left=10] node [above] {DefenderKing} (battle)
            (person) edge [bend right=30] node [below] {AttackerCommander} (battle)
            (person) edge [bend right=10] node [below] {AttackerKing} (battle)
            (house) edge [bend left=10] node [right, near start] {Defender} (battle)
            (house) edge [bend right=10] node [left, near start] {Attacker} (battle)
            (battle) edge node [above right] {IsIn} (location)
            (battle) edge node [left] {IsIn} (region)
            (location) edge [loop above] node [above] {IsIn} (location)
            (location) edge node [above] {IsIn} (region)
            ;
        \end{tikzpicture}
    \end{adjustbox}
    \caption{Schéma de la base \index{GOT}}
\end{figure}

\subsection{Prédicats et contraintes}

\begin{table}[H]
    \centering
    \begin{adjustbox}{max width=\linewidth}
        \begin{tabular}{l|c|l|l}
            Prédicat            & Arité & Termes                                    & Description                               \\
            \hline
            \hline
            $Book$              & 3     & $bookId, name, bookNumber$                & La description d'un livre                 \\
            $House$             & 2     & $houseId, name$                           & La description d'une maison               \\
                                &       & $personId, age, birthYear, deathChapter,$ &                                           \\
            $Person$            & 10    & $introChapter, community,deathYear,$      & La description d'une personne             \\
                                &       & $gender, name, title$                     &                                           \\
                                &       & $battleId, attackerSize, battleType,$     &                                           \\
            $Battle$            & 10    & $defenderSize, majorCapture, majorDeath,$ & La description d'une bataille             \\
                                &       & $name, note, summer, year$                &                                           \\
            $Status$            & 2     & $statusId, name$                          & La description d'un statut                \\
            $Culture$           & 2     & $cultureId, name$                         & La description d'une culture              \\
            $Region$            & 1     & $regionName$                              & Une région                                \\
            $Location$          & 1     & $locationName$                            & Un lieu                                   \\
            $Defender$          & 3     & $houseId, battleId, outcome$              & Une maison défendant dans une bataille    \\
            $Attacker$          & 3     & $houseId, battleId, outcome$              & Une maison attaquant dans une bataille    \\
            $DefenderCommander$ & 2     & $personId, battleId$                      & Le commandant défendant dans une bataille \\
            $AttackerCommander$ & 2     & $personId, battleId$                      & Le commandant attaquant dans une bataille \\
            $DefenderKing$      & 2     & $personId, battleId$                      & Le roi défendant dans une bataille        \\
            $AttackerKing$      & 2     & $personId, battleId$                      & Le roi attaquant dans une bataille        \\
            $AppearedIn$        & 2     & $personId, bookId$                        & Un personnage est apparu dans un livre    \\
            $DiedIn$            & 2     & $personId, bookId$                        & Un personnage est mort dans un livre      \\
            $HasStatus$         & 2     & $personId, statusId$                      & Un personnage a un certain statut         \\
            $BelongsTo$         & 2     & $personId, houseId$                       & Un personnage appartient à une maison     \\
            $MemberOfCulture$   & 2     & $personId, cultureId$                     & Un personnage est membre d'une culture    \\
            $Interacts$         & 4     & $personId, otherId, bookId, weight$       & Interaction entre deux personnages        \\
            $RelatedTo$         & 3     & $personId, otherId, relationName$         & Relation entre deux personnages           \\
            $IsIn$              & 2     & $id, locationId$                          & Une bataille se déroule dans un lieu      \\
                                &       &                                           & ou un lieu est dans un autre              \\
        \end{tabular}
    \end{adjustbox}
    \caption{Description de la base \emph{GOT}}
\end{table}

\begin{figure}[H]
    \centering
    \begin{adjustbox}{max width=\linewidth,max height=.5\textheight}
        \parbox{\linewidth}{\begin{align*}
                AppearedIn(personId, bookId)                 & \to Person(personId, \dots)        \\
                                                             & \to Book(bookId, name, bookNumber) \\
                BelongsTo(personId, houseId)                 & \to Person(personId, \dots)        \\
                                                             & \to House(houseId, name)           \\
                Attacker(houseId, battleId, outcome)         & \to House(houseId, name)           \\
                                                             & \to Battle(battleId, \dots)        \\
                Defender(houseId, battleId, outcome)         & \to House(houseId, name)           \\
                                                             & \to Battle(battleId, \dots)        \\
                DiedIn(personId, bookId)                     & \to Person(personId, \dots)        \\
                                                             & \to Book(bookId, name, bookNumber) \\
                HasStatus(personId, statusId)                & \to Person(personId, \dots)        \\
                                                             & \to Status(statusId, name)         \\
                Interacts(personId, otherId, bookId, weight) & \to Person(personId, \dots)        \\
                                                             & \to Person(otherId, \dots)         \\
                                                             & \to Book(bookId, name, bookNumber) \\
                MemberOfCulture(personId, cultureId)         & \to Person(otherId, \dots)         \\
                                                             & \to Culture(cultureId, name)       \\
                RelatedTo(personId, otherId, relationName)   & \to Person(personId, \dots)        \\
                                                             & \to Person(otherId, \dots)
            \end{align*}}
    \end{adjustbox}
    \caption{Liste des contraintes de la base \emph{GOT}}
\end{figure}

\begin{figure}[H]
    \ContinuedFloat
    \centering
    \begin{adjustbox}{max width=\linewidth,max height=.5\textheight}
        \parbox{\linewidth}{\begin{align*}
                Culture(cultureId, name) & \to MemberOfCulture(personId, cultureId) \\
                House(houseId, name)     & \to BelongsTo(personId, houseId)         \\
                Battle(battleId, \dots)  & \to Attacker(houseId, battleId, outcome) \\
                                         & \to Defender(houseId, battleId, outcome) \\
                                         & \to IsIn(battleId, locationId)           \\
                Person(personId, \dots)  & \to AppearedIn(personId, bookId)         \\
                                         & \to BelongsTo(personId, houseId)         \\
                                         & \to HasStatus(personId, statusId)        \\
                % & \to  Region(personId, regionName)                                \\
                                         & \to MemberOfCulture(personId, cultureId)
            \end{align*}}
    \end{adjustbox}
    \caption{Liste des contraintes de la base \emph{GOT}}
\end{figure}

\section{Social}

\subsection{Description}

\emph{Social} (\url{https://ldbcouncil.org/benchmarks/snb/}), ou \gls{ldbcsnb} est un graphe non attribué et généré artificiellement par le \glsfirst{ldbc} pour l'évaluation d'algorithmes sur les graphes.
Ce graphe représente un réseau social avec des collections de postes, des commentaires, etc.
Cette base est utilisée pour l'évaluation des systèmes de gestion de bases de données.
Les benchmarks sur la base \glsfirst{ldbcsnb} regroupes deux catégories :
\begin{description}
    \item[L'informatique décisionnelle] qui se concentre sur des requêtes complexes, comprenant beaucoup d'agrégation et de jointure touchant une grande partie du graphe avec des micro-lots d'opérations d'insertion/suppression.
    \item[Une utilisation interactive] qui capture un scénario de traitement transactionnel des graphes avec des requêtes de lecture complexes qui accèdent au voisinage d'un nœud donné dans le graphe et des opérations de mise à jour qui insèrent continuellement de nouvelles données dans le graphe.
\end{description}

\begin{figure}[H]
    \centering
    \begin{adjustbox}{max width=.9\linewidth, max height=.5\textheight}
        \begin{tikzpicture}[->,>=stealth',shorten >=1pt,auto,block/.style={rectangle,align=center},node distance=2.5cm]
            \node[draw] (forum) {Forum};
            \node[draw, right = 4cm of forum] (person) {Person};
            \node[draw, below = of person] (post) {Post};
            \node[draw, right = 4cm of person] (place) {Place};
            \node[draw, below = of place] (comment) {Comment};
            \node[draw, above = of place] (org) {Org};
            \node[draw, below = of forum] (tag) {Tag};
            \node[draw, below = of tag] (tagclass) {TagClass};

            \path
            (forum) edge node [below left, near end] {ContainerOf} (post)
            (forum) edge [bend left=10] node [above] {HasMember} (person)
            (forum) edge [bend right=10] node [below] {HasModerator} (person)
            (forum) edge node [right] {HasTag} (tag)
            (post) edge [bend left=15] node [left, near end] {HasCreator} (person)
            (post) edge [bend right=100,looseness=2.5,in=300,out=270] node [above left] {IsLocatedIn} (place)
            (post) edge node {HasTag} (tag)
            (comment) edge node [below] {ReplyOf} (post)
            (comment) edge [loop right] node [below left=.5cm] {ReplyOf} (comment)
            (comment) edge [bend right=10] node [above right] {HasCreator} (person)
            (comment) edge node [right] {IsLocatedIn} (place)
            (comment) edge [bend left] node [below] {HasTag} (tag)
            (person) edge [loop above] node {Knows} (person)
            (person) edge [bend right=80,looseness=2,in=280] node [below right] {HasInterest} (tag)
            (person) edge [bend left=10] node [above left] {WorkAt} (org)
            (person) edge [bend right=10] node [below right] {StudyAt} (org)
            (person) edge [bend left=15] node [right,near end] {Likes} (post)
            (person) edge [bend right=10] node [below left,near end] {Likes} (comment)
            (person) edge node [above] {IsLocatedIn} (place)
            (org) edge node [below right] {IsLocatedIn} (place)
            (tag) edge node [right] {HasType} (tagclass)
            (tagclass) edge [loop right] node {IsSubclassOf} (tagclass)
            (place) edge [loop right] node {IsPartOf} (place)
            ;
        \end{tikzpicture}
    \end{adjustbox}
    \caption{Schéma de la base \emph{Social}}
\end{figure}

\subsection{Prédicats et contraintes}

\begin{table}[H]
    \centering
    \begin{adjustbox}{max width=\linewidth}
        \begin{tabular}{l|c|l|l}
            Prédicat       & Arité & Termes                   & Description                                              \\
            \hline
            \hline
            $Forum$        & 1     & $forum$                  & Un forum qui contient un ensemble de posts               \\
            $Post$         & 1     & $post$                   & Un poste dans un forum                                   \\
            $Comment$      & 1     & $comment$                & Un commentaire à un poste                                \\
            $Tag$          & 1     & $tag$                    & Un label apposé sur un poste ou un forum                 \\
            $TagClass$     & 1     & $class$                  & Un type de label                                         \\
            $Person$       & 1     & $person$                 & Une personne                                             \\
            $Organisation$ & 1     & $org$                    & Une entreprise, un organisme, une université             \\
            $Place$        & 1     & $place$                  & Un lieu ville/région/pays/continent                      \\
            $HasCreator$   & 2     & $commentOrPost, person$  & Lie un commentaire ou un poste a son auteur              \\
            $ContainerOf$  & 2     & $forum, post$            & Lie un poste au forum dans lequel il est présent         \\
            $ReplyOf$      & 2     & $comment, commentOrPost$ & Lie un commentaire au poste/commentaire auquel il répond \\
            $IsLocatedIn$  & 2     & $x, place$               & Donne la localisation d'un objet ou d'une personne       \\
            $Knows$        & 2     & $person1, person2$       & Deux personnes se connaissent                            \\
            $HasInterest$  & 2     & $person, tag$            & Une personne qui est intéressée par un topique           \\
            $HasMember$    & 2     & $forul, person$          & Une personne est membre d'un forum                       \\
            $WorkAt$       & 2     & $person, org$            & L'emploi d'une personne                                  \\
            $StudyAt$      & 2     & $person, org$            & Les études d'une personne                                \\
            $IsSubclassOf$ & 2     & $subClass, superClass$   & Une relation hiérarchique entre les types de labels      \\
            $IsPartOf$     & 2     & $subPlace, place$        & Une relation d'inclusion entre les lieux                 \\
            $HasType$      & 2     & $tag, superClass$        & Lie un label a son type                                  \\
        \end{tabular}
    \end{adjustbox}
    \caption{Description de la base \emph{Social}}
\end{table}

\begin{figure}[H]
    \centering
    \begin{adjustbox}{max width=\linewidth,max height=.5\textheight}
        \parbox{\linewidth}{\begin{align*}
                Comment(comment)                  & \to HasCreator(comment, person)     \\
                                                  & \to ReplyOf(comment, commentOrPost) \\
                                                  & \to IsLocatedIn(comment, place)     \\
                ReplyOf(comment, commentOrPost)   & \to Comment(comment)                \\
                Post(post)                        & \to ContainerOf(forum, post)        \\
                                                  & \to HasCreator(post, person)        \\
                                                  & \to IsLocatedIn(post, place)        \\
                ContainerOf(forum, post)          & \to Forum(forum)                    \\
                                                  & \to Post(post)                      \\
                HasMember(forum, person)          & \to Forum(forum)                    \\
                                                  & \to Person(person)                  \\
                HasModerator(forum, person)       & \to Forum(forum)                    \\
                                                  & \to Person(person)                  \\
                HasCreator(commentOrPost, person) & \to Person(person)                  \\
                HasInterest(person, tag)          & \to Person(person)                  \\
                                                  & \to Tag(tag)                        \\
                Knows(person, other)              & \to Person(person)                  \\
                                                  & \to Person(other)                   \\
                                                  & \to Knows(other, person)            \\
                WorkAt(person, org)               & \to Person(person)                  \\
                                                  & \to Organisation(org)               \\
                Likes(person, commentOrPost)      & \to Person(person)
            \end{align*}}
    \end{adjustbox}
    \caption{Liste des contraintes de la base \emph{Social}}
\end{figure}

\begin{figure}[H]
    \ContinuedFloat
    \centering
    \begin{adjustbox}{max width=\linewidth,max height=.5\textheight}
        \parbox{\linewidth}{\begin{align*}
                StudyAt(person, org)                                             & \to Person(person)                     \\
                                                                                 & \to Organisation(org)                  \\
                Organisation(org)                                                & \to IsLocatedIn(org, place)            \\
                TagClass(class)                                                  & \to IsSubclassOf(class, `tagclass:0`)  \\
                IsSubclassOf(subClass, superClass)                               & \to TagClass(subClass)                 \\
                                                                                 & \to TagClass(superClass)               \\
                HasTag(something, tag)                                           & \to Tag(tag)                           \\
                Tag(tag)                                                         & \to HasType(tag, `tagclass:0`)         \\
                HasType(tag, class)                                              & \to Tag(tag)                           \\
                                                                                 & \to TagClass(class)                    \\
                IsLocatedIn(something, place)                                    & \to Place(place)                       \\
                Place(place)                                                     & \to IsPartOf(place, `place:0`)         \\
                IsPartOf(subPlace, place)                                        & \to Place(subPlace)                    \\
                                                                                 & \to Place(place)                       \\
                Post(post), HasCreator^-(post, person), ContainerOf(forum, post) & \to HasMember(forum, person)           \\
                IsSubclassOf^-(subClass, class), IsSubclassOf(class, superClass) & \to IsSubclassOf(subClass, superClass) \\
                HasType(tag, subClass)^-, IsSubclassOf(subClass, superClass)     & \to HasType(tag, superClass)
            \end{align*}}
    \end{adjustbox}
    \caption{Liste des contraintes de la base \emph{Social}}
\end{figure}
